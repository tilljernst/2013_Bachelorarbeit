%%%%%%%%%%%%%%%%%%%%%%%%%%%%%%%%%%%%%%%%%%%%%%%%%%%%%%%%%%%%%%%%%
%_____________ ___    _____  __      __ 
%\____    /   |   \  /  _  \/  \    /  \  Institute of Applied
%  /     /    ~    \/  /_\  \   \/\/   /  Psychology
% /     /\    Y    /    |    \        /   Zuercher Hochschule 
%/_______ \___|_  /\____|__  /\__/\  /    fuer Angewandte Wissen.
%        \/     \/         \/      \/                           
%%%%%%%%%%%%%%%%%%%%%%%%%%%%%%%%%%%%%%%%%%%%%%%%%%%%%%%%%%%%%%%%%
%
% Project     : Bachelorarbeit
% Title       : Header
% File        : header.tex Rev. 00
% Date        : 06.12.2013
% Author      : Till J. Ernst
%
%%%%%%%%%%%%%%%%%%%%%%%%%%%%%%%%%%%%%%%%%%%%%%%%%%%%%%%%%%%%%%%%%

\documentclass[
	appendixprefix,
	numbers=noenddot,
	oneside,
	12pt,
	parskip=half,
	ngerman,
	version=first
]{scrreprt}
%article scrartcl
%report scrreprt / report
%book scrbook
%letter scrlttr2

%***********************************************************************
% include some libs
%***********************************************************************
\usepackage[utf8]{inputenc}
\usepackage[T1]{fontenc}
\usepackage{textcomp}
\usepackage{listings}
\usepackage{color}
\usepackage{fancyhdr}
\usepackage{rotating}
\usepackage{titlesec}
\usepackage{mathptmx} 
\usepackage[scaled=.90]{helvet}
%\usepackage{courier}
\usepackage[urw-garamond]{mathdesign}
\usepackage{setspace} %Zeilenabstand
\onehalfspacing % 1,5 Zeilenabstand
%\renewcommand*\familydefault{\sfdefault} %% Only if the base font of the document is to be sans serif
\usepackage[]{ragged2e}
\usepackage[ngerman]{babel}
\usepackage[babel]{csquotes}
\usepackage[squaren]{SIunits}
\usepackage{graphicx}
\usepackage{url}
\usepackage[a4paper]{geometry}
\usepackage[absolute]{textpos}
\usepackage{makeidx}
\usepackage{colortbl}
\usepackage{pdflscape}
\usepackage{pdfpages}
\usepackage{tabularx}
\usepackage{lmodern}
\usepackage{longtable}
\usepackage{multirow}
\usepackage{array}
\usepackage{float}
\usepackage{scrhack}
\usepackage{wallpaper}
\usepackage{titleref}
\usepackage{tocloft} % fancy TOC


% Bereitstellung Hyperlinkfunktionen (PDF) (muss als letztes Paket geladen werden)
\usepackage[
	colorlinks=true,
	breaklinks=true,
	linkcolor=black,
	citecolor=black,
	urlcolor=black,
	anchorcolor=black,
	pdfpagelabels,
	pdftitle={Beeinflusst Multitasking das eigene Glück},
	pdfsubject={Die Auswirkungen von Medien-Multitasking auf das subjektive Wohlbefinden von Studierenden},
	pdfkeywords={KEYWORDS},
	pdfauthor={Till J. Ernst (ernsttil)}
]{hyperref}

%version 1 working
%Biblatex APA 6 Style
\usepackage{apacite}

%\usepackage[american]{babel}
%\usepackage{csquotes}
%\usepackage[
%    style=apa,
%    backend=biber
%]{biblatex}
%\DeclareLanguageMapping{german}{german-apa}
%\addbibresource{Bibliography.bib}

% Bereitstellen Anhang
\usepackage[toc,page]{appendix}

% Bereitstellung Glossar
\usepackage[acronym]{glossaries}
\makeglossaries


%***********************************************************************
% various styles
%**********************************************************************
%create index
\makeindex

%define pagestyle
\pagestyle{fancy}

%define page margin
\geometry{a4paper, top=25mm, left=25mm, right=25mm, bottom=25mm,headsep=10mm,footskip=10mm}

%textpos parameter
%\setlength{\TPHorizModule}{30mm}
%\setlength{\TPVertModule}{\TPHorizModule}
%\textblockorigin{10mm}{10mm} % start everything near the top-left corner
%The \setlength command is used to set the value of a length command, len-cmd, which is specified as the first argument.
%\setlength{\headheight}{28pt}
%\setlength{\parindent}{0pt} % Bei Absatz neuer Einzug
\setlength{\emergencystretch}{1em}

%horizontal lines for titlepage 
\newcommand{\HRule}{\rule{\linewidth}{0.5mm}}

%reference to source items inlc source number
\newcommand{\srcref}[1]{\nameref{src:#1} \cite{#1}}

% Umdefinieren des Layouts (otional)
% - - - - - - - - - -
\fancyhf{} %alle Kopf- und Fußzeilenfelder bereinigen
%\fancyhead[OR,EL]{\rightmark} %die Section-Name
\fancyhead[L]{\leftmark} % Chapter-Name
%\fancyhead[OL,ER]{\leftmark} % Chapter-Name -> nur nötig bei 2 Seiten Layout
\renewcommand{\headrulewidth}{0.3pt}
\fancyfoot[C]{\textbf \Large \thepage} 
\renewcommand{\footrulewidth}{0pt}
\renewcommand{\arraystretch}{1.5} % für tabellen

%\fancyhead[LO,RE]{} %clear headings for contents 
%\fancyhead[RO,LE]{\nouppercase{\rightmark}} %right odd pages and left even pages
%\fancyhead[LO,RE]{\MakeUppercase{\leftmark}} %left odd pages and right even pages
%\fancyfoot[LE,RO]{\thepage} %page numbering
%\fancyfoot[C]{} %clear centered page numbering 

%define some colors
\definecolor{gray}{rgb}{0.95,0.95,0.95}
\definecolor{darkgray}{rgb}{0.4,0.4,0.4}
%listing colors
\definecolor{lgray}{RGB}{250,250,250}
\definecolor{lgreen}{RGB}{63,127,95}
\definecolor{lred}{RGB}{127,0,85}
\definecolor{lblue}{RGB}{42,0,255}

% Für Tabellen: Textausrichtung bei fester Spaltenbreite
\newcolumntype{L}[1]{>{\raggedright\arraybackslash}p{#1}} % linksbündig mit Breitenangabe
\newcolumntype{C}[1]{>{\centering\arraybackslash}p{#1}} % zentriert mit Breitenangabe
\newcolumntype{R}[1]{>{\raggedleft\arraybackslash}p{#1}} % rechtsbündig mit Breitenangabe


\makeatletter
\renewcommand\thechapter{}
\renewcommand{\thesection}{\arabic{chapter}.\arabic{section}}
%\renewcommand\thesubsection{\@arabic\c@section.\@arabic\c@subsection}
\makeatother

%***********************************************************************
% listing
%***********************************************************************

\lstset{		
		basicstyle=\small\ttfamily,
		frame=single,
		numbers=left,	
		numberstyle=\tiny,
		%firstnumber=auto,
		numberblanklines=true,
		captionpos=b,
		extendedchars=true,
		float=ht,
		showtabs=false,
		tabsize=2,
		showspaces=false,
		showstringspaces=false,
		breaklines=true,
		%prebreak=\Righttorque,
		backgroundcolor=\color{lgray},
		keywordstyle=\color{lred}\bfseries, 
		commentstyle=\color{lgreen}\ttfamily,
%		morekeywords={printstr, printhexln},
		stringstyle=\color{lblue},
		xleftmargin=0.5cm,
		xrightmargin=0.5cm
}

