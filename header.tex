%%%%%%%%%%%%%%%%%%%%%%%%%%%%%%%%%%%%%%%%%%%%%%%%%%%%%%%%%%%%%%%%%
%_____________ ___    _____  __      __ 
%\____    /   |   \  /  _  \/  \    /  \  Institute of Applied
%  /     /    ~    \/  /_\  \   \/\/   /  Psychology
% /     /\    Y    /    |    \        /   Zuercher Hochschule 
%/_______ \___|_  /\____|__  /\__/\  /    fuer Angewandte Wissen.
%        \/     \/         \/      \/                           
%%%%%%%%%%%%%%%%%%%%%%%%%%%%%%%%%%%%%%%%%%%%%%%%%%%%%%%%%%%%%%%%%
%
% Project     : Bachelorarbeit
% Title       : Header
% File        : header.tex Rev. 00
% Date        : 06.12.2013
% Author      : Till J. Ernst
%
%%%%%%%%%%%%%%%%%%%%%%%%%%%%%%%%%%%%%%%%%%%%%%%%%%%%%%%%%%%%%%%%%

\documentclass[
	appendixprefix,
	numbers=noenddot,
	oneside,
	12pt,
	parskip=half, % Absatzabstand
	headings=small, % header sind klein
	toc=flat, %
	ngerman,
	version=first
]{scrreprt}
%article scrartcl
%report scrreprt / report
%book scrbook
%letter scrlttr2

%***********************************************************************
% include some libs
%***********************************************************************
\usepackage[utf8]{inputenc}
\usepackage[T1]{fontenc}
\usepackage{textcomp}
\usepackage{listings}
\usepackage{color}
\usepackage{rotating}
\usepackage[compact]{titlesec} % space before and after sections
\usepackage{mathptmx} 
\usepackage[scaled=.90]{helvet}
%\usepackage{courier}
\usepackage[urw-garamond]{mathdesign}
\usepackage{setspace} %Zeilenabstand
\onehalfspacing % 1,5 Zeilenabstand
%\renewcommand*\familydefault{\sfdefault} %% Only if the base font of the document is to be sans serif
\usepackage[]{ragged2e}
\usepackage[ngerman]{babel}
\usepackage[babel]{csquotes}
\usepackage[squaren]{SIunits}
\usepackage{graphicx}
\usepackage{url}
\usepackage[a4paper]{geometry}
\usepackage[absolute]{textpos}
\usepackage{makeidx}
\usepackage{colortbl}
\usepackage{pdflscape}
\usepackage{pdfpages}
\usepackage{tabularx}
\usepackage{lmodern}
\usepackage{longtable}
\usepackage{multirow}
\usepackage{array}
\usepackage{float}
\usepackage{scrhack}
\usepackage{wallpaper}
\usepackage{titleref}
\usepackage{tocloft} % fancy TOC
\usepackage{scrpage2} % KOMA script für Header und Footer

% Bereitstellen Anhang
\usepackage[toc,page]{appendix}


% Bereitstellung Hyperlinkfunktionen (PDF) (muss als letztes Paket geladen werden)
\usepackage[
	colorlinks=true,
	breaklinks=true,
	linkcolor=black,
	citecolor=black,
	urlcolor=black,
	anchorcolor=black,
	pdfpagelabels,
	pdftitle={Einfluss von Medien-Multitasking auf das SWB},
	pdfsubject={Die Auswirkungen von Medien-Multitasking auf das subjektive Wohlbefinden von Studierenden},
	pdfkeywords={KEYWORDS},
	pdfauthor={Till J. Ernst (ernsttil)}
]{hyperref}

%version 1 working
%Biblatex APA 6 Style
\usepackage[nosectionbib]{apacite} 
% notocbib - % erscheint nicht direkt im TOC
% nosectionbib - % Bibliography is a chapter

% Bereitstellung Glossar
\usepackage[nopostdot]{glossaries} %[nopostdot] - kein Punkt am Ende der Beschreibung
\makeglossaries


%***********************************************************************
% Eigenen Formatierung
%**********************************************************************
%create index
\makeindex

% -----------------------------------------------------------------
% TOC 
% -----------------------------------------------------------------
% toc deep
\setcounter{tocdepth}{1} % bis und mit unterkapitel



% -----------------------------------------------------------------
% Header und Footer
% -----------------------------------------------------------------
\pagestyle{scrheadings} % KOMA script
\clearscrheadfoot % formatierung zurücksetzen
\lohead[Einfluss von Medien-Multitasking auf das subjektive Wohlbefinden]{Einfluss von Medien-Multitasking auf das subjektive Wohlbefinden}
\rohead[\pagemark]{\pagemark} % seitenzahl im kopf rechts (in [] für Chapter)
\cfoot{} % seitenzahl in der mitte unten entfernen
\setkomafont{pageheadfoot}{\normalfont} % Formatierung von Header Beschriftung
\setlength{\headheight}{1.1\baselineskip}

% -----------------------------------------------------------------
% Titel und Kapitel anpassen
% -----------------------------------------------------------------

% Chapter (see package titlesec)
%----------
%\titleformat{hcommandi}[hshapei]{hformati}{hlabeli}{hsepi}{hbefore-codei}[hafter-codei]
%\titlespacing*{hcommandi}{hlefti}{hbefore-sepi}{hafter-sepi}[hright-sepi]
% Titelformat
\titleformat{\chapter}[display] {\normalfont\LARGE\bfseries\centering}{}{0pt}{}
% Abstand vor und nach dem Titel-Chapter
\titlespacing*{\chapter}{0pt}{0pt}{0pt}

% Section
%----------
\titleformat{\section}[hang] {\normalfont\large\bfseries}{}{0pt}{} %{\thesection}
\titlespacing*{\section}{0pt}{8pt}{0pt}

% SubSection
%----------
\titleformat{\subsection}[runin]{\normalfont\normalsize\bfseries}{}{25pt}{}[\,\,] %\thesubsection
\titlespacing*{\subsection}{0pt}{5pt}{0pt}

% SubsubSection
%----------
\titleformat{\subsubsection}[runin]{\normalfont\normalsize\bfseries\itshape}{}{}{\,\,\,\,\,\,\,\,\,\,}[\,\,] %\thesubsection
\titlespacing*{\subsubsection}{0pt}{0pt}{0pt}


% -----------------------------------------------------------------
%define page margin
% -----------------------------------------------------------------
\geometry{a4paper, top=25.4mm, left=25.4mm, right=25.4mm, bottom=25.4mm,headsep=10mm,footskip=10mm}
% headsep - Legt den Abstand zwischen der Kopfzeile und dem Rumpf der Seite fest.
% footskip - Legt den Abstand zwischen dem Rumpf der Seite und der zugehörigen Fußzeile fest.


% -----------------------------------------------------------------
%define text körper
% -----------------------------------------------------------------
\setlength{\emergencystretch}{1em}
\setlength{\parindent}{4em}


% -----------------------------------------------------------------
% Für Tabellen: Textausrichtung bei fester Spaltenbreite
% -----------------------------------------------------------------
\newcolumntype{L}[1]{>{\raggedright\arraybackslash}p{#1}} % linksbündig mit Breitenangabe
\newcolumntype{C}[1]{>{\centering\arraybackslash}p{#1}} % zentriert mit Breitenangabe
\newcolumntype{R}[1]{>{\raggedleft\arraybackslash}p{#1}} % rechtsbündig mit Breitenangabe

%***********************************************************************
% listing
%***********************************************************************

\lstset{		
		basicstyle=\small\ttfamily,
		frame=single,
		numbers=left,	
		numberstyle=\tiny,
		%firstnumber=auto,
		numberblanklines=true,
		captionpos=b,
		extendedchars=true,
		float=ht,
		showtabs=false,
		tabsize=2,
		showspaces=false,
		showstringspaces=false,
		breaklines=true,
		%prebreak=\Righttorque,
		backgroundcolor=\color{lgray},
		keywordstyle=\color{lred}\bfseries, 
		commentstyle=\color{lgreen}\ttfamily,
%		morekeywords={printstr, printhexln},
		stringstyle=\color{lblue},
		xleftmargin=0.5cm,
		xrightmargin=0.5cm
}

