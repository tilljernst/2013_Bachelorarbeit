%%%%%%%%%%%%%%%%%%%%%%%%%%%%%%%%%%%%%%%%%%%%%%%%%%%%%%%%%%%%%%%%%
%_____________ ___    _____  __      __ 
%\____    /   |   \  /  _  \/  \    /  \  Institute of Applied
%  /     /    ~    \/  /_\  \   \/\/   /  Psychology
% /     /\    Y    /    |    \        /   Zuercher Hochschule 
%/_______ \___|_  /\____|__  /\__/\  /    fuer Angewandte Wissen.
%        \/     \/         \/      \/                           
%%%%%%%%%%%%%%%%%%%%%%%%%%%%%%%%%%%%%%%%%%%%%%%%%%%%%%%%%%%%%%%%%
%
% Project     : Bachelorarbeit
% Title       : Abstract
% File        : abstract.tex Rev. 00
% Date        : 06.12.2013
% Author      : Till J. Ernst
%
%%%%%%%%%%%%%%%%%%%%%%%%%%%%%%%%%%%%%%%%%%%%%%%%%%%%%%%%%%%%%%%%%
\thispagestyle{empty} 
\let\raggedsection\centering
\chapter*{Abstract}\label{label.abstract}
\let\raggedsection\raggedright 
\textbf{Hintergrund:} In den letzten Jahren wurde ein Anstieg in der Mediennutzung und ein starker Trend in der Häufigkeit von Medien-Multitasking, das simultane Betreiben von mehreren Medien, verzeichnet. Dieser Trend scheint sich weiter fortzusetzen und tritt in allen Altersschichten auf. Ein Grund dafür wird in den sich immer weiter entwickelnden technologischen Möglichkeiten wie Computer, Smartphone und Tablets gesehen. Diese lassen ein simultanes Bearbeiten mehrerer Aufgaben zu und fördern dieses Verhalten. Dieser Trend wirft die Fragen nach den Auswirkungen von Medien-Multitasking auf und welchen Einfluss dieses Verhalten auf kognitive sowie psychosoziale Faktoren hat. Die vorliegende Studie setzt sich mit der Fragestellung nach den Auswirkungen von Medien-Multitasking auf das subjektive Wohlbefinden und die Fähigkeit zur Aufmerksamkeitskontrolle auseinander.
\par 
\textbf{Methoden:} Zur Beantwortung der Fragestellung wurde ein Onlinefragebogen konstruiert, der die Mediennutzung und das Multitasking-Verhalten mittels Media-Use-Questionnaire (MUQ) erfasste. Neben den demographischen Daten wurde weiter die Fähigkeit zur Aufmerksamkeitskontrolle mittels Attentional Control Scale (ACS) und das subjektive Wohlbefinden erfasst. Das subjektive Wohlbefinden wurde mittels zweier Skalen ermittelt: Der Flourishing Scale (FS), für die Erfassung des menschlichen Aufblühens und der Skala für das Erfassen der positiven und negativen Erfahrungen - Scale of Positive and Negative Experience (SPANE).
\par 
\textbf{Resultate:} Zwischen dem Medien-Multitasking und dem subjektiven Wohlbefinden konnte ein geringer, negativer Zusammenhang festgestellt werden. Dieser Zusammenhang viel bei Personen die stark Medien-Multitasking betreiben am höchsten  aus. Bei Personen, die wenig Medien-Multitasking betreiben, konnte eine positive Auswirkung auf das subjektive Wohlbefinden festgestellt werden. Die Studie konnte keinen direkten Zusammenhang zwischen Medien-Multitasking und der Fähigkeit zur Aufmerksamkeitskontrolle erfassen. Jedoch scheint diese einen geringen bis mittleren Einfluss auf das subjektive Wohlbefinden zu haben.
\par 
\textbf{Schlussfolgerung:} Wenig ist bekannt zwischen den Auswirkungen von Medien-Multitasking auf das subjektive Wohlbefinden. Im Gegensatz zu anderen Studien legt die vorliegende Untersuchung nahe, dass es einen Zusammenhang zwischen diesen Konstrukten geben könnte. Welche Faktoren diesen Zusammenhang beeinflussen und welche weiteren Konstrukte eine Rolle spielen sollte in weiteren Studien untersucht werden. 


