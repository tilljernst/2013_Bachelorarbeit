%%%%%%%%%%%%%%%%%%%%%%%%%%%%%%%%%%%%%%%%%%%%%%%%%%%%%%%%%%%%%%%%%
%_____________ ___    _____  __      __ 
%\____    /   |   \  /  _  \/  \    /  \  Institute of Applied
%  /     /    ~    \/  /_\  \   \/\/   /  Psychology
% /     /\    Y    /    |    \        /   Zuercher Hochschule 
%/_______ \___|_  /\____|__  /\__/\  /    fuer Angewandte Wissen.
%        \/     \/         \/      \/                           
%%%%%%%%%%%%%%%%%%%%%%%%%%%%%%%%%%%%%%%%%%%%%%%%%%%%%%%%%%%%%%%%%
%
% Project     : Bachelorarbeit
% Title       : Einleitung
% File        : einleitung Rev. 00
% Date        : 06.12.2013
% Author      : Till J. Ernst
%
%%%%%%%%%%%%%%%%%%%%%%%%%%%%%%%%%%%%%%%%%%%%%%%%%%%%%%%%%%%%%%%%%
\glsresetall
\let\raggedsection\centering 
\chapter{Einleitung (Gegenwart)}\label{chap.einleitung}
%\chapter*{Einleitung}\label{chap.einleitung}
%\addcontentsline{toc}{chapter}{Einleitung}
\let\raggedsection\raggedright 

%####################################################
% Kapitel Ausgangslage
%####################################################
\section{Ausgangslage}\label{section.ausgangslage}
Blablablbal

% Kapitel Hintergrund, Begründung und Ziel der Studie
%---------------------
\section{Hintergrund, Begründung und Ziel der Studie}\label{section.hintergrund}
%####################################################
% Kapitel Theoretischer Hintergrund
%####################################################
\section{Theoretischer Hintergrund}\label{section.theoHintegrund}
% Begriffsbestimmungen, Definitionen ---------------------
\subsection{Begriffsbestimmungen, Definitionen}
\label{subsection.begriffsbestimmung}
\begin{itemize}
    \item Media und Technologie \cite{Lin20091}
    \item Working Memory \cite{Mayer2003} and cognitive overload
    \item Cognitive Control
\end{itemize}

% Medien-Multitasking ---------------------
\subsection{Medien-Multitasking}\label{subsection.medienMultitasking}
\begin{itemize}
    \item Flynn Effekt
\end{itemize}
Unter Medien-Multitasking wird das Erledigen von mehr als einer Medienaktivität zur selben Zeit verstanden \cite{Foehr2006}. Es scheint, als ob neue Technologien wie Computer und mobile Endgeräte das obsessive Wechseln zwischen verschiedenen Aufgaben fördern, wie zum Beispiel das Schreiben von Email, das Versenden von Instant-Nachrichten, das Lesen von Online-Zetischriften und andere computer basierte Tätigkeiten. Dies führt dazu, das Jugendliche mehr Zeit mit der Benützung von verschiedenen Medien gleichzeitig verbringen als je zuvor \cite{Kaiser2004}. \\
Medien-Multitasking ist eine spezifische Form von Multitasking, was im Grunde genommen das Erledigen vieler (aus dem Lateinischen von \textit{multi}) Aufgaben (aus dem Englischen von \textit{task}) zur gleichen Zeit bedeutet \cite{Spitzer2012}. Im Folgenden wird aus theoretischer Sicht auf dieses Verhalten eingegangen. In einem ersten Teil auf das Multitasking im allgemeinen und später auf das Medien-Multitasking im Speziellen.\\
Eine Untersuchung an Arbeitsplätzen in den USA hat ergeben, dass die Beschäftigten ungefähr jede dritte Minute einer Unterbrechung oder Ablenkung ausgesetzt sind und dass Personen, welche an einem Computer arbeiten, durchschnittlich acht verschiedene Fenster geöffnet haben \cite{Thompson2005}. Daraus geht hervor, dass ständige Ablenkungen, die in Form von Eindrücken und Informationen auf diese Personen eingehen, eine gewachsene Anforderung in der heutigen Zeit darstellt \cite{Klingberg2008}. Innerhalb der Psychologie und der Hirnforschung wurde festgestellt, dass die Probleme beim Multitasking auf eine einzige zentrale Beschränkung zurückzuführen ist: Die Fähigkeit, Informationen unmittelbar im Sinn zu behalten \cite{Klingberg2008}. Es gibt wenig Übereinstimmung in der neurologischen und psychologieschen Literatur, wenn es darum geht die Funktionsprinzipien des Gehirns zu definieren, wenn es darum geht mehr als eine Nachricht zu verarbeiten oder mehrere Aufgaben gleichzeitig zu erfüllen \cite{Kieras1997}. Die meisten Informationsverarbeitungstheorien sehen eine Limitierung der gleichzeitigen Informationsverarbeitung im Gehirn \cite{Kieras1997, Pashler2000}. Untersuchungen haben gezeigt, dass zwar zwei gleichzeitige Stimuli im Gehirn eingehen können, diese aber nicht simultan bearbeitet werden können \cite{Pashler2000}. Diese Phänomen wird als \gls{labelPRP} bezeichnet \cite{Welford1952} und bezeichnet den Zeitintervall der zusätzlich vergeht, um einen Reiz zu verarbeiten, der unmittelbar nach einem vorhergehenden Reiz eingetroffen ist. Es wurden jedoch Reize gefunden, bei denen keine solche Verzögerung auftritt \cite{Foehr2006}. Forscher sind sich uneins, was genau die Minimierung verursacht. Viele schrebben diesen Flachenhals in der Verarbeitung der Auffindung oder der Planungsphase zu, ohne jedoch genau zu wissen, wie simultane Aufgaben verarbeitet werden \cite{Kieras1997}. Einige Forscher spekulieren auf eine zentrale Verarbeitungs-Einheit, die Aufgaben in einer Liste verwaltet. Andere gehen davon aus, dass es sich beim Gehirn um den limitierenden Faktor handelt, das nicht mehr als eine aktive Zuordnung einer Aufgabe bewerkstelligen kann \cite{Pashler2000}. Eine der Hauptbelastungen im Zusammenhang mit Multitasking hat mit Gehirn Ressourcen zu tun. Mittels Magnetresonanztomographie konnte bei Probanden, die mehrere Aufgaben gleichzeitig zu lösen hatten, nachgewiesen werden, dass das Aktivitätvolumen bei zwei simultan ausgeführten Aufgaben im Hirn signifikant kleiner war, als wenn diese beiden Aktivitäten zusammengezählt, unabhängig voneinander ausgeführt wurden \cite{Just2001, Klingberg1997}. Dieser Befund spricht für die Verarbeitung von verschiedenen Aufgaben im selben Bereich des Hirns \cite{Klingberg1997}, sowohl für die Trennung nach räumlicher Beziehung und semantischer Kategorisierung \cite{Just2001}.  \\
Eine Interpretation dieses Befundes lässt den Schluss zu, dass im Hirn eine Höchstgrenze an aktivem Hirngewebe zu einem Zeitpunkt vorhanden ist. Wenn also zwei Aufgaben simultan verarbeitet werden, werden reduzierte Ressourcen für je eine Aufgabe aufgewendet \cite{Just2001}. Eine andere Interpretation dieser Resultate schlägt eine Limitierung der Aufmerksamkeit vor, die eine Person erbringen kann \cite{Just2001}. Dieser Erklärungsmodell deckt sich auch mit derjenigen von \citeA{Lang2000} und anderen Informationsverarbeitungs-Theorien \cite{Kieras1997}.\\
Weitere Studien haben ergeben \cite{Grimes1990, Grimes1991} dass semantisch identische Informationen (wie zum Beispiel Audio- und Visuelle-Informationen in einer Informationssendung) von Benutzern gut gefolgt und deren Information mit Leichtigkeit wiedergegeben werden können (perzeptueller Gruppierungsvorgang). Wohingegen bei semantisch unterschiedliche Kanälen die Benutzer mehr Mühe haben sich auf einen Kanal zu konzentrieren und im Nachhinein Informationen abzurufen. \\
Der Name Medien-Multitasking umfasst eine Vielzahl von unterschiedlichen Aufgaben, die gleichzeitig verarbeitet werden. In vielen solcher gleichzeitigen Aufgaben geht es nicht darum, nicht ergänzende Nachrichten simultan zu verarbeiten, vielmehr wird zwischen den einzelnen Aufgaben hin und her gewechselt. Neurologen konnten das Gebiet im Gehirn für dieses Wechseln zwischen den Aufgaben identifizieren \cite{Wallis2006, Wood2003}, doch ist über die Auswirkung von diesem hin und her Wechseln bekannt. Der Fokus lag und liegt oftmals in der Identifizierung von negativen Effekten von Medien-Multitasking. Es wäre aber auch durchaus möglich, dass die Auseinadersetzung mit unteschiedlichen Medien gleichzeitig einen positiven Effekt haben könnte \cite{Foehr2006}.   

% Subjektives Wohlbefinden ---------------------
\subsection{Subjektives Wohlbefinden}\label{subsection.subjektivesWohlbefinden}


%####################################################
% Kapitel Bisherige Forschung
%####################################################
\section{Bisherige Forschung}\label{section.bisherigeForschung}


%####################################################
% Kapitel Fazit und Forschungslücke
%####################################################
\section{Fazit und Forschungslücke}\label{section.fazitLücke}

%####################################################
% Kapitel Fragestellung und Hypothesen
%####################################################
\section{Fragestellung und Hypothesen}\label{section.fragestellung}
Basierend auf den oben beschriebenen theoretischen Hintergründen und anhand der aufgezeigten Forschungslücken erschliesst sich die im Folgenden aufgelistete Fragestellung. Aus dieser Fragestellung erschliesst sich die Haupthypothese, die für eine verständlichere Darstellung in drei Arbeitshypothesen unterteilt wurde. Dies Aufteilung soll auch dazu dienen, einzelne Aspekte konkreter betrachten zu können. \par
\textbf{Fragestellung:} Beeinflusst die Häufigkeit und die Fähigkeit bezogen auf Medien-Multitasking das subjektive Wohlbefinden von Studierenden?\par
Aus dieser Fragestellung lassen sich folgende Hypothesen formulieren:\par
\textbf{Haupthypothese:}
Studierende, die am häufigsten Medien-Multitasking anwenden sind diejenigen, die am wenigsten Fähig sind Ablenkung abzublocken und somit Medien-Multitasking zu praktizieren. Dies wiederum wirkt sich negativ auf das subjektive Wohlbefinden dieser Studenten aus.\par
\textbf{Arbeitshypothese 1:} Zwischen der Häufigkeit wie Medien-Multitasking angewendet wird und dem subjektiven Wohlbefinden besteht ein negativer Zusammenhang. Je mehr Media-Multitasking angewendet wird, desto negativer wirkt sich dies auf das subjektive Wohlbefinden aus.\par
\textbf{Arbeitshypothese 2:} Zwischen der Fähigkeit Medien-Multitasking zu betreiben und dem subjektiven Wohlbefinden besteht ein positiver Zusammenhang. Fähige Studenten können besser zwischen schädlichem und nützlichen Medien-Multitasking unterscheiden und beeinflussen dadurch das eigene subjektive Wohlbefinden positiv.\par
\textbf{Arbeitshypothese 3:} Zwischen der Fähigkeit von Studierenden Medien-Multitasking zu betreiben und der Häufigkeit besteht ein negativer Zusammenhang. Je fähiger Studierende im Umgang Medien-Multitasking sind, desto weniger werden sie Medien-Multitasking anwenden.





