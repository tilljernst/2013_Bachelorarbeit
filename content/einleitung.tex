%%%%%%%%%%%%%%%%%%%%%%%%%%%%%%%%%%%%%%%%%%%%%%%%%%%%%%%%%%%%%%%%%
%_____________ ___    _____  __      __ 
%\____    /   |   \  /  _  \/  \    /  \  Institute of Applied
%  /     /    ~    \/  /_\  \   \/\/   /  Psychology
% /     /\    Y    /    |    \        /   Zuercher Hochschule 
%/_______ \___|_  /\____|__  /\__/\  /    fuer Angewandte Wissen.
%        \/     \/         \/      \/                           
%%%%%%%%%%%%%%%%%%%%%%%%%%%%%%%%%%%%%%%%%%%%%%%%%%%%%%%%%%%%%%%%%
%
% Project     : Bachelorarbeit
% Title       : Einleitung
% File        : einleitung Rev. 00
% Date        : 06.12.2013
% Author      : Till J. Ernst
%
%%%%%%%%%%%%%%%%%%%%%%%%%%%%%%%%%%%%%%%%%%%%%%%%%%%%%%%%%%%%%%%%%
\glsresetall
\let\raggedsection\centering 
\chapter{Einleitung (Gegenwart)}\label{chap.einleitung}
%\chapter*{Einleitung}\label{chap.einleitung}
%\addcontentsline{toc}{chapter}{Einleitung}
\let\raggedsection\raggedright 

% Kapitel Ausgangslage
%---------------------
\section{Ausgangslage}\label{section.ausgangslage}
Blablablbal

% Kapitel Hintegrund, Begründung und Ziel der Studie
%---------------------
\section{Hintegrund, Begründung und Ziel der Studie}\label{section.hintergrund}

% Kapitel Theoretischer Hintergrund
%---------------------
\section{Theoretischer Hintergrund}\label{section.theoHintegrund}
\subsection{Begriffsbestimmungen, Definitionen}
\label{subsection.begriffsbestimmung}
\begin{itemize}
    \item Media und Technologie \cite{Lin20091}
    \item Working Memory \cite{Mayer2003} and cognitive overload
    \item Cognitive Control
\end{itemize}

\subsection{Medien-Multitasking}\label{subsection.medienMultitasking}
\begin{itemize}
    \item Flynn Effekt
\end{itemize}


\subsection{Subjektives Wohlbefinden}\label{subsection.subjektivesWohlbefinden}

% Kapitel Bisherige Forschung
%---------------------
\section{Bisherige Forschung}\label{section.bisherigeForschung}

% Kapitel Fazit und Forschungslücke
%---------------------
\section{Fazit und Forschungslücke}\label{section.fazitLücke}

% Kapitel Fragestellung und Hypothesen
%---------------------
\section{Fragestellung und Hypothesen}\label{section.fragestellung}
Basierend auf den oben beschriebenen theoretischen Hintergründen und anhand der aufgezeigten Forschungslücken erschliesst sich die im Folgenden aufgelistete Fragestellung. Aus dieser Fragestellung erschliesst sich die Haupthypothese, die für eine verständlichere Darstellung in drei Arbeitshypothesen unterteilt wurde. Dies Aufteilung soll auch dazu dienen, einzelne Aspekte konkreter betrachten zu können. \par
\textbf{Fragestellung:} Beeinflusst die Häufigkeit und die Fähigkeit bezogen auf Medien-Multitasking das subjektive Wohlbefinden von Studierenden?\par
Aus dieser Fragestellung lassen sich folgende Hypothesen formulieren:\par
\textbf{Haupthypothese:}
Studierende, die am häufigsten Medien-Multitasking anwenden sind diejenigen, die am wenigsten Fähig sind Ablenkung abzublocken und somit Medien-Multitasking zu praktizieren. Dies wiederum wirkt sich negativ auf das subjektive Wohlbefinden dieser Studenten aus.\par
\textbf{Arbeitshypothese 1:} Zwischen der Häufigkeit wie Medien-Multitasking angewendet wird und dem subjektiven Wohlbefinden besteht ein negativer Zusammenhang. Je mehr Media-Multitasking angewendet wird, desto negativer wirkt sich dies auf das subjektive Wohlbefinden aus.\par
\textbf{Arbeitshypothese 2:} Zwischen der Fähigkeit Medien-Multitasking zu betreiben und dem subjektiven Wohlbefinden besteht ein positiver Zusammenhang. Fähige Studenten können besser zwischen schädlichem und nützlichen Medien-Multitasking unterscheiden und beeinflussen dadurch das eigene subjektive Wohlbefinden positiv.\par
\textbf{Arbeitshypothese 3:} Zwischen der Fähigkeit von Studierenden Medien-Multitasking zu betreiben und der Häufigkeit besteht ein negativer Zusammenhang. Je fähiger Studierende im Umgang Medien-Multitasking sind, desto weniger werden sie Medien-Multitasking anwenden.





