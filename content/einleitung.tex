%%%%%%%%%%%%%%%%%%%%%%%%%%%%%%%%%%%%%%%%%%%%%%%%%%%%%%%%%%%%%%%%%
%_____________ ___    _____  __      __ 
%\____    /   |   \  /  _  \/  \    /  \  Institute of Applied
%  /     /    ~    \/  /_\  \   \/\/   /  Psychology
% /     /\    Y    /    |    \        /   Zuercher Hochschule 
%/_______ \___|_  /\____|__  /\__/\  /    fuer Angewandte Wissen.
%        \/     \/         \/      \/                           
%%%%%%%%%%%%%%%%%%%%%%%%%%%%%%%%%%%%%%%%%%%%%%%%%%%%%%%%%%%%%%%%%
%
% Project     : Bachelorarbeit
% Title       : Einleitung
% File        : einleitung Rev. 00
% Date        : 06.12.2013
% Author      : Till J. Ernst
%
%%%%%%%%%%%%%%%%%%%%%%%%%%%%%%%%%%%%%%%%%%%%%%%%%%%%%%%%%%%%%%%%%
\glsresetall
\let\raggedsection\centering 
\chapter{Einleitung (Gegenwart)}\label{chap.einleitung}
%\chapter*{Einleitung}\label{chap.einleitung}
%\addcontentsline{toc}{chapter}{Einleitung}
\let\raggedsection\raggedright 

% Kapitel Ausgangslage
%---------------------
\section{Ausgangslage}\label{section.ausgangslage}
Blablablbal

% Kapitel Hintegrund, Begründung und Ziel der Studie
%---------------------
\section{Hintegrund, Begründung und Ziel der Studie}\label{section.hintergrund}

% Kapitel Theoretischer Hintergrund
%---------------------
\section{Theoretischer Hintergrund}\label{section.theoHintegrund}
\subsection{Begriffsbestimmungen, Definitionen}\label{subsection.begriffsbestimmung}
\subsection{Medien-Multitasking}\label{subsection.medienMultitasking}
\subsection{Subjektives Wohlbefinden}\label{subsection.subjektivesWohlbefinden}

% Kapitel Bisherige Forschung
%---------------------
\section{Bisherige Forschung}\label{section.bisherigeForschung}

% Kapitel Fazit und Forschungslücke
%---------------------
\section{Fazit und Forschungslücke}\label{section.fazitLücke}

% Kapitel Fragestellung und Hypothesen
%---------------------
\section{Fragestellung und Hypothesen}\label{section.fragestellung}
Basierend auf den oben beschriebenen theoretischen Hintergründen und anhand der aufgezeigten Forschungslücken erschliesst sich die im Folgenden aufgelistete Fragestellung. Für eine bessere Übersicht wurde die Fragestellung in eine Hauptfragestellung, eine Nebenfragestellung und drei Unterfragen unterteilt. Dies Aufteilung soll dazu dienen, einzelne Aspekte konkreter betrachten zu können. \par
\textbf{Hauptfragestellung:} Welchen Einfluss hat Medien-Multitasking auf das subjektive Wohlbefinden von Studierenden?\par
\textbf{Nebenfragestellung:} Welchen Einfluss hat die Aufmerksamkeitskontrolle bei Studierenden im Zusammenhang mit Medien-Multitasking auf das subjektive Wohlbefinden?\par
\textbf{Unterfragen:}
\begin{itemize}
    \item Wie wirkt sich das Geschlecht auf das Medien-Multitasking-Verhalten und das subjektive Wohlbefinden aus?
    \item Welche Rolle spielt das Alter bezüglich Medien-Multitasking?
    \item Gibt es Unterschiede im Medien-Multitasking-Verhalten bezüglich dem subjektive Wohlbefinden bei Digital Native und Digital Immigrants?
\end{itemize}
Für die Untersuchung dieser Fragestellung wurden im Fplgenden gemäss den oben genannte Fragen entsprechende Hypothesen formuliert:\par
\textbf{Haupthypothese:}
Studierende, die häufig Medien-Multitasking betreiben, haben ein tieferes subjektives Wohlbefinden als Studierende, die weniger Medien-Mulitasking betreiben.\par
\textbf{Nebenhypothese:}
Je häufiger Studierende Medien-Multitasking betreiben, desto tiefer ist deren Aufmerksamkeitskontrolle, was sich wiederum negativ auf das subjektive Wohlbefinden auswirkt.\par
\textbf{Unterhypothese 1:} Weibliche Studierende sind stärker von den Auswirkungen von Medien-Multitasking auf das subjektive Wohlbefinden betroffen, als ihre männlichen Kollegen.\par
\textbf{Unterhypothese 2:} Bei jüngeren Studierenden hat das Medien-Multitasking weniger Einfluss auf das subjektive Wohlbefinden als bei den älteren Studierenden.\par
\textbf{Unterhypothese 3:} Digital Natives haben eine höhere Aufmerksamkeitskontrolle bezüglich Medien-Multitasking als digital Immigrants. Was sich wiederum positiv auf das subjektive Wohlbefinden auswirkt.





