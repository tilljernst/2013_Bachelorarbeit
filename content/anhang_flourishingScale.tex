%%%%%%%%%%%%%%%%%%%%%%%%%%%%%%%%%%%%%%%%%%%%%%%%%%%%%%%%%%%%%%%%%
%_____________ ___    _____  __      __ 
%\____    /   |   \  /  _  \/  \    /  \  Institute of Applied
%  /     /    ~    \/  /_\  \   \/\/   /  Psychology
% /     /\    Y    /    |    \        /   Zuercher Hochschule 
%/_______ \___|_  /\____|__  /\__/\  /    fuer Angewandte Wissen.
%        \/     \/         \/      \/                           
%%%%%%%%%%%%%%%%%%%%%%%%%%%%%%%%%%%%%%%%%%%%%%%%%%%%%%%%%%%%%%%%%
%
% Project     : Bachelorarbeit
% Title       : 
% File        : Rev. 00
% Date        : 06.12.2013
% Author      : Till J. Ernst
%
%%%%%%%%%%%%%%%%%%%%%%%%%%%%%%%%%%%%%%%%%%%%%%%%%%%%%%%%%%%%%%%%%
\glsresetall

\let\raggedsection\centering 
\chapter{Übersetzung -- Flourishing Scale}\label{chap.appendix_fs}
\let\raggedsection\raggedright 
\begin{RaggedRight}
% Flourishing Scale
Die Skala für die Erfassung des menschlichen Wohls (engl.: Flourishing Scale - FS) \cite{Diener:2010} besteht aus einer einleitenden Anweisung und den zugehörenden Fragebogen-Elemente. Bewertet werden die Elemente auf einer siebenstufigen Skala: 1 = trifft überhaupt nicht zu (engl.: strongly disagree); 2 = trifft nicht zu (engl.: disagree); 3 = trifft kaum zu (engl.: slightly disagree); 4 = gemischt oder weder Zustimmung noch Ablehnung (engl.: mixed or neither agree nor disagree); 5 = trifft etwas zu (engl.: slightly agree); 6 = trifft zu (engl.: agree); 7 = trifft voll und ganz zu (engl.: strongly agree).

% Table
\begin{center}
    \begin{longtable}[t]{|p{15 cm}|}
    \caption{FS Anweisung - Übersetzung} \\ \hline
        \textbf{Englisch} \\ \hline
        Below are eight statements with which you may agree or disagree. Using the 1–7 scale below, indicate your agreement with each item by indicating that response for each statement. \\ \hline
        \textbf{Deutsch} \\ \hline 
        Unten findest Du acht Aussagen, denen Du zustimmen oder widersprechen kannst. Bewerte jede Aussage auf einer Skala von 1 bis 7 (1 = trifft überhaupt nicht zu bis 7 = trifft voll und ganz zu). \\ \hline   
    \end{longtable}
	\label{tab:FsAnweisung}
\end{center}

% Table
\begin{center}
    \begin{longtable}[t]{|p{0.8 cm}|p{6.6 cm}|p{6.6 cm}|}
    \caption{SPANE Elemente - Übersetzung} \\ \hline
        \textbf{Nr.} & \textbf{Englisch} & \textbf{Deutsch} \\ \hline
        \endfirsthead
        \hline
        \textbf{Nr.} & \textbf{Englisch} & \textbf{Deutsch} \\ \hline
        \endhead 
        & \multicolumn{2}{|c|}{Fortsetzung auf der nächsten Seite $...$ } \\ \hline
        \endfoot
        \hline
        \endlastfoot
        1 & I lead a purposeful and meaningful life. & Ich führe ein zielgerichtetes und sinnvolles Leben.\\
        2 & My social relationships are supportive and rewarding. & Meine sozialen Beziehungen stärken und bereichern mich. \\
        3 & I am engaged and interested in my daily activities. & Ich bin engagiert und interessiere mich für meine täglichen Aktivitäten.\\
        4 & I actively contribute to the happiness and well-being of others. & Ich trage aktiv zum Glück und zum Wohlbefinden anderer bei.\\
        5 & I am competent and capable in the activities that are important to me. & In den Aktivitäten, die mir wichtig sind, bin ich kompetent und fähig.\\
        6 & I am a good person and live a good life. & Ich bin ein guter Mensch und ich führe ein gutes Leben. \\
        7 & I am optimistic about my future. & Ich bin optimistisch, was meine Zukunft betrifft.\\
        8 & People respect me. & Die Leute respektieren mich.\\        
    \end{longtable}
	\label{tab:FsElemente}
\end{center}

\end{RaggedRight}
