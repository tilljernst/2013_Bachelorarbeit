%%%%%%%%%%%%%%%%%%%%%%%%%%%%%%%%%%%%%%%%%%%%%%%%%%%%%%%%%%%%%%%%%
%_____________ ___    _____  __      __ 
%\____    /   |   \  /  _  \/  \    /  \  Institute of Applied
%  /     /    ~    \/  /_\  \   \/\/   /  Psychology
% /     /\    Y    /    |    \        /   Zuercher Hochschule 
%/_______ \___|_  /\____|__  /\__/\  /    fuer Angewandte Wissen.
%        \/     \/         \/      \/                           
%%%%%%%%%%%%%%%%%%%%%%%%%%%%%%%%%%%%%%%%%%%%%%%%%%%%%%%%%%%%%%%%%
%
% Project     : Bachelorarbeit
% Title       : 
% File        : anhang.tex Rev. 00
% Date        : 06.12.2013
% Author      : Till J. Ernst
%
%%%%%%%%%%%%%%%%%%%%%%%%%%%%%%%%%%%%%%%%%%%%%%%%%%%%%%%%%%%%%%%%%
\glsresetall
%\pagenumbering{Roman}
%\appendix
\let\raggedsection\centering 
\mychapter{1}{Anhang A \\ Übersetzung der Fragebögen}\label{chap.anhang}
\let\raggedsection\raggedright 

% Attentional Control Scale
\section{Übersetzung des \textit{Media Use Questionnaire}}\label{appendix.acs}


% Attentional Control Scale
\section{Übersetzung des \textit{Attentional Control Scale}}\label{appendix.acs}
Der aus dem Englisch stammende Fragebogen \textit{Attentional Control Scale - (ACS)} \cite{Derryberry:2002} wurde für diese Arbeit ins Deutsche übersetzt. \newline
Die Bewertung der Elemente erfolgte mittels vierstufigen Skala: 1 = fast nie (engl. almost never); 2 = manchmal (engl. sometimes); 3 = oft (engl. often); 4 = immer (engl. always).\newline
Im Folgenden werden die einzelnen Elemente inkl. englischer Übersetzung aufgelistet. (R) steht für invertiert bewertete Elemente:
\begin{center}
    \begin{longtable}[t]{|l|p{6.6 cm}|p{6.6 cm}|}
    \caption{Attentional Control Scale - Übersetzung} \\ \hline
        \textbf{Nr.} & \textbf{Englisch} & \textbf{Deutsch} \\ \hline
        \endfirsthead
        \hline
        \textbf{Nr.} & \textbf{Englisch} & \textbf{Deutsch} \\ \hline
        \endhead 
        & \multicolumn{2}{|c|}{Fortsetzung auf der nächsten Seite $...$ } \\ \hline
        \endfoot
        \hline
        \endlastfoot
        1 & It’s very hard for me to concentrate on a difficult task when there are noises around. (R) & In einer lauten Umgebung habe ich Mühe, mich auf eine schwierige Aufgabe zu konzentrieren.\\ 
        2 & When I need to concentrate and solve a problem, I have trouble focusing my attention. (R) & Wenn mich auf die Lösung eines Problems konzentrieren muss, habe ich Mühe meine Aufmerksamkeit zu fokussieren.\\ 
        3 & When I am working hard on something, I still get distracted by events around me. (R) & Wenn ich mich intensiv mit etwas beschäftige, werde ich durch Ereignisse um mich herum abgelenkt.\\ 
        4 & My concentration is good even if there is music in the room around me. & Musik im gleichen Raum stört meine Konzentration nicht.\\
        5 & When concentrating, I can focus my attention so that I become unaware of what’s going on in the room around me. & Wenn ich mich auf etwas konzentriere,  so kann ich meine Aufmerksamkeit derart fokussieren, dass ich alles um mich herum vergesse. \\
        6 & When I am reading or studying, I am easily distracted if there are people talking in the same room. (R) & Wenn ich lesen oder etwas lernen muss, lasse ich mich leicht ablenken, wenn Leute im selben Raum miteinander sprechen. \\
        7 & When trying to focus my attention on something, I have difficulty blocking out distracting thoughts. (R) & Wenn ich meine Aufmerksamkeit auf etwas zu konzentrieren versuche, so habe ich Schwierigkeiten ablenkende Gedanken abzublocken. \\
        8 & I have a hard time concentrating when I’m excited about something. (R) & Ich habe Mühe, mich zu konzentrieren, wenn ich aufgeregt bin. \\
        9 & When concentrating I ignore feelings of hunger or thirst. & Wenn ich mich konzentriere vergesse ich, dass ich durstig oder hungrig bin. \\
        10 & I can quickly switch from one task to another. & Ich kann rasch von einer Aufgabe zur nächsten Aufgabe wechseln. \\
        11 & It takes me a while to get really involved in a new task. (R) & Es dauert eine Weile, bis ich mich in eine neue Aufgabe eingearbeitet habe. \\
        12 & It is difficult for me to coordinate my attention between the listening and writing required when taking notes during lectures. (R) & Es bereitet mir Schwierigkeiten, meine Aufmerksamkeit zwischen dem Zuhören und dem Niederschreiben von Informationen während der Vorlesung zu koordinieren. \\
        13 & I can become interested in a new topic very quickly when I need to. & Wenn erforderlich, kann ich mich leicht für ein neues Thema begeistern.  \\
        14 & It is easy for me to read or write while I’m also talking on the phone. & Es fällt mir leicht, während eines Telefonats gleichzeitig zu lesen oder zu schreiben.  \\
        15 & I have trouble carrying on two conversations at once. (R) & Es bereitet mir Mühe, zwei verschiedene Gespräche gleichzeitig zu führen. \\
        16 & I have a hard time coming up with new ideas quickly. (R) & Es fällt mir schwer, spontan auf neue Ideen zu kommen. \\
        17 & After being interrupted or distracted, I can easily shift my attention back to what I was doing before. & Nachdem ich abgelenkt oder unterbrochen wurde, kann ich meine Aufmerksamkeit mühelos auf die vorherige Arbeit lenken. \\
        18 & When a distracting thought comes to mind, it is easy for me to shift my attention away from it. & Wenn mir ein störender Gedanke in den Sinn kommt, so kann ich ihn ohne grosse Mühe wieder vergessen.  \\
        19 & It is easy for me to alternate between two different tasks. & Es fällt mir leicht, zwischen zwei unterschiedlichen Tätigkeiten hin und her zu wechseln. \\
        20 & It is hard for me to break from one way of thinking about something and look at it from another point of view. (R) & Es fällt mir schwer, mich von meiner bisherigen Denkweise zu lösen und einen Sachverhalt von einer anderen Seite zu betrachten. \\ \hline
    \end{longtable}
	\label{tab:AcsUebersetzung}
\end{center}

