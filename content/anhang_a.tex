%%%%%%%%%%%%%%%%%%%%%%%%%%%%%%%%%%%%%%%%%%%%%%%%%%%%%%%%%%%%%%%%%
%_____________ ___    _____  __      __ 
%\____    /   |   \  /  _  \/  \    /  \  Institute of Applied
%  /     /    ~    \/  /_\  \   \/\/   /  Psychology
% /     /\    Y    /    |    \        /   Zuercher Hochschule 
%/_______ \___|_  /\____|__  /\__/\  /    fuer Angewandte Wissen.
%        \/     \/         \/      \/                           
%%%%%%%%%%%%%%%%%%%%%%%%%%%%%%%%%%%%%%%%%%%%%%%%%%%%%%%%%%%%%%%%%
%
% Project     : Bachelorarbeit
% Title       : 
% File        : anhang_a.tex Rev. 00
% Date        : 06.12.2013
% Author      : Till J. Ernst
%
%%%%%%%%%%%%%%%%%%%%%%%%%%%%%%%%%%%%%%%%%%%%%%%%%%%%%%%%%%%%%%%%%
\glsresetall

\let\raggedsection\centering 
\chapter{Anhang A}\label{chap.appendix_a}
\let\raggedsection\raggedright 
\begin{RaggedRight}

% Attentional Control Scale
\section*{Media Use Questionnaire}\label{appendix.muq}
Für die Erfassung des \textit{Media Multitasking Index (MMI)} von \citeA{Ophir2009} wurde der aus dem englisch sprachigen Raum stammende Fragebogen \textit{Media Use Questionnaire} aus dem Englischen ins Deutsche übersetzt.
Die Bewertung der Medien-Multitasking-Matrix (siehe \nameref{section.erhebungsinstrumente}) erfolgt mittels folgender Skala: 1 = meistens (engl.: most of the time); 0.67 = etwas (engl.: some of the time); 0.33 = wenig (engl.: a little of the time); 0 = nie (engl.: never).  \\

%Tabelle
\begin{center}
    \begin{longtable}[t]{|l|p{6.6 cm}|p{6.6 cm}|}
    \caption{Medie Use Questionnaire - Medienübersetzung} \\ \hline
        \textbf{Nr.} & \textbf{Englisch} & \textbf{Deutsch} \\ \hline
        \endfirsthead
        \hline
        \textbf{Nr.} & \textbf{Englisch} & \textbf{Deutsch} \\ \hline
        \endhead 
        & \multicolumn{2}{|c|}{Fortsetzung auf der nächsten Seite $...$ } \\ \hline
        \endfoot
        \hline
        \endlastfoot
        1 & print media & Druckmedien \\
        2 & television & Fernsehen \\
        3 & computer-based video (such as YouTube or online television episodes) & Online Video (wie zum Beispiel Youtube) \\
        4 & music & Musik \\
        5 & nonmusic audio & Nicht-musikalische Audiomedien (z.B. Hörbücher, Podcast, etc.) \\
        6 & video or computer games & Video oder Computer Games \\
        7 & telephone and mobile phone voice calls & Telefonieren (Mobil- und / oder Festnetz) \\
        8 & instant messaging & Instant Messaging (z.B. Skype, Windows Live Messenger, Yahoo Messenger, etc.) \\
        9 & SMS (text messaging) & SMS (Textnachrichten) \\
        10 & email & Email \\
        11 & web surfing & Internet-Surfen \\
        12 & other computer-based application  & Andere Computerbasierte-Tätigkeiten (z.B. Word, Videobearbeitung, programmieren, etc.) \\
        \end{longtable}
	\label{tab.muqUebersetzung}
\end{center}


% Attentional Control Scale
\section{Attentional Control Scale}\label{appendix.acs}
Die Bewertung der Elemente des aus dem Englisch stammenden Fragebogens \textit{Attentional Control Scale - (ACS)} \cite{Derryberry2002} erfolgte mittels vierstufigen Skala: 1 = fast nie (engl.: almost never); 2 = manchmal (engl.: sometimes); 3 = oft (engl.: often); 4 = immer (engl.: always).\newline
Im Folgenden werden die einzelnen Elemente inkl. englischer Übersetzung aufgelistet. (R) steht für invertiert bewertete Elemente:
\begin{center}
    \begin{longtable}[t]{|l|p{6.6 cm}|p{6.6 cm}|}
    \caption{Attentional Control Scale - Übersetzung} \\ \hline
        \textbf{Nr.} & \textbf{Englisch} & \textbf{Deutsch} \\ \hline
        \endfirsthead
        \hline
        \textbf{Nr.} & \textbf{Englisch} & \textbf{Deutsch} \\ \hline
        \endhead 
        & \multicolumn{2}{|c|}{Fortsetzung auf der nächsten Seite $...$ } \\ \hline
        \endfoot
        \hline
        \endlastfoot
        1 & It’s very hard for me to concentrate on a difficult task when there are noises around. (R) & In einer lauten Umgebung habe ich Mühe, mich auf eine schwierige Aufgabe zu konzentrieren.\\ 
        2 & When I need to concentrate and solve a problem, I have trouble focusing my attention. (R) & Wenn mich auf die Lösung eines Problems konzentrieren muss, habe ich Mühe meine Aufmerksamkeit zu fokussieren.\\ 
        3 & When I am working hard on something, I still get distracted by events around me. (R) & Wenn ich mich intensiv mit etwas beschäftige, werde ich durch Ereignisse um mich herum abgelenkt.\\ 
        4 & My concentration is good even if there is music in the room around me. & Musik im gleichen Raum stört meine Konzentration nicht.\\
        5 & When concentrating, I can focus my attention so that I become unaware of what’s going on in the room around me. & Wenn ich mich auf etwas konzentriere,  so kann ich meine Aufmerksamkeit derart fokussieren, dass ich alles um mich herum vergesse. \\
        6 & When I am reading or studying, I am easily distracted if there are people talking in the same room. (R) & Wenn ich lesen oder etwas lernen muss, lasse ich mich leicht ablenken, wenn Leute im selben Raum miteinander sprechen. \\
        7 & When trying to focus my attention on something, I have difficulty blocking out distracting thoughts. (R) & Wenn ich meine Aufmerksamkeit auf etwas zu konzentrieren versuche, so habe ich Schwierigkeiten ablenkende Gedanken abzublocken. \\
        8 & I have a hard time concentrating when I’m excited about something. (R) & Ich habe Mühe, mich zu konzentrieren, wenn ich aufgeregt bin. \\
        9 & When concentrating I ignore feelings of hunger or thirst. & Wenn ich mich konzentriere vergesse ich, dass ich durstig oder hungrig bin. \\
        10 & I can quickly switch from one task to another. & Ich kann rasch von einer Aufgabe zur nächsten Aufgabe wechseln. \\
        11 & It takes me a while to get really involved in a new task. (R) & Es dauert eine Weile, bis ich mich in eine neue Aufgabe eingearbeitet habe. \\
        12 & It is difficult for me to coordinate my attention between the listening and writing required when taking notes during lectures. (R) & Es bereitet mir Schwierigkeiten, meine Aufmerksamkeit zwischen dem Zuhören und dem Niederschreiben von Informationen während der Vorlesung zu koordinieren. \\
        13 & I can become interested in a new topic very quickly when I need to. & Wenn erforderlich, kann ich mich leicht für ein neues Thema begeistern.  \\
        14 & It is easy for me to read or write while I’m also talking on the phone. & Es fällt mir leicht, während eines Telefonats gleichzeitig zu lesen oder zu schreiben.  \\
        15 & I have trouble carrying on two conversations at once. (R) & Es bereitet mir Mühe, zwei verschiedene Gespräche gleichzeitig zu führen. \\
        16 & I have a hard time coming up with new ideas quickly. (R) & Es fällt mir schwer, spontan auf neue Ideen zu kommen. \\
        17 & After being interrupted or distracted, I can easily shift my attention back to what I was doing before. & Nachdem ich abgelenkt oder unterbrochen wurde, kann ich meine Aufmerksamkeit mühelos auf die vorherige Arbeit lenken. \\
        18 & When a distracting thought comes to mind, it is easy for me to shift my attention away from it. & Wenn mir ein störender Gedanke in den Sinn kommt, so kann ich ihn ohne grosse Mühe wieder vergessen.  \\
        19 & It is easy for me to alternate between two different tasks. & Es fällt mir leicht, zwischen zwei unterschiedlichen Tätigkeiten hin und her zu wechseln. \\
        20 & It is hard for me to break from one way of thinking about something and look at it from another point of view. (R) & Es fällt mir schwer, mich von meiner bisherigen Denkweise zu lösen und einen Sachverhalt von einer anderen Seite zu betrachten. \\ \hline
    \end{longtable}
	\label{tab:AcsUebersetzung}
\end{center}

% SPANE
\section{Scale of Positive and Negative Experience - SPANE}\label{appendix.spane}
Die Skala für die Erfassung von positiven und negativen Erfahrungen (engl.: Scale of Positive and Negative Experience - SPANE) \cite{Diener:2010} besteht aus einer einleitenden Anweisung und den zugehörenden Fragebogen-Elemente. Bewertet werden die Elemente auf einer fünfstufiger Skala: 1 = sehr selten oder nie (engl.: very rarely or never); 2 = selten (engl.: rarely); 3 = manchmal (engl.: sometimes); 4 = oft (engl.: often); 5 = sehr oft oder immer (engl.: very often or always). 

% Table
\begin{center}
    \begin{longtable}[t]{|p{15 cm}|}
    \caption{SPANE Anweisung - Übersetzung} \\ \hline
        \textbf{Englisch} \\ \hline
        Please think about what you have been doing and experiencing during the past 4 weeks. Then report how much you experienced each of the following feelings, using the scale below. For each item, select a number from 1 to 5, and indicate that number on your response sheet. \\ \hline
        \textbf{Deutsch} \\ \hline 
        Erinnere dich daran, was Du in den letzten vier Wochen gemacht hast und wie Du Dich dabei gefühlt hast. Anschliessend entscheide Dich anhand der unten stehenden Skala, wie oft Du diese unten stehenden Empfindungen erlebt hast. Bewerte jede Aussage auf einer Skala von 1 bis 5 (1 = sehr selten oder nie; 2 = selten; 3 = manchmal; 4 = oft; 5 = sehr oft oder immer). \\ \hline   
    \end{longtable}
	\label{tab:SpaneAnweisung}
\end{center}

% Table
\begin{center}
    \begin{longtable}[t]{|p{0.8 cm}|p{6.6 cm}|p{6.6 cm}|}
    \caption{SPANE Elemente - Übersetzung} \\ \hline
        \textbf{Nr.} & \textbf{Englisch} & \textbf{Deutsch} \\ \hline
        \endfirsthead
        \hline
        \textbf{Nr.} & \textbf{Englisch} & \textbf{Deutsch} \\ \hline
        \endhead 
        & \multicolumn{2}{|c|}{Fortsetzung auf der nächsten Seite $...$ } \\ \hline
        \endfoot
        \hline
        \endlastfoot
        1 & Positive & Positiv\\
        2 & Negative & Negativ\\
        3 & Good & Gut\\
        4 & Bad & Schlecht\\
        5 & Pleasant & Angenehm\\
        6 & Unpleasant & Unangenehm\\
        7 & Happy & Glücklich\\
        8 & Sad & Traurig\\
        9 & Afraid & Ängstlich\\
        10 & Joyful & Froh\\
        11 & Angry & Wütend\\
        12 & Contented & Zufrieden\\  
    \end{longtable}
	\label{tab:SpaneElemente}
\end{center}

% Flourishing Scale
\section{Flourishing Scale - FS}\label{appendix.fs}
Die Skala für die Erfassung des menschlichen Wohls (engl.: Flourishing Scale - FS) \cite{Diener:2010} besteht aus einer einleitenden Anweisung und den zugehörenden Fragebogen-Elemente. Bewertet werden die Elemente auf einer siebenstufigen Skala: 1 = trifft überhaupt nicht zu (engl.: strongly disagree); 2 = trifft nicht zu (engl.: disagree); 3 = trifft kaum zu (engl.: slightly disagree); 4 = gemischt oder weder Zustimmung noch Ablehnung (engl.: mixed or neither agree nor disagree); 5 = trifft etwas zu (engl.: slightly agree); 6 = trifft zu (engl.: agree); 7 = trifft voll und ganz zu (engl.: strongly agree).

% Table
\begin{center}
    \begin{longtable}[t]{|p{15 cm}|}
    \caption{FS Anweisung - Übersetzung} \\ \hline
        \textbf{Englisch} \\ \hline
        Below are eight statements with which you may agree or disagree. Using the 1–7 scale below, indicate your agreement with each item by indicating that response for each statement. \\ \hline
        \textbf{Deutsch} \\ \hline 
        Unten findest Du acht Aussagen, denen Du zustimmen oder widersprechen kannst. Bewerte jede Aussage auf einer Skala von 1 bis 7 (1 = trifft überhaupt nicht zu bis 7 = trifft voll und ganz zu). \\ \hline   
    \end{longtable}
	\label{tab:FsAnweisung}
\end{center}

% Table
\begin{center}
    \begin{longtable}[t]{|p{0.8 cm}|p{6.6 cm}|p{6.6 cm}|}
    \caption{SPANE Elemente - Übersetzung} \\ \hline
        \textbf{Nr.} & \textbf{Englisch} & \textbf{Deutsch} \\ \hline
        \endfirsthead
        \hline
        \textbf{Nr.} & \textbf{Englisch} & \textbf{Deutsch} \\ \hline
        \endhead 
        & \multicolumn{2}{|c|}{Fortsetzung auf der nächsten Seite $...$ } \\ \hline
        \endfoot
        \hline
        \endlastfoot
        1 & I lead a purposeful and meaningful life. & Ich führe ein zielgerichtetes und sinnvolles Leben.\\
        2 & My social relationships are supportive and rewarding. & Meine sozialen Beziehungen stärken und bereichern mich. \\
        3 & I am engaged and interested in my daily activities. & Ich bin engagiert und interessiere mich für meine täglichen Aktivitäten.\\
        4 & I actively contribute to the happiness and well-being of others. & Ich trage aktiv zum Glück und zum Wohlbefinden anderer bei.\\
        5 & I am competent and capable in the activities that are important to me. & In den Aktivitäten, die mir wichtig sind, bin ich kompetent und fähig.\\
        6 & I am a good person and live a good life. & Ich bin ein guter Mensch und ich führe ein gutes Leben. \\
        7 & I am optimistic about my future. & Ich bin optimistisch, was meine Zukunft betrifft.\\
        8 & People respect me. & Die Leute respektieren mich.\\        
    \end{longtable}
	\label{tab:FsElemente}
\end{center}


\end{RaggedRight}


