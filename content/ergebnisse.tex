%%%%%%%%%%%%%%%%%%%%%%%%%%%%%%%%%%%%%%%%%%%%%%%%%%%%%%%%%%%%%%%%%
%_____________ ___    _____  __      __ 
%\____    /   |   \  /  _  \/  \    /  \  Institute of Applied
%  /     /    ~    \/  /_\  \   \/\/   /  Psychology
% /     /\    Y    /    |    \        /   Zuercher Hochschule 
%/_______ \___|_  /\____|__  /\__/\  /    fuer Angewandte Wissen.
%        \/     \/         \/      \/                           
%%%%%%%%%%%%%%%%%%%%%%%%%%%%%%%%%%%%%%%%%%%%%%%%%%%%%%%%%%%%%%%%%
%
% Project     : Bachelorarbeit
% Title       : Ergebnisse
% File        : ergebnisse.tex Rev. 00
% Date        : 06.12.2013
% Author      : Till J. Ernst
%
%%%%%%%%%%%%%%%%%%%%%%%%%%%%%%%%%%%%%%%%%%%%%%%%%%%%%%%%%%%%%%%%%
\let\raggedsection\centering 
\chapter{Ergebnisse}
%\chapter*{Methode}\label{chap.methode}
%\addcontentsline{toc}{chapter}{Methode}
\let\raggedsection\raggedright 
\glsresetall
Zusammenfassung der folgenden Kapitel.\\
Verweis auf den Anhang mit der Mittelwertstabelle.
% Kapitel Deskriptive Analyse der Stichprobendaten
\section{Deskriptive Analyse der Stichprobendaten}
\label{label.stichprobe}
Die Stichprobe belief sich nach Beendigung der Umfrage auf 1376 Personen, die den Fragebogen komplett abgeschlossen haben ($N = 1367$). Insgesamt haben 2181 Personen zumindest den Link für die Onlinebefragung angeklickt. Davon wiederum haben 1943 Personen die Umfrage begonnen, was einer Ausschöpfungsquote von 89,1\% entspricht. 29,6\% derjenigen, die den Fragebogen zumindest angefangen haben, haben die Umfrage nicht bis am Schluss durchgeführt. Wie bereits zu Beginn erwähnt, haben 1376 Personen die Befragung abgeschlossen, was einer Beendigungsquote von 62,7\% entspricht.\\
Die mittlere Bearbeitungszeit für die abgeschlossene Beantwortung der Fragen belief sich auf 15 Minuten und 44 Sekunden. Durchschnittlich haben den Fragebogen 81 Personen pro Tag ausgefüllt. Auf zwei Seiten des Fragebogens wurden die meisten Abbrüche verzeichnet. Bei der einen Seite handelte es sich um die Startseite, die als Einstiegsseite mit den Erläuterungen zur Umfrage und zum Wettbewerb diente, und bei der zweiten Seite handelte es sich um die Befragung zum Medien-Multitasking-Verhalten. Diese beiden erwähnten Seite können im Anhang \ref{chap.appendix_fragebogen} unter Sektion \nameref{anhangSesction.Startseite} und \nameref{anhangSection.muq} gefunden werden.\\
Im folgenden werden auf die einzelnen Variablen der Stichprobendaten genauer eingegangen. Dazu dienen einzelne Häufigkeitstabellen dem besseren Verständnis. Aus Gründen der Übersichtlichkeit wurden einzelne Tabellen in den Anhang \ref{chap.appendix_hauefigkeitstabelle} verschoben.
\subsection{Demographische Daten}
Die Studienpopulation belief sich auf 1367 Personen ($N = 1367$), davon waren 960 Personen weiblich (70,2\%), 404 Personen männlich (29,6\%) und 3 Personen, die keine Angabe (0,2\%) zu ihrem Geschlecht gemacht haben. Das durchschnittliche Alter betrug zur Zeit der Befragung 25,4 Jahre ($SD = 6,5$) wobei die jüngste Person 15 Jahre und die älteste Person 67 Jahre alt war. Wie aus Tabelle \ref{table.sozidemoAlter} zu entnehmen ist, waren 68,4\% der Befragten unter 25 Jahre alt. \\ 
%Tabelle Altergruppen
\begin{table}[ht]
    \centering 
    \caption{Soziodemografische Charakteristik der Altersgruppen, Häufigkeitstabelle}
    \begin{tabular}[t]{|r|R{20mm}|R{20mm}|R{20mm}|R{20mm}|} 
        \hline
        \multicolumn{5}{|c|}{\textbf{Altersgruppen}}\\        
        \multicolumn{1}{|c}{} & \multicolumn{1}{c|}{Häufigkeit} & \multicolumn{1}{|c|}{Prozent} & \multicolumn{1}{|c|}{Gültige} & \multicolumn{1}{|c|}{Kumulierte}\\
        \multicolumn{1}{|c}{} & \multicolumn{1}{c|}{(N)} & \multicolumn{1}{|c|}{(\%)} & \multicolumn{1}{|c|}{Prozente} & \multicolumn{1}{|c|}{Prozente} \\
        \hline
        < 16 & 1 & ,1 & ,1 & ,1\\
        16 - 20 & 163 & 11,9 & 11,9 & 12,0\\
        21 - 25 & 771 & 56,4 & 56,4 & 68,4\\
        26 - 30 & 244 & 17,8 & 17,8 & 86,2\\
        31 - 35 & 86 & 6,3 & 6,3 & 92,5\\
        36 - 40 & 30 & 2,2 & 2,2 & 94,7\\
        41 - 45 & 28 & 2,0 & 2,0 & 96,8\\  
        46 - 50 & 26 & 1,9 & 1,9 & 98,7\\
        > 50  & 18 & 1,3 & 1,3 & 100\\
        Gesamt  & 1367 & 100 & 100 & \\
        \hline
    \end{tabular}
    \label{table.sozidemoAlter}
\end{table}
Insgesamt studierten 374 Personen der Stichprobe an einer Universität (27,4\%) und 993 an einer Fachhochschule (72,6\%). Der Studentenstatus belief sich auf 1084 Vollzeitstudierende (79,3\%) und 282 Teilzeitstudierende (20,6\%). \\
766 Studierende waren zur Zeit der Befragung berufstätig (56\%) und 601 gingen neben dem Studium keiner genannten Arbeit (44\%) nach. Im Durchschnitt arbeiteten diejenigen, die einer Arbeit nachgehen 15 Stunden pro Woche ($Min = 0,5$ Stunden, $Max = 80$ Stunden pro Woche).\\
1244 der Befragten waren zum Zeitpunkt der Befragung ledig (91\%) und 1286 gaben an, keine eigenen Kinder (94,1\%) zu besitzen.

\subsection{Media Multitasking Index -- MMI}
Der Medien Multitasking Index ($MMI$ und $MMI_{angepasst}$) kennzeichnet die durchschnittliche Menge von Medien Multitasking, die während einer typischen Stunde Mediennutzung auftritt (siehe dazu auch Kapitel \ref{subsection.muq} - \nameref{subsection.muq}). Wie aus Tabelle \ref{table.deskrptMedien} zu entnehmen ist, belief sich der Mittelwert für $MMI_{angepasst}$ auf 1,35 ($SD =  1,00$) und für $MMI$ auf 1,29 ($SD = ,97$). Dieser Index lässt die Unterteilung zu in Personen, die schwach Medien-Multitasking betreiben und solche, die stark Medien-Multitasking betreiben.\\
% Tabelle MMI
\begin{table}[ht] 
    \centering
    \caption{Charakteristik des Medien-Multitasking-Index, Häufigkeit und Verteilung}
    \begin{tabular}[t]{|r r|r|r|} 
        \hline
        \multicolumn{4}{|c|}{\textbf{Multimedia-Multitasking-Index}}\\ 
        \hline       
        \multicolumn{2}{|c}{} & \multicolumn{1}{c|}{$MMI_{angepasst}$} & \multicolumn{1}{|c|}{$MMI$}\\
        \multicolumn{2}{|c}{} & \multicolumn{1}{c|}{(Ergebnis mit SMS)} & \multicolumn{1}{|c|}{(Ergebnis ohne SMS)}\\
        \hline
        Gesamtwert (N) & Gültig & 1359 & 1359\\
        & Fehlend & 8 & 8\\
        Mittelwert & $M$ & 1,35 & 1,29\\
        Median & $med$ & 1,16 & 1,10\\
        Standardabweichung & $SD$ & 1,00 & ,97\\
        Minimum & $Min$ & ,00 & ,00\\
        Maximum & $Max$ & 6,48 & 6,38\\
        Starke MMT & $HMMs$ & 2,35 & 2,26\\
        Schwache MMT & $LMMs$ & ,35 & ,32\\
        \hline
    \end{tabular}
    \label{table.deskrptMedien}
\end{table}
Für die Unterteilung in starke Medien-Mulitasker lag der Wert beim $MMI_{angepasst}$ für $HMMs >= 2,35$ ($M + SD$ resp. 1,35 + 1,00) und für schwache Multitasker bei $LMMs <= 0,35$ ($M - SD$ resp. 1,35 - 1,00). \\
Analog dazu wurden die Werte für die Variable $MMI$ berechnet: $HMMs >= 2,26$ und $LMMs <= 0,32$. \\
Bezogen auf die Variable $MMI_{angepasst}$ ergab sich zum Zeitpunkt der Befragung eine Einteilung von 13,8\% (188) als schwache Medien-Multitasker und 14,8\% (202) als starke Medien-Multitasker. 70,9\% (969) Personen befanden sich zwischen der Obergrenze $HMMs$ und der Untergrenze $LMMs$.
Bezogen auf die Variable $MMI$ waren es 13,2\% (181) schwache und 15,1\% (207) starke Medien-Multitasker. 71\% (971) Personen befanden sich zwischen $HMMs$ und $LMMs$. Diese Angaben sind ebenso in Anhang \ref{anhang.hauefigkeitstabelle} zu finden.

% ACS
\subsection{Aufmerksamkeitskontroll-Skala -- ACS}
Die Aufmerksamkeitskontroll-Skale dient zur Erfassung der allgemeinen Unterschiede der Porbanden in der selbstintierten Aufmerksamkeitskontrolle (siehe dazu Kapitel \ref{subsection.acs} - \nameref{subsection.acs}). \\ Für diese Variable füllten alle Probanden ($N=1367$) die Fragen zur Aufmerskamkeitskontrolle korrekt aus. Dabei entstand ein Mittelwert von $M = 53,43$, mit einer Standardabweichung von $SD = 6,82$. Das Minimum belief sich auf einen Wert von 34, das Maximum auf 75 (mögliches Minimum = 20 und mögliches Maximum = 80). Eine detaillierte Tabelle dieser Werte ist im Anhang \ref{anhang.hauefigkeitstabelle} zu finden.
% FS und SpanE
\subsection{Skala für menschliches Aufblühen -- FS und Skala der positiven und Negativen Erfahrungen -- SPANE}
In diesem Teil werden die Variablen, die für die Erfassung des subjektiven Wohlbefindens verwendet werden beschrieben (siehe dazu Kapitel \ref{subsection.flourishingScale} - \nameref{subsection.flourishingScale}).\\
Gemäss Tabelle \ref{table.deskrptFsSpane} steht die Variable FS für das menschliche Aufblühen und weist einen Gesamtwert $N$ von 1364 Personen auf (3 fehlende Werte). Der Mittelwert $M$ beträgt 46,44 mit einer Standardabweichung $SD$ von 5,19. In diesem Sample wurde ein Minimalwert von 17 und ein Maximalwert von 56 gemessen (mögliches Minimal = 8 und mögliches Maximum = 56).\\
Die Variablen für das Erfassen der positiven ($SPANE-P$) und negativen ($SPANE-N$) Erfahrungen konnte in eine weitere Variable $SPANE-B$ kombiniert werden. Gemäss Tabelle \ref{table.deskrptFsSpane} wurde dieser Fragebogen von 1361 Personen beantwortet (6 Personen wiesen keine gültigen Werte auf). Die Variabel $SPANE-P$ wies einen Mittelwert $M$ von 23,16 mit einer Standardabweichung $SD$ von 3,52 auf. Der gesamte mögliche Bereich von Minumum 6 bis Maximum 30 wurde ausgeschöpft. Die Variabel $SPANE-N$ wies einen Mittelwert $M$ von 14,22 und einer Standardabweichung von 4,02 auf. Der Bereich ging von einem Minimum von 6 bis zu einem Maximum von 29 (mögliches Minimum = 6, mögliches Maximum = 30). Die zusammengefasste Variabel $SPANE-B$ wie einen Mittelwert $M$ von 8,94 und eine Standardabweichung $SD$ von 6,69 auf. Der Bereich dieser Variabel reichte von einem Minimum von -18, zu einem Maximum von 24 (mölgiches Minimum = -24 und mögliches Maximum = 24).
% Tabelle FS und SPANE
\begin{table}[H] 
    \centering
    \caption{Charakteristik der Skalen für menschliches Aufblühen und der positiven und Negativen Erfahrungen, Häufigkeit und Verteilung}
    \begin{tabular}[t]{|r r|R{15mm}|R{15mm}|R{15mm}|R{15mm}|} 
        \hline
        \multicolumn{6}{|c|}{\textbf{Flourishing Scale und SPANE}}\\ 
        \hline       
        \multicolumn{2}{|c}{} & \multicolumn{1}{c|}{$FS$} & \multicolumn{3}{|c|}{$SPANE$}\\
        \multicolumn{3}{|c|}{} & \multicolumn{1}{c|}{$P$} &  \multicolumn{1}{c|}{$N$} & \multicolumn{1}{|c|}{$B$}\\
        \hline
        Gesamtwert (N) & Gültig & 1364 & 1361 & 1361 & 1361\\
        & Fehlend & 3 & 6 & 6 & 6 \\
        Mittelwert & $M$ & 46,44 & 23,16 & 14,22 & 8,94\\
        Median & $med$ & 47& 24 & 14 & 10 \\
        Standardabweichung & $SD$ & 5,19 & 3,53 & 4,02 & 6,69\\
        Minimum & $Min$ & 17& 6 & 6 & -18 \\
        Maximum & $Max$ & 56& 30 & 29 & 24 \\
        \hline
    \end{tabular}
    \label{table.deskrptFsSpane}
\end{table}
% Haupthypo
\section{Haupthypothese: Auswirkung von Medien-Multitasking auf SWB}\label{label.haupthypothese}
Die Haupthypothese lautet: Studierende, die häufig Medien-Multitasking betreiben, haben ein tieferes subjektives Wohlbefinden als Studierende, die weniger Medien-Mulitasking betreiben. Zur Überprüfung der Hypothese wurde diese wie folgt operationalisiert:\par
Nullhypothese:\\
Es gibt keinen Zusammenhang zwischen dem Medien-Multitasking Verhalten und dem subjektiven Wohlbefinden. Das heisst, der Medien-Multitasking-Index $MMI$ wirkt sich nicht auf die Skalen $FS$ und $SPANE$ aus.
\par
Alternativhypothese:\\
Es gibt einen gerichteten Zusammenhang zwischen dem Medien-Multitasking Verhalten und dem subjektiven Wohlbefinden. Das heisst, der Medien-Multitasking-Index $MMI$ wirkt sich negativ auf die Skala $FS$ und $SPANE-B$ aus (negativ auf $SPANE-P$ und positiv auf $SPANE-N$).
% Nebenhypo
\section{Nebenhypothese: Aufmerksamkeitskontrolle als beeinflussende Variable}\label{label.nebenhypothese}
Die Nebenhypothese lautet: Je häufiger Studierende Medien-Multitasking betreiben, desto tiefer ist deren Aufmerksamkeitskontrolle, was sich wiederum negativ auf das subjektive Wohlbefinden auswirkt. Zur Überprüfung der Hypothese wurde diese wie folgt operationalisiert:\par
Nullhypothese:\\
\par
Alternativhypothese:\\

