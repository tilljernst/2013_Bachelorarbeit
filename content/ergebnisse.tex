%%%%%%%%%%%%%%%%%%%%%%%%%%%%%%%%%%%%%%%%%%%%%%%%%%%%%%%%%%%%%%%%%
%_____________ ___    _____  __      __ 
%\____    /   |   \  /  _  \/  \    /  \  Institute of Applied
%  /     /    ~    \/  /_\  \   \/\/   /  Psychology
% /     /\    Y    /    |    \        /   Zuercher Hochschule 
%/_______ \___|_  /\____|__  /\__/\  /    fuer Angewandte Wissen.
%        \/     \/         \/      \/                           
%%%%%%%%%%%%%%%%%%%%%%%%%%%%%%%%%%%%%%%%%%%%%%%%%%%%%%%%%%%%%%%%%
%
% Project     : Bachelorarbeit
% Title       : Ergebnisse
% File        : ergebnisse.tex Rev. 00
% Date        : 06.12.2013
% Author      : Till J. Ernst
%
%%%%%%%%%%%%%%%%%%%%%%%%%%%%%%%%%%%%%%%%%%%%%%%%%%%%%%%%%%%%%%%%%
\let\raggedsection\centering 
\chapter{Ergebnisse}
%\chapter*{Methode}\label{chap.methode}
%\addcontentsline{toc}{chapter}{Methode}
\let\raggedsection\raggedright 
\glsresetall
Zusammenfassung der folgenden Kapitel.\\
Verweis auf den Anhang mit der Mittelwertstabelle.
% Kapitel Deskriptive Analyse der Stichprobendaten
\section{Deskriptive Analyse der Stichprobendaten}
\label{label.stichprobe}
Das Gesamtsample (100\%) belief sich auf 2181 Personen. Die Nettobeteiligung betrug 1943 Personen, was einer Ausschöpfungsquote von 89,1\% entspricht. Die Nettobeteiligung alle diejenigen Personen, die den Fragebogen zumindest begonnen haben (unterbrochen und beendet). Davon haben haben 70,4\% den Fragebogen vollständig ausgefüllt und abgeschlossen (1376) und 29,6\% haben den Fragebogen unterbrochen oder abgebrochen (576). Vom Gesamtsample ausgehend entspricht dies einer Beendigungsquote von 62,7\%. Die Stichprobenanzahl für diese Arbeit belief sich somit auf 1367 Personen ($N = 1367$). Die mittlere Bearbeitungszeit für die abgeschlossene Beantwortung der Fragen belief sich auf 15 Minuten und 44 Sekunden. Durchschnittlich haben den Fragebogen 81 Personen im Tag ausgefüllt. Zwei Seiten des Fragebogens wiesen die höchsten Abbrüche auf. Die eine war die Startseite mit der Beschreibung der Umfrage und bei der andere Seite handelte es sich um die Fragen zum Medien-Multitasking-Verhalten (siehe Anhang \ref{chap.appendix_fragebogen} Sektion \nameref{anhangSesction.Startseite} und \nameref{anhangSection.muq}). \\
Im folgenden wird auf die Stichprobendaten genauer eingegangen. Dazu dienen einzelne Häufigkeitstabellen dem besseren Verständnis. Eine ausführlicherer Häufigkeitstabelle inklusive der Lagemasse aller soziodemografischen Charakteristiken der Studienpopulation  ist in Anhang \ref{chap.appendix_hauefigkeitstabelle} zu finden.
\subsection{Demographische Daten}
Die Studienpopulation belief sich auf 1367 Personen ($N = 1367$), davon waren 960 Personen weiblich (70,2\%), 404 Personen männlich (29,6\%) und 3 Personen, die keine Angabe (0,2\%) zu ihrem Geschlecht gemacht haben. Das durchschnittliche Alter betrug zur Zeit der Befragung 25,4 Jahre ($SD = 6,5$) wobei die jüngste Person 15 Jahre und die älteste Person 67 Jahre alt war. Wie aus Tabelle \ref{table.sozidemoAlter} zu entnehmen ist, waren 68,4\% der Befragten unter 25 Jahre alt. \\ 
%Tabelle Altergruppen
\begin{table}[ht]
    \centering 
    \caption{Soziodemografische Charakteristik der Altersgruppen, Häufigkeitstabelle}
    \begin{tabular}[t]{|r|r|r|r|r|} 
        \hline
        \multicolumn{5}{|c|}{\textbf{Altersgruppen}}\\        
        \multicolumn{1}{|c}{} & \multicolumn{1}{c|}{Häufigkeit} & \multicolumn{1}{|c|}{Prozent} & \multicolumn{1}{|c|}{Gültige} & \multicolumn{1}{|c|}{Kumulierte}\\
        \multicolumn{1}{|c}{} & \multicolumn{1}{c|}{(N)} & \multicolumn{1}{|c|}{(\%)} & \multicolumn{1}{|c|}{Prozente} & \multicolumn{1}{|c|}{Prozente} \\
        \hline
        < 16 & 1 & ,1 & ,1 & ,1\\
        16 - 20 & 163 & 11,9 & 11,9 & 12,0\\
        21 - 25 & 771 & 56,4 & 56,4 & 68,4\\
        26 - 30 & 244 & 17,8 & 17,8 & 86,2\\
        31 - 35 & 86 & 6,3 & 6,3 & 92,5\\
        36 - 40 & 30 & 2,2 & 2,2 & 94,7\\
        41 - 45 & 28 & 2,0 & 2,0 & 96,8\\  
        46 - 50 & 26 & 1,9 & 1,9 & 98,7\\
        > 50  & 18 & 1,3 & 1,3 & 100\\
        Gesamt  & 1367 & 100 & 100 & \\
        \hline
    \end{tabular}
    \label{table.sozidemoAlter}
\end{table}
Insgesamt studierten 374 Personen der Stichprobe an einer Universität (27,4\%) und 993 an einer Fachhochschule (72,6\%). Der Studentenstatus belief sich auf 1084 Vollzeitstudierende (79,3\%) und 282 Teilzeitstudierende (20,6\%). \\
766 Studierende waren zur Zeit der Befragung berufstätig (56\%) und 601 gingen neben dem Studium keiner Arbeit (44\%) nach. Im Durchschnitt arbeiteten diejenigen, die einer Arbeit nachgehen 15 Stunden pro Woche ($Min = 0,5$ Stunden, $Max = 80$ Stunden pro Woche).\\
Wie aus Tabelle \ref{table.sozidemoZivil} hervorgeht waren 1244 der Befragten zur Zeit der Befragung ledig (91\%). 1286 gaben an, keine eigenen Kinder (94,1\%) zu besitzen.
%Tabelle Zivilstand
\begin{table}[ht] 
    \centering
    \caption{Soziodemografische Charakteristik des Zivilstands, Häufigkeitstabelle}
    \begin{tabular}[t]{|r|r|r|r|r|} 
        \hline
        \multicolumn{5}{|c|}{\textbf{Zivilstand}}\\ 
        \hline       
        \multicolumn{1}{|c}{} & \multicolumn{1}{c|}{Häufigkeit} & \multicolumn{1}{|c|}{Prozent} & \multicolumn{1}{|c|}{Gültige} & \multicolumn{1}{|c|}{Kumulierte}\\
        \multicolumn{1}{|c}{} & \multicolumn{1}{c|}{(N)} & \multicolumn{1}{|c|}{(\%)} & \multicolumn{1}{|c|}{Prozente} & \multicolumn{1}{|c|}{Prozente} \\
        \hline
        keine Angabe & 2 & ,1 & ,1 & ,1\\
        ledig & 1244 & 91,0 & 91,0 & 91,1\\
        verheiratet & 100 & 7,3 & 7,3 & 98,5\\
        getrennt & 5 & ,4 & ,4 & 98,8\\
        geschieden & 16 & 1,2 & 1,2 & 100\\
        Gesamt  & 1367 & 100,0 & 100,0 & \\
        \hline
    \end{tabular}
    \label{table.sozidemoZivil}
\end{table}
\subsection{Media Multitasking Index MMI}
tbd
% Tabelle MMI
\begin{table}[ht] 
    \centering
    \caption{Charakteristik des Medien-Multitasking-Index, Häufigkeit und Verteilung}
    \begin{tabular}[t]{|rr|r|r|} 
        \hline
        \multicolumn{4}{|c|}{\textbf{Multimedia Index}}\\ 
        \hline       
        \multicolumn{2}{|c}{} & \multicolumn{1}{c|}{$MMI_{angepasst}$} & \multicolumn{1}{|c|}{$MMI$}\\
        \multicolumn{2}{|c}{} & \multicolumn{1}{c|}{(Ergebnis mit SMS)} & \multicolumn{1}{|c|}{(Ergebnis ohne SMS)}\\
        \hline
        Gesamtwert (N) & Gültig & 1359 & 1359\\
        & Fehlend & 8 & 8\\
        Mittelwert & $M$ & 1,35 & 1,29\\
        Median & $med$ & 1,16 & 1,10\\
        Standardabweichung & $SD$ & 1,00 & ,97\\
        Minimum & $Min$ & ,00 & ,00\\
        Maximum & $Max$ & 6,48 & 6,38\\
        \hline
    \end{tabular}
    \label{table.sozidemoZivil}
\end{table}
\subsection{Aufmerksamkeitskontroll-Skala ACS}
TBD

\subsection{Skala für menschliches Aufblühen FS und Skala der positiven und Negativen Erfahrungen SPANE}
TBD
\section{Haupthypothese}\label{label.haupthypothese}
TBD
\section{Nebenhypothese}\label{label.nebenhypothese}
TBD

