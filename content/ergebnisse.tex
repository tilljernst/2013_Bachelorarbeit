%%%%%%%%%%%%%%%%%%%%%%%%%%%%%%%%%%%%%%%%%%%%%%%%%%%%%%%%%%%%%%%%%
%_____________ ___    _____  __      __ 
%\____    /   |   \  /  _  \/  \    /  \  Institute of Applied
%  /     /    ~    \/  /_\  \   \/\/   /  Psychology
% /     /\    Y    /    |    \        /   Zuercher Hochschule 
%/_______ \___|_  /\____|__  /\__/\  /    fuer Angewandte Wissen.
%        \/     \/         \/      \/                           
%%%%%%%%%%%%%%%%%%%%%%%%%%%%%%%%%%%%%%%%%%%%%%%%%%%%%%%%%%%%%%%%%
%
% Project     : Bachelorarbeit
% Title       : Ergebnisse
% File        : ergebnisse.tex Rev. 00
% Date        : 06.12.2013
% Author      : Till J. Ernst
%
%%%%%%%%%%%%%%%%%%%%%%%%%%%%%%%%%%%%%%%%%%%%%%%%%%%%%%%%%%%%%%%%%
\let\raggedsection\centering 
\chapter{Ergebnisse}
%\chapter*{Methode}\label{chap.methode}
%\addcontentsline{toc}{chapter}{Methode}
\let\raggedsection\raggedright 
\glsresetall

% Kapitel Deskriptive Analyse der Stichprobendaten
\section{Deskriptive Analyse der Stichprobendaten}
\label{label.stichprobe}
\cite{bioshock2}
\cite{7.08:56}
\begin{itemize}
      \item deskriptive Statistik
      \item erreichte Teilnehmerzahl
      \item Stichprobengrösse
      \item Vergleichen 
\end{itemize}
\subsection{Demographische Daten}
\subsection{Media Multitasking Index MMI}
\subsection{Aufmerksamkeitskontroll-Skala ACS}
\subsection{Skala für menschliches Aufblühen FS und Skala der positiven und Negativen Erfahrungen SPANE}


\section{Haupthypothese}\label{label.haupthypothese}
\section{Nebenhypothese}\label{label.nebenhypothese}


