%%%%%%%%%%%%%%%%%%%%%%%%%%%%%%%%%%%%%%%%%%%%%%%%%%%%%%%%%%%%%%%%%
%_____________ ___    _____  __      __ 
%\____    /   |   \  /  _  \/  \    /  \  Institute of Applied
%  /     /    ~    \/  /_\  \   \/\/   /  Psychology
% /     /\    Y    /    |    \        /   Zuercher Hochschule 
%/_______ \___|_  /\____|__  /\__/\  /    fuer Angewandte Wissen.
%        \/     \/         \/      \/                           
%%%%%%%%%%%%%%%%%%%%%%%%%%%%%%%%%%%%%%%%%%%%%%%%%%%%%%%%%%%%%%%%%
%
% Project     : Bachelorarbeit
% Title       : Ergebnisse
% File        : ergebnisse.tex Rev. 00
% Date        : 06.12.2013
% Author      : Till J. Ernst
%
%%%%%%%%%%%%%%%%%%%%%%%%%%%%%%%%%%%%%%%%%%%%%%%%%%%%%%%%%%%%%%%%%
\let\raggedsection\centering 
\chapter{Ergebnisse}
%\chapter*{Methode}\label{chap.methode}
%\addcontentsline{toc}{chapter}{Methode}
\let\raggedsection\raggedright 
\glsresetall
% Kapitel Deskriptive Analyse der Stichprobendaten
\section{Deskriptive Analyse der Stichprobendaten}
\label{label.stichprobe}
Das Gesamtsample (100\%) belief sich auf 2181 Personen. Die Nettobeteiligung betrug 1943 Personen, was einer Ausschöpfungsquote von 89,1\% entspricht Die Nettostichprobe enthält die beendeten Interviews sowie Teilnehmer, die die Befragung unterbrochen haben. Die Beendigungsquote belief sich auf 62,7\%. Von der Nettobeteiligung haben 70,4\% den Fragebogen beendet (1376) und 29,6\% unterbrochen oder abgebrochen (576). Somit haben insgesamt 1367 Personen den Fragebogen vollständig ausgefüllt und abgeschlossen. Die mittlere Bearbeitungszeit für die abgeschlossene Beantwortung der Fragen belief sich auf 15 Minuten und 44 Sekunden. Durchschnittlich haben den Fragebogen 81 Personen im Tag ausgefüllt. Die Seiten, auf der die meisten Abbrüche erfolgten, war die Seite mit den Fragen zum Medien-Multitasking-Verhalten (siehe \nameref{chap.appendix_fragebogen} Sektion \nameref{anhangSection.muq}) und die Startseite. 
\subsection{Demographische Daten}
\subsection{Media Multitasking Index MMI}
\subsection{Aufmerksamkeitskontroll-Skala ACS}
\subsection{Skala für menschliches Aufblühen FS und Skala der positiven und Negativen Erfahrungen SPANE}


\section{Haupthypothese}\label{label.haupthypothese}
\section{Nebenhypothese}\label{label.nebenhypothese}


