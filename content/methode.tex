%%%%%%%%%%%%%%%%%%%%%%%%%%%%%%%%%%%%%%%%%%%%%%%%%%%%%%%%%%%%%%%%%
%_____________ ___    _____  __      __ 
%\____    /   |   \  /  _  \/  \    /  \  Institute of Applied
%  /     /    ~    \/  /_\  \   \/\/   /  Psychology
% /     /\    Y    /    |    \        /   Zuercher Hochschule 
%/_______ \___|_  /\____|__  /\__/\  /    fuer Angewandte Wissen.
%        \/     \/         \/      \/                           
%%%%%%%%%%%%%%%%%%%%%%%%%%%%%%%%%%%%%%%%%%%%%%%%%%%%%%%%%%%%%%%%%
%
% Project     : Bachelorarbeit
% Title       : Methode
% File        : methode.tex Rev. 00
% Date        : 06.12.2013
% Author      : Till J. Ernst
%
%%%%%%%%%%%%%%%%%%%%%%%%%%%%%%%%%%%%%%%%%%%%%%%%%%%%%%%%%%%%%%%%%
\glsresetall
\chapter{Methode (Vergangenheit)}

% Kapitel Studiendesign
%---------------------
\section{Studiendesign}\label{section.studiendesign}
In der vorliegenden Studie handelt es sich um eine empirische Querschnitts-Studie mit quantitativem Charakter. Die Generierung der Daten für die Beantwortung der Fragestellung erfolgte mittels Internet-Fragebogen. Dazu wurden die für die Hypothesen relevanten Parameter empirisch erhoben und mittels deskriptiver Statistik ausgewertet. Um eine möglichst umfassende Beantwortung der Forschungsfrage zu ermöglichen, wurde eine grösstmögliche Stichprobe von Probanden angestrebt, die täglich Medien nutzen. Bei den befragten Probanden handelte es sich um Studierende, die aktuell ein Studium an einer Universität oder einer Fachhochschule in der Schweiz absolvieren. Der Fragebogen wurde in deutscher Sprache verfasst, weshalb ausschliesslich Schulen der deutschsprachigen Schweiz berücksichtigt wurden.\\
Diese Studie verfolgte das Ziel, mögliche Tendenzen aufzuzeigen. Durch die Befragung einer spezifischen Bevölkerungsgruppe können die Resultate nicht verallgemeinert werden und dienen einer exemplarischen Betrachtung.

%---------------------
\subsection{Hauptzielparameter}\label{subsection.hauptzielparameter}
Die folgenden Hauptzielparameter dienten zur Beantwortung der Fragestellung (siehe \nameref{section.fragestellung}). Einen möglichen Einflusses von Medien-Multitasking auf das subjektive Wohlbefinden wurde mittels bereits existierender Fragebögen erhoben. Diese Fragebögen dienten zur Erfassung der Häufigkeit der Nutzung von einzelnen Medien, der Häufigkeit der simultanen Mediennutzung, der Fähigkeit der simultanen Mediennutzung und des aktuellen subjektiven Wohlbefindens der Probanden. Mittels Erhebung der demographischen Daten wurde das Geschlecht, Alter, Studienform (Teilzeit- oder Vollzeitstudium), Schulform (Universität oder Fachhochschule), berufliche Tätigkeit, Zivilstand und Familienstatus erhoben. Diese Daten dienten zur Aufteilung der Stichprobe in verschiedene Gruppen, um mögliche Einflüsse auf das subjektive Wohlbefinden anhand demographischer Daten zu untersuchen.

%---------------------
\subsection{Durchführung der Untersuchung}
Die Durchführung dieser Arbeit erfolgte im Rahmen einer Bachelorarbeit an der Zürcher Hochschule für angewandte Wissenschaften im Bereich angewandter Psychologie. Die gesamte Projektdauer betrug sechs Monate und wurde im Zeitrahmen des Frühlingssemesters 2014 durchgeführt.

% Kapitel Auswahl der Versuchspersonen
%---------------------
\section{Auswahl der Versuchspersonen}\label{section.auswahlVersuchsp}
Gemäss Fragestellung wurde der Einfluss von Medien-Multitasking auf das subjektive Wohlbefinden von Studierenden untersucht. Dazu wurden Studierende aus der deutschsprachigen Schweiz, die an einer Universität oder Fachhochschule studieren, mittels Fragebogen zu ihrem Medien-Multitasking Verhalten befragt. Die Schulen wurden anhand der von SWITCH \cite{Switch2014} zur Verfügung gestellten Liste der Universitäten und Fachhochschulen der Schweiz ausgewählt (SWITCH ist ein Hochleistungsnetzwerk der Schweiz, welches die Verbindung von Internetnutzer weltweit und für die Vernetzung von Fachhochschulen im Bereich Schweizer Wissenschaft zuständig ist). Prinzipiell wurde versucht alle Studierenden der ausgewählten Schulen zu erreichen, die entweder im Teilzeit- oder Vollzeitmodus studieren. \\
Mit Hilfe der Poweranalyse \cite{Faul2009} wurde für die Stichprobenumfangsplanung die angestrebte  Stichprobengrösse im Vorfeld auf $N = 134$ angesetzt. Diese Grösse wurde mittels der Software G*Power Version 3.1 für Apple Computer für die Beantwortung der ungerichteten Hauptfragestellung mittels Bivariater-Korrelation berechnet. Folgende Werte wurden dazu verwendet: Es wurde von einem gewünschten mittlerer Effekt $r = .3$  \cite{Cohen1988} für den Korrelationskoeffizienten ausgegangen, einem Signifikanzniveau von $\alpha=.05$ und einer Teststärke von $1-\beta=.95$.
%---------------------
\subsection{Rekrutierung}\label{subsection.rekrutierung}
Die Rekrutierung der Studierenden an den Universitäten und Fachhochschulen der deutschen Schweiz erfolgte wenn immer möglich direkt über die Mailverteiler der einzelnen Schulen (Liste der angeschriebenen Schulen siehe Anhang \ref{chap.appendix_schulListe}). Damit für diese Untersuchung eine möglichst grosse Stichprobe generiert werden konnte, wurde wenn immer möglich innerhalb der einzelnen Schulen die einzelnen Departemente separat angeschrieben. Da diese Mailverteiler der Schulen in den meisten Fällen nicht öffentlich zugänglich sind, wurde direkt versucht auf den einzelnen Schulsekretariaten oder Departementsstellen (z.B.: Dekanat) die Umfrage zu platzieren. Dies wurde teilweise mittels telefonischem Kontakt und teilweise per vordefiniertem Mail vorgenommen. Das Mailing, das zu diesem Zweck erstellt wurde, ist in Anhang \ref{chap.appendix_mailing} zu finden. In diesem Schreiben wurden die einzelnen Schulen um Unterstützung gebeten, in dem sie die Umfrage an ihre Studierenden weiter leiteten. Für die Weiterleitung an die Studierenden wurde dem Mailing eine beschreibende Nachricht der Studie mit dem Link zum Fragebogen angehängt. \\
Schulen die eher konservativ gegenüber Umfragebögen anderer Schulen eingestellt sind, lehnen das Weiterleiten der Umfrage an die Studierenden oftmals ab. Für diese Fälle wurde ein Aushang für das schwarze Brett der Schule erstellt, welcher auf Wunsch der entsprechenden Stelle nachgeliefert und von den Studiumsverantwortlichen selbständig angebracht werden konnte (siehe Aushang Anhang \ref{chap.appendix_aushang}). \\
In dieser Untersuchung wurde das Mailing an über 70 Stellen der in Kapitel \ref{section.auswahlVersuchsp} genannten Schulen und deren Departemente versendet.
%---
\subsection{Einschlusskriterien}\label{subsection.einschlusskriterien}
Alle Studierende von Universitäten und Fachhochschulen aus dem deutschsprachigen Raum, die einem Teilzeit- oder einem Vollzeitstudium nachgehen.
%---
\subsection{Ausschlusskriterien}\label{subsection.ausschlusskriterien}
Ausgeschlossen werden Studierende eines Fernstudiums, da hier weitere Bedingungen gegeben sind, die eine Vergleichbarkeit erschwert. Der Fragebogen wird lediglich auf Deutsch zur Verfügung gestellt. Aus diesem Grund wurden nur Personen berücksichtigt, die der deutschen Sprache mächtig sind. 


% Kapitel Erhebungsinstrumente
%---------------------
\section{Erhebungsinstrumente}\label{section.erhebungsinstrumente}
Die Erhebung der vorliegenden Arbeit basiert auf bereits existierenden Fragebögen aus Studien und der Erfassung von soziodemographischen Daten. Diese Fragen wurde in Form einer Onlineumfrage umgesetzt. In den folgenden Kapiteln werden die zur Überprüfung der Hypothesen verwendeten Fragebögen in der Reihenfolge der Durchführung beschrieben. Diese Fragebögen wurden aus dem Englischen übersetzt, deren Übersetzungen sich im Anhang \ref{chap.appendix_mediaUseQuestionnaire} und folgende befinden. \\
\subsection{Soziodemographischen Daten}\label{subsection.soziDaten}
Bei den soziodemographischen Daten wurden folgende Parameter erfasst: Geschlecht, Alter, Studium (Universität oder Fachhochschule), Studienrichtung, Studentenstatus (Vollzeit- oder Teilzeitstudium), Berufsstand (arbeitstätig neben dem Studium), Zivilstand und Familienstatus (Kinder). \\
%---
\subsection{Media Use Questionnaire und Media Multitasking Index}\label{subsection.muq}
Für die Bestimmung des Multitasking-Verhaltens wurde der \textit{Medien Multitasking Index} generiert. Dieser Index wurde mittels Fragebogenerhebung berechnet, der die Häufigkeit der Nutzung von einzelnen Medien und die gleichzeitige Nutzung verschiedener Medien miteinander misst. Der Medien Multitasking Index und der Fragebogen zur Häufigkeit von Mediennutzung basieren auf dem von \citeA{Ophir2009} entwickelten \textit{\gls{labelMMI}} und dem \textit{\gls{labelMUQ}}. Diese Instrumente wurden für diese Arbeit aus dem Englischen ins Deutsche übersetzt (siehe Anhang \ref{chap.appendix_mediaUseQuestionnaire}). Der Fragebogen registriert wie viele Stunden eine Person eine von 12 vordefinierten Medienformen in einer Woche nutzt und wie oft diese Person diese Medienform mit anderen Medienformen simultan verwendet. \\ 
Für dieser Umfrage wurde der \gls{labelMMI} leicht angepasst. Anstelle von ganzen Stunden in einer Woche, wurde die durchschnittliche Anzahl Minuten an einem Tag erfasst. Dies erschien für Medien wie Textnachrichten einfacher abschätzbar zu sein. Für die Berechnung des \gls{labelMMI} mussten anschliessend die erfassten Minuten in Stunden einer Woche hochgerechnet werden. Dies wurde mittels folgender Formel berechnet: \(\frac{n_{i} \times 7_{(Tage)}}{60}\). \(n_{i}\) entspricht der Anzahl Minuten pro Tag, die für die Nutzung eines Mediums verwendet werden.\\
Die für die Mediennutzung erfassten Medien beliefen sich auf 12 Medienformen. Diese waren Druckmedien, Fernsehen, Online Video (wie zum Beispiel Youtube), Musik, Nicht-musikalische Audiomedien (z.B. Hörbücher, Podcast, etc.), Video oder Computer Games, Telefonieren (Mobil- und / oder Festnetz), Instant Messaging (z.B. Skype, Windows Live Messenger, Yahoo Messenger, etc.), SMS (Textnachrichten), Email, Internet-Surfen und Andere Computerbasierte-Tätigkeiten (z.B. Word, Videobearbeitung, programmieren, etc.). \\
Neben der Nutzungsdauer einzelner Medien wurde die gleichzeitige Nutzung verschiedener Medien mittels Medien-Multitasking-Matrix gebildet. Diese bestand im Wesentlichen aus den 12 definierten Medien, die in Hauptmedien (vertikale Spalte) und Nebenmedien (horizontale Zeile) aufgeteilt wurden. Die Aufgabe für die Probanden bestand darin, zu jedem Hauptmedium eine Schätzung abzugeben, wie oft ein weiteres Nebenmedium gleichzeitig zusammen mit dem jeweiligen Hauptmedium verwendet wurde. Die Bewertung innerhalb der Matrize erfolgte mittels definierter Auswahl \textit{meistens}, \textit{etwas}, \textit{wenig} oder \textit{nie}.\\
Mittels der Medien-Nutzungsdauer und der gleichzeitigen Nutzung einzelner Medien wurde der Medien Multitasking Index gebildet. Hierbei ist zu beachten, dass die Textnachrichten in der ursprünglichen Formel gemäss \cite{Ophir2009}  nicht berücksichtigt wurden. Die Autoren gingen davon aus, dass die Abschätzung der Nutzung von Textnachrichten in Stunden nicht exakt vorgenommen werden konnte und liessen sie deshalb weg. Dieses Medium wurde im ursprünglichen Fragebogen als Sekundärmedium belassen. In dieser Studie wurde die Erfassung der Medien-Nutzung in Minuten vorgenommen, weshalb die Textnachrichten als Hauptmedien dazu gezählt wurden. Die angepasste und die ursprüngliche Formel für die Berechnung des Indexes lautete:
%Formel
\begin{equation}\label{formel.mmiext}
    MMI_{ext}=\sum_{i=1}^{12} \frac{m_{i} \times \frac{n_{i} \times 7_{(Tage)}}{60}}{h_{total}}
\end{equation}
%Formel
\begin{equation}\label{formel.mmi}
    MMI=\sum_{i=1}^{11} \frac{m_{i} \times h_{i}}{h_{total}}
\end{equation}
In der ursprünglichen Formel $MMI$ steht \(h_{i}\) für die Anzahl Stunden pro Woche, die für ein Hauptmedium \(i\) verwendet wurde. In der angepassten Formel $MMI_{ext}$ wird die Anzahl Stunden pro Woche aus der Anzahl Minuten pro Tag \(n_i\) auf eine Woche hochgerechnet (\(h_{i}=\frac{n_{i} \times 7}{60}\)). \(h_{total}\) steht für die totale Anzahl Stunden aller Medien, die in einer Woche verwendet werden. \(m_i\) beinhaltet die Summe der Abschätzungen des Probanden bezüglich seiner simultanen Nutzung eines oder mehrerer Nebenmedien. Die Summenbildung ergibt die durchschnittliche Anzahl der Medien, die neben dem Hauptmedium verwendet werden. Für die Berechnung wurde den Abschätzungen der Probanden numerische Werte wie folgt zugewiesen: 1 = \textit{meistens}, .67 = \textit{etwas}, .33 = \textit{wenig} oder 0 = \textit{nie}. Der resultierende Medien Multitasking Index $MMI$ (resp. $MMI_{ext}$) kennzeichnet die durchschnittliche Menge von Medien Multitasking, die während einer typischen Stunde Mediennutzung auftritt. Dieser Index lässt die Unterteilung in Personen zu, die stark Medien-Multitasking betreiben ($HMMs$) und Personen, die schwach Medien-Multitasking betreiben ($LMMs$). Für alle die ein mittleres Medien-Multitasking betreiben, wird die Bezeichnung $NMMs$ verwendet. Für die Unterscheidung dieser beiden Bereiche wurde die Standardabweichung hinzugezogen (eine Standardabweichung oder mehr über dem Mittel für starkes Multitasking und eine Standardabweichung oder weniger unter dem Mittel für schwaches Multitasking): 
%Formel HMMs und LMMs
\begin{equation}\label{formel.hmms}
    HMMs>=M+SD
\end{equation}
\begin{equation}\label{formel.lmms}
    LMMs<=M-SD
\end{equation}

%---  
\subsection{Attentional Control Scale} \label{subsection.acs}
Bei der \textit{Aufmerksamkeitskontroll-Skala} (engl.: \gls{labelACS}) handelt es sich um einen Selbstbeurteilungsfragebogen, der von \citeA{Derryberry2002} für die Messung von individuellen Unterschieden in der Aufmerksamkeitskontrolle entwickelt wurde. Aufmerksamkeit entsteht in miteinander verbundenen Netzwerken im Gehirn, eine davon ist das anteriore Aufmerksamkeitssystem (engl.: anterior attentional system), welches als exekutive Kontrollfunktion über die restlichen Aufmerksamkeitsprozesse waltet \cite{Posner1998}. Aufgrund unterschiedlich vorgeschlagener Funktionen des anterioren Systems, wurde die Aufmerksamkeitskontroll-Skala als Instrument für die Messung allgemeiner Unterschiede in der selbstinitierten Aufmerksamkeitskontrolle entwickelt \cite{Derryberry2001}.\\
Der Fragebogen umfasste 20 Fragen, die Ursprünglich in zwei unterschiedlichen Skalen enthalten waren \cite{Derryberry1988}: Ausrichten der Aufmerksamkeit und Aufmerksamkeitsverschiebung. Unter der Ausrichtung der Aufmerksamkeit verstanden die Forscher die Kapazität, den Aufmerksamkeitsfokus absichtlich in ein gewünschte Richtung zu lenken, ohne sich dabei unabsichtlich von irrelevanten oder störenden Reizen ablenken zu lassen. Unter Aufmerksamkeitsverschiebung verstanden die Forscher die Kapazität, die Aufmerksamkeit absichtlich von einer Richtung in die andere Richtung zu lenken und dabei die Aufmerksamkeit nicht in eine spezifische Richtung zu fokussieren. In den letzten Jahren wurden die zwei Skalen unter dem Sammelbegriff der Aufmerksamkeitskontroll-Skala zusammengefasst, die als Messmittel für die Fähigkeit der Aufmerksamkeitskontrolle diente. Gemäss \citeA{Derryberry2002} ergab die Faktoranalyse des \gls{labelACS} bezogen auf die Fähigkeit einen Zusammenhang aus wechselseitigen Subfaktoren wie (a) sich auf etwas zu fokussieren, (b) zwischen verschiedenen Aufgaben hin und her zu wechseln und (c) eine flexible Gedankenkontrolle \cite{Derryberry2002}. \\
Die \gls{labelACS} wurde für diese Untersuchung aus dem Englischen übersetzt (siehe Anhang \ref{chap.appendix_attentionalControlScale}). Die Skala des ACS bestand aus 20 Fragen, die mittels vierstufiger Likert-Skala beantwortet werden konnten (1 = fast nie; 2 = manchmal; 3 = oft; 4 = immer). Je höher die Antwortzahl ausgefallen ist, desto höher war die Aufmerksamkeitskontrolle des Probanden (Max. 80 Punkte, Min. 20 Punkte). Hierbei ist zu beachten, dass 11 der 20 Fragen für die Bewertung invertiert werden mussten. Dies waren die Fragen 1 bis 3, 6 bis 8, 11, 12, 15, 16, 18 und 20.

%---
\subsection{Flourishing Scale und Scale of Positive and Negative Experience} \label{subsection.flourishingScale}
Die von \citeA{Diener:2010} entwickelte Skalen \textit{\gls{labelFS}} und \textit{\gls{labelSPANE}} sind Fragebögen für die Erfassung des psychologischen und sozialen Wohlbefindens. Mit diesen Skalen wurde das psychosoziale Aufblühen (aus dem Englischen von flourishing), basierend auf dem psychologischen und sozialem Wohlbefinden, und dem positiven und negativen Befinden, erfasst. \\
Die \gls{labelFS}-Skala wurde auf dem Hintergrund von sozialen Beziehungen entwickelt. Aus dem englischen übersetzt steht \textit{flourishing} im Zusammenhang mit Glück für menschliches Aufblühen, wachsen, entfalten und gutem Gedeihen \cite{Esch2014}. In dieser Arbeit wird vom menschlichen Aufblühen gesprochen, wenn auf diese Skala referenziert wird. Der Fragebogen erfasste menschliches Aufblühen in relevanten Bereichen wie Lebensinhalt, Beziehungen, Selbstwertgefühl, Gefühl der Kompetenz und Optimismus \cite{Silva2013}. Dieser Bereich von positivem Funktionieren scheint einen signifikanten Einfluss auf das persönliche Wohlbefinden zu haben \cite<e.g.,>{Ryan2000, Ryff1989}.
Die Skala umfasste acht Items aus wichtigen Bereichen des menschlichen Funktionierens, wie positive Beziehungen, Gefühle der Kompetenz und Führen eines zielgerichteten und sinnvollen Lebens \cite{Diener:2010}.\\ 
Die aus dem Englisch stammenden Fragen wurden für diese Arbeit ins Deutsche übersetzt (siehe dazu Anhang \ref{chap.appendix_fs}). Jedes Item konnte auf einer Skala von 1 - 7 beantwortet werden. Von 1 = trifft überhaupt nicht zu bis zu 7 = triff voll und ganz zu. Alle Fragen wurden in einer positiven Richtung formuliert. Die Skala verfügte über einen einzigen Ergebniswert für das psychologische Wohlbefinden. Die Summe der Punkte konnte zwischen 8 (trifft überhaupt nicht zu) und 56 (trifft voll und ganz zu) liegen. Hohe Werte deuteten darauf hin, dass sich die befragte Person in einem positiven Licht sieht, bezogen auf ihr Funktionieren und über psychologische Ressourcen und Stärken verfügte. Die Skala zeigte jedoch keine einzelnen Aspekte des Wohlbefindens auf. Vielmehr erstellte sie eine Übersicht aus positivem Funktionieren aus Bereichen des Lebens, die weithin als wichtig angeschauten werden. 
\par
Die Skala für das Erfassung von positiven und negativen Erfahrungen (\gls{labelSPANE}) beinhaltete einen zwölfteiligen Fragebogen, der aus je sechs Items für die Erfassung von positiven und sechs Items für die Erfassung von negativen Empfindungen bestand \cite{Diener:2010}. Mittels dieser 12 Fragen wurde eine weite Bandbreite von negativen und positiven Erleben und Gefühlen abgeschätzt, die mittels Erfassung aller erlebten Gefühlen der vergangenen vier Wochen hergeleitet wurde. Für die Skalen der positiven und der negativen Empfindungen waren drei Fragen genereller Natur (z.B.: positiv, negativ) und drei Fragen spezifischer Natur (z.B.: froh, traurig). Aufgrund der Erfassung von generellen positiven und negativen Gefühlen, erfasste der Fragebogen die gesamte Bandbreite von positiven und negativen Erlebnissen, inklusive spezifischer Gefühle, die unter Umständen in verschiedenen Kulturen unterschiedlich lauten. Dadurch wird angenommen, dass der Fragebogen kulturunabhängig funktionierte \cite{Silva2013}. Des Weiteren erfasste dieser Fragebogen nicht nur die erfreulichen und weniger erfreulichen Emotionen, sondern gab auch Zustände wie Interessen, Flow, positives Engagement und physikalischer Freude wieder.\\
Jede Frage des \gls{labelSPANE}-Fragebogens wurde auf einer fünfstufigen Likert Skala von 1 bis 5, von 1 = sehr selten oder nie, bis 5 = sehr oft oder immer, beantwortet. Die positive und die negative Skala wurden aufgrund der teilweise unabhängigen Typen der Empfindungen separat ausgewertet. Die einzelnen Fragebogenitems wurden aus dem Englischen ins Deutsche übersetzt und sind im Anhang \ref{chap.appendix_spane} aufgeführt. Die Summe der positiven Resultate ($SPANE-P$) sowie der negativen Resultate ($SPANE-N$) konnten zwischen einem Bereich von 6 und 30 liegen. Die Skala $SPANE-P$ beinhaltete die positiven Gefühle positiv, gut, angenehm, glücklich, froh und zufrieden. Die Skala $SPANE-N$ beinhaltete die negativen Gefühle negativ, schlecht, unangenehm, traurig, ängstlich und wütend. Diese beiden Skalen konnten miteinander kombiniert werden, indem die Resultate der negativen Gefühl-Skala von der positiven Gefühl-Skala abgezogen wurde. Die Summe dieser neuen Skala ($SPANE-B$) konnte zwischen -24 (am unglücklichsten) und 24 (höchste Affektstabilität) liegen. Probanden mit einem sehr hohen Ergebnis von 24 gaben an, dass sie selten oder nie negative Gefühle erleben und sehr oft oder immer positive Gefühle verspüren.


% Kapitel Datenerhebung
%---------------------
\section{Datenerhebung}\label{section.datenerhebung}
Die Datenerhebung erfolgte mittels elektronischer Online-Befragung. Diese Befragung wurde mit Hilfe der von der ZHAW zur Verfügung gestellten Umfragesoftware EFS Survey von QuestBack Unipark \cite{QuestBack2014} erstellt. Der Fragebogen wurde anhand der zur Verfügung gestellten Dokumentation erstellt. Die Befragung erfolgte zu Beginn des Frühlingssemesters 2014 und wurde für den Zeitraum von vier Wochen aktiv geschaltet. Der Zugang zur Befragung erfolgte über die von der QuestBack Unipark bereitgestellte Internetadresse, welcher zum Zeitpunkt der Befragung wie folgt lautete: \url{http://www.unipark.de/uc/multitasking_and_swb/}.\\
Die Umsetzung der Erhebungsinstrumente erfolgte anhand von der Software vorgegebenen Fragemasken. Ein Auszug des Fragebogens in Papierform ist im \nameref{chap.appendix_fragebogen} zu finden. Dabei ist zu beachten, dass die Umsetzung in Papierform von der originalen Umfrage in digitaler Form abweichen kann. Der Fragebogen beinhaltete zu Beginn eine kurze Einführung mit der Anonymitätserklärung der Datenerhebnung und dem Hinweis auf den Wettbewerb. Danach folgten die einzelnen Erhebungsinstrumente, beginnend mit der Erfassung der demographischen Daten, gefolgt vom Mediennutzungs-Fragebogen, dem Aufmerksamkeitskontroll-Skala und dem Fragebogen zum Wohlbefinden. Am Ende der Befragung konnten sich die Teilnehmer für die Teilnahme an einem Wettbewerb für Gutscheine im Athleticum Sportmarket \cite{Athleticum2014} eintragen. Neben dem Wettbewerb konnten sich die Teilnehmer für die Ergebnisse der Studie eintragen, die am Ende dieser Arbeit versendet werden. \\
Der Fragebogen wurde mittels Pretest auf seine Funktionalität hin überprüft und erfolgte in zwei Schritten. Einerseits erlaubte die von QuestBack Unipark zur Verfügung gestellte Software eine elektronische Testung des Fragebogens. Dabei ging es um die Wählbarkeit der Frageoptionen und der Erreichbarkeit aller aufgestellter Fragen. In einem weiteren Schritt wurden 15 willkürlich ausgewählte Testpersonen aus dem Bekanntenkreis des Autors als Testpersonen rekrutiert, mit der Anweisung, den Fragebogen auszufüllen. Damit sollte sichergestellt werden, dass die Fragen und deren Anweisung verständlich formuliert wurden und die Beantwortung der Fragen möglich war. Vor der Aktivschaltung des Fragebogens wurden alle bis dahin erfassten Daten der Tester gelöscht und die gesamte Umfrage frisch initialisiert. 


% Kapitel Datenaufbereitung
%---------------------
\section{Datenaufbereitung}\label{section.datenaufbereitung}
Die erhobenen Daten wurden mittels SPSS Version 20 für Mac OS X aufbereitet. Die Stichprobendaten wurden direkt aus dem Fragebogen Onlinewerkzeug QuestBack Unipark \cite{QuestBack2014} mittels Datenexport für SPSS importiert. Allfällige Fehlerquellen wurden bei der Datenbereinigung korrigiert. Dies wurde durch die Umfragesoftware erleichtert, indem nur abgeschlossene Datensätze exportiert wurden und eine erste Werteprüfung bei der Eingabe erfolgte (z.B.: es wurden nur Zahlen bei der Eingabe des Alters zugelassen). Fehlende Werte wurden bereits von der Umfragesoftware gesetzt und konnten innerhalb von SPSS mit einem Wertelabel versehen werden. Für die Berechnung des Mulitmedia Indexes $MMI$ \cite{Ophir2009}, die Aufmerksamkeitskontroll-Skala  $ACS$ \cite{Posner1998}, die Skala des menschlichen Aufblühens $FS$ \cite{Diener:2010} und die Skala für das Erfassung von positiven und negativen Erfahrungen $SPANE-P$ und $SPANE-N$ \cite{Diener:2010} wurden zusätzliche Variablen in SPSS erstellt und anhand der von Umfrage übernommenen Werten berechnet. 

% Kapitel Statistische Verfahren
%---------------------
\section{Statistische Verfahren}\label{section.statistischeVerfahren}
Für die Beantwortung der Fragestellung erfolgte nach der Datenerfassung und der Datenaufbereitung die Datenauswertung. In einem ersten Schritt wurde eine deskriptive, univariate Analyse der Stichprobendaten durchgeführt, um mögliche Fehler bei der Datenerfassung und/oder Ausreisser im Datensatz zu entdecken. Dies erfolgte mittels Häufigkeitsverteilung, um die verschiedenen Merkmalsausprägungen einer Variablen im Datensatz zu beschreiben und mittels Verteilungsparameter, die die Verteilung der Merkmalsausprägung charakterisieren (Lage- und Streuungsparameter). Des weiteren wurde die Schiefe der Häufigkeitsverteilung gegenüber der Normalverteilung überprüft. Die Prüfung der Normalverteilung wurde mittels Sichtvergleich vorgenommen. Dazu wurde eine Graphik der Häufigkeitsverteilung der Stichprobendaten mit einem Bild einer Normalverteilung verglichen. Diese optische Prüfung erfolgte mit Hilfe von SPSS und der Darstellung eines Histogramms, das mit einer Häufigkeitsauszählung und der Darstellung der Normalverteilungskurve ausgegeben wurde. Ebenfalls mit SPSS wurde zusätzlich ein Q-Q-Diagram (Quantile-Quantile-Plot) erstellt, welches die Stichprobendaten mit einer Gerade vergleicht, die eine Normalverteilung repräsentiert. Anschliessend an die deskriptive Analyse wurde in einem weiteren Schritt die statistischen Zusammenhänge der Stichprobendaten für die Haupthypothese und die Arbeitshypothesen berechnet. \par 
\textbf{Haupthypothese:} Ziel dieser Hypothese war es die Häufigkeit von Medien-Multitasking (MUQ) und die kognitive Fähigkeit (Aufmerksamkeitskontrolle) als treibenden Faktor für den Einfluss von Medien-Multitasking auf das subjektive Wohlbefinden zu benennen. Für diese Beantwortung wurde einerseits der Zusammenhang zwischen Medien-Multitasking und Aufmerksamkeitskontrolle und der Zusammenhang zwischen Aufmerksamkeitskontrolle und subjektiven Wohlbefinden mittels Produkt-Moment-Korrelationskoeffizienten (Pearson-Korrelations-Koeffizienten) berechnet (siehe Arbeitshypothesen weiter unten). Anschliessend wurde mittels Partialkorrelation der Scheinzusammenhang der Variable Aufmerksamkeitskontrolle berechnet. Das Signifikanzniveau der Analyse liegt bei $p=.05$. Es wurden die Werte gemäss \cite{Cohen1988} übernommen, um Aussagen über die Stärke des Effekts zu machen ($r=.10$ -- kleine Effektstärke; $r=.30$ -- mittlerer Effektstärke; $r=.50$ -- grosse Effektstärke). Zudem wurden heterogene Untergruppen weitgehend aufgeteilt und jeweils einen gesonderten Korrelationskoeffizienten gebildet \cite{Renkewitz2008}. Dies wurde angewendet, um verdeckte Korrelationen zwischen heterogenen Untergruppen aufzudecken \cite{Ebermann2014}.

\textbf{Arbeitshypothesen:} Ziel der Arbeitshypothesen war es den Zusammenhang zwischen Medien-Multitasking und dem subjektiven Wohlbefinden, den Zusammenhang zwischen der kognitiven Fähigkeit und dem subjektiven Wohlbefinden und den Zusammenhang zwischen der kognitiven Fähigkeit und der Häufigkeit von Medien-Multitasking zu belegen. Dies wurde mittels Produkt-Moment-Korrelationskoeffizienten (Pearson-Korrelations-Koeffizienten) berechnet. Das Signifikanzniveau der Analyse liegt bei $p=.05$. Wiederum wurden heterogene Untergruppen aufgeteilt, um verdeckte Korrelationen aufzudecken \cite{Ebermann2014}. \par







