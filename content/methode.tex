%%%%%%%%%%%%%%%%%%%%%%%%%%%%%%%%%%%%%%%%%%%%%%%%%%%%%%%%%%%%%%%%%
%_____________ ___    _____  __      __ 
%\____    /   |   \  /  _  \/  \    /  \  Institute of Applied
%  /     /    ~    \/  /_\  \   \/\/   /  Psychology
% /     /\    Y    /    |    \        /   Zuercher Hochschule 
%/_______ \___|_  /\____|__  /\__/\  /    fuer Angewandte Wissen.
%        \/     \/         \/      \/                           
%%%%%%%%%%%%%%%%%%%%%%%%%%%%%%%%%%%%%%%%%%%%%%%%%%%%%%%%%%%%%%%%%
%
% Project     : Bachelorarbeit
% Title       : Methode
% File        : methode.tex Rev. 00
% Date        : 06.12.2013
% Author      : Till J. Ernst
%
%%%%%%%%%%%%%%%%%%%%%%%%%%%%%%%%%%%%%%%%%%%%%%%%%%%%%%%%%%%%%%%%%
\glsresetall
\let\raggedsection\centering 
\chapter{Methode}
%\chapter*{Methode}\label{chap.methode}
%\addcontentsline{toc}{chapter}{Methode}
\let\raggedsection\raggedright 

% Kapitel Studiendesign
%---------------------
\section{Studiendesign}\label{section.studiendesign}
In der vorliegenden Studie handelt es sich um eine empirische Querschnitts-Studie mit quantitativem Charakter. Die Generierung der Daten für die Beantwortung der Fragestellung erfolgte mittels Internet-Fragebogen. Dazu wurden die für die Hypothesen relevanten Parameter empirisch erhoben und mittels deskriptiver Statistik ausgewertet. Um eine möglichst umfassende Beantwortung der Forschungsfrage zu ermöglichen, wurde eine grösstmögliche Stichprobe von Probanden angestrebt, die täglich Medien nutzen. Bei den befragten Probanden handelte es sich um Studierende, die aktuell ein Studium an einer Universität oder einer Fachhochschule in der Schweiz absolvieren. Der Fragebogen wurde in deutscher Sprache verfasst, weshalb ausschliesslich Schulen der deutschsprachigen Schweiz berücksichtigt wurden.\\
Durch diese Studie sollen mögliche Tendenzen aufgezeigt werden. Durch die Befragung einer spezifischen Bevölkerungsgruppe können die Resultate nicht verallgemeinert werden und dienen einer exemplarischen Betrachtung.

%---------------------
\subsection{Hauptzielparameter}\label{subsection.hauptzielparameter}
Die folgenden Hauptzielparameter dienten zur Beantwortung der oben genannten Fragestellung (siehe \nameref{section.fragestellung}). Einen möglichen Einflusses der Medien-Multitasking Nutzung auf das subjektive Wohlbefinden wurde mittels bereits existierender Fragebögen erhoben. Diese Fragebögen dienten zur Erfassung der Häufigkeit der Nutzung von einzelnen Medien, der Häufigkeit der simultanen Mediennutzung, der Fähigkeit der simultanen Mediennutzung und das aktuelle subjektive Wohlbefinden der Probanden. Mittels Erhebung der demographischen Daten wurde das Geschlecht, Alter, Studienform (Teilzeit- oder Vollzeitstudium), Schulform (Unviersität oder Fachhochschule), berufliche Tätigkeit, Zivilstand und Familienstatus erhoben. Diese Daten dienten dazu die Stichprobe in verschiedene Gruppen aufteilen zu können, um mögliche Einflüsse auf das subjektive Wohlbefinden anhand demographischer Daten zu untersuchen.

%---------------------
\subsection{Durchführung der Untersuchung}
Die Durchführung dieser Arbeit erfolgte im Rahmen einer Bachelorarbeit. Die gesamte Projektdauer betrug sechs Monate und wurde im Zeitrahmen des Frühlingssemesters 2014 durchgeführt. Die Übersicht der Planung der einzelnen Arbeitsschritte ist im \nameref{chap.appendix_projektPlanung} zu finden.

% Kapitel Auswahl der Versuchspersonen
%---------------------
\section{Auswahl der Versuchspersonen}\label{section.auswahlVersuchsp}
Gemäss Fragestellung wurde der Einfluss von Medien-Multitasking auf das subjektive Wohlbefinden von Studierenden untersucht. Dazu wurden Studierende aus der deutschsprachigen Schweiz, die an einer Universität oder Fachhochschule studieren, mittels Fragebogen zu ihrem Medien-Multitasking Verhalten befragt. Die Schulen wurden anhand der von SWITCH \cite{Switch2014} zur Verfügung gestellten Liste der Universitäten und Fachhochschulen der Schweiz ausgewählt (SWITCH ist ein Hochleistungsnetzwerk der Schweiz, welches die Verbindung von Internetnutzer weltweit und für die Vernetzung von Fachhochschulen im Bereich Schweizer Wissenschaft zuständig ist). Prinzipiell wurde versucht alle Studierenden der ausgewählten Schulen zu erreichen, die entweder im Teilzeit- oder Vollzeitmodus studieren. \\
Mit Hilfe der Poweranalyse \cite{Faul2009} wurde für die Sitchprobenumfangsplanung die angestrebte  Stichprobengrösse im Vorfeld auf $N = 134$ angesetzt. Diese Grösse wurde mittels der Software G*Power \cite{Gpower2014} für die Beantwortung der ungerichteten Hauptfragestellung mittels Bivariater-Korrelation berechnet. Folgende Werte wurden dazu verwendet: Es wurde von einem gewünschten mittlerer Effekt $r = 0.3$  \cite{Cohen1988} für den Korrelationskoeffizienten ausgegangen, einem Signifikanzniveau von $\alpha=0.05$ und einer Teststärke von $1-\beta=0.95$.
%---------------------
\subsection{Rekrutierung}\label{subsection.rekrutierung}
Die Rekrutierung der Studierenden an den Universitäten und Fachhochschulen der deutschen Schweiz erfolgte wenn immer möglich direkt über die Mailverteiler der einzelnen Schulen (Liste der angeschriebenen Schulen siehe \nameref{chap.appendix_schulListe}). Damit für diese Untersuchung eine möglichst grosse Stichprobe generiert werden konnte, wurde wenn immer möglich versucht innerhalb der einzelnen Schulen die einzelnen Departemente separat anzuschreiben. Da diese Mailverteiler der Schulen in den meisten Fällen nicht öffentlich zugänglich sind, wurde direkt auf den einzelnen Schulsekretariaten oder Departementsstellen (z.B. Dekanat) versucht die Umfrage zu platzieren. Dies wurde teilweise mittels telefonischem Kontakt und teilweise per vordefiniertem Mail vorgenommen. Das Mailing, das zu diesem Zweck erstellt wurde, ist in \nameref{chap.appendix_mailing} zu finden. In diesem Schreiben wurden die einzelnen Schulen um Unterstützung gebeten, in dem sie die Umfrage an ihre Studierenden weiter leiteten. Für die Weiterleitung an die Studierenden wurde dem dem Mailing eine weitere Nachricht mit dem Link zum Fragebogen angehängt. \\
Schulen die eher konservativ gegenüber Umfragebögen anderer Schulen eingestellt sind, lehnten das Weiterleiten der Umfrage an die Studierenden ab. In diesen Fällen war es jedoch oftmals möglich, einen Aushang am schwarzen Brett der Schule anbringen zu lassen. Dieser Anhang wurde auf Wunsch der entsprechenden Stelle nachgeliefert und von den Studiumsverantwortlichen selbständig angebracht (siehe Aushang \nameref{chap.appendix_aushang}). \\
In dieser Untersuchung wurde das Mailing an über 70 Stellen der in Kapitel '\nameref{section.auswahlVersuchsp}' genannten Schulen und deren Departemente versendet.
%---
\subsection{Einschlusskriterien}\label{subsection.einschlusskriterien}
Alle Studierende von Universitäten und Fachhochschulen aus dem deutschsprachigen Raum, die einem Teilzeit- oder einem Vollzeitstudium nachgehen.
%---
\subsection{Ausschlusskriterien}\label{subsection.ausschlusskriterien}
Ausgeschlossen werden Studierende eines Fernstudiums, da hier weitere Bedingungen gegeben sind, die eine Vergleichbarkeit erschwert. Der Fragebogen wird lediglich auf Deutsch zur Verfügung gestellt. Aus diesem Grund wurden nur Personen berücksichtigt, die der deutschen Sprache mächtig sind. 


% Kapitel Erhebungsinstrumente
%---------------------
\section{Erhebungsinstrumente}\label{section.erhebungsinstrumente}
Die Erhebungsinstrumente der vorliegenden Arbeit wurden basierend auf bereits existierenden Fragebögen, der Erfassung von soziodemographischen Daten zusammengestellt und in Form einer Onlineumfrage umgesetzt. In den folgenden Kapiteln werden die zur Überprüfung der Hypothesen verwendeten Fragebögen in der Reihenfolge der Durchführung beschrieben. Diese Fragebögen wurden aus dem Englischen übersetzt, deren Übersetzungen sich im Anhang (\nameref{chap.appendix_mediaUseQuestionnaire} und folgende) befinden. \\
Bei den soziodemographischen Daten wurden folgende Parameter erfasst: Geschlecht, Alter, Studium (Universität oder Fachhochschule), Studienrichtung, Studentenstatus (Vollzeit- oder Teilzeitstudium), Berufsstand (arbeitstätig neben dem Studium), Zivilstand und Familienstatus (Kinder). \\
%---
\subsection{Media Use Questionnaire und Media Multitasking Index}\label{subsection.muq}
Für die Bestimmung, ob es sich in der Stichprobe um Personen handelt die starkes oder schwaches Medien-Multitasking betreiben, wurde der Media Multitasking Index verwendet. Dieser Index wurde mittels Fragebogenerhebung berechnet, der die Häufigkeit der Nutzung von einzelnen Medien und die gleichzeitige Nutzung verschiedener Medien miteinander misst. Der Media Multitasking Index und der Fragebogen zur Häufigkeit von Mediennutzung basieren auf dem von \citeA{Ophir2009} entwickelten \textit{Media Multitasking Index (MMI)} und dem \textit{Media Use Questionnaire (MUQ)}. Diese Instrumente wurden für diese Arbeit aus dem Englischen ins Deutsche übersetzt (siehe \nameref{chap.appendix_mediaUseQuestionnaire}). Der Fragebogen registriert wie viele Stunden eine Person eine von 12 vordefinierten Medienformen in einer Woche nutzt und wie oft diese Person diese Medienform mit anderen Medienformen gleichzeitig verwendet. \\ 
In dieser Umfrage wurde anstelle von Stunden in einer Woche, die durchschnittliche Anzahl Minuten an einem Tag erfasst. Dies erschien für Medien wie Textnachrichten einfacher abschätzbar zu sein. Durch diese Anpassung mussten anschliessend die erfassten Minuten in Stunden einer Woche hochgerechnet werden. Dies wurde mittels folgender Formel berechnet: \(\frac{n_{i} \times 7_{(Tage)}}{60}\), wobei \(n_{i}\) die Anzahl Minuten pro Tag sind, die für die Nutzung eines Mediums verwendet wurden. Die 12 erfassten Medienformen sind Druckmedien, Fernsehen, Online Video (wie zum Beispiel Youtube), Musik, Nicht-musikalische Audiomedien (z.B. Hörbücher, Podcast, etc.), Video oder Computer Games, Telefonieren (Mobil- und / oder Festnetz), Instant Messaging (z.B. Skype, Windows Live Messenger, Yahoo Messenger, etc.), SMS (Textnachrichten), Email, Internet-Surfen und Andere Computerbasierte-Tätigkeiten (z.B. Word, Videobearbeitung, programmieren, etc.). \\
Neben der Nutzungsdauer einzelner Medien wurde die gleichzeitige Nutzung verschiedener Medien mittels Medien-Multitasking-Matrix gebildet. Diese besteht im Wesentlichen aus den 12 definierten Medien, die in der vertikalen Spalte als Hauptmedien und in der horizontalen Zeile als Nebenmedien aufgelistet werden. Die Aufgabe für die Probanden bestand darin, zu jedem Hauptmedium eine Schätzung abzugeben, wie oft ein allfällig weiteres Nebenmedium gleichzeitig zusammen mit dem Hauptmedium verwendet wird. Die Bewertung in der Matrize erfolgte mittels definierter Auswahl \textit{meistens}, \textit{etwas}, \textit{wenig} oder \textit{nie}.\\
Mittels der Nutzungsdauer und der gleichzeitigen Nutzung einzelner Medien wird der Medien Multitasking Index gebildet. Hierbei ist zu beachten, dass die Textnachrichten in der ursprünglichen Formel gemäss \cite{Ophir2009}  nicht berücksichtigt wurden. Die Autoren gingen davon aus, dass die Abschätzung der Nutzung von Textnachrichten in Stunden nicht exakt vorgenommen werden konnte. Trotzdem wurde das Medium im ursprünglichen Fragebogen als Sekundärmedium belassen. In dieser Studie wurden die Textnachrichten dazu gezählt, da die Abschätzung innerhalb des Fragebogens in Minuten vorgenommen werden konnte und die Hochrechnung in Stunden in einem weiteren Schritt erfolgte. Die angepasste und die ursprüngliche Formel für die Berechnung des Indexes lautet:
%Formel
\begin{equation}\label{formel.mmiext}
    MMI_{angepasst}=\sum_{i=1}^{12} \frac{m_{i} \times \frac{n_{i} \times 7_{(Tage)}}{60}}{h_{total}}
\end{equation}
%Formel
\begin{equation}\label{formel.mmi}
    MMI=\sum_{i=1}^{11} \frac{m_{i} \times h_{i}}{h_{total}}
\end{equation}

In der ursprünglichen Formel steht \(h_{i}\) für die Anzahl Stunden pro Woche, die für ein Hauptmedium \(i\) verwendet wurde. In der angepassten Formel wird die Anzahl Stunden pro Woche aus der Anzahl Minuten pro Tag \(n_i\) auf eine Woche hochgerechnet (\(h_{i}=\frac{n_{i} \times 7}{60}\)). \(h_{total}\) steht für die totale Anzahl Stunden aller Medien in einer Woche. \(m_i\) beinhaltet die Summe der Abschätzungen des Probanden bezüglich seiner Nutzung eines oder mehrerer Nebenmedien, während dem er ein Hauptmedium nutzt. Diese Summenbildung ergibt die durchschnittliche Anzahl der Medien, die neben dem Hauptmedium verwendet werden. Den Abschätzungen des Probanden wurden numerische Werte wie folgt zugewiesen: 1 = \textit{meistens}, 0.67 = \textit{etwas}, 0.33 = \textit{wenig} oder 0 = \textit{nie}. \\
Der resultierende Medien Multitasking Index \
$MMI$ (resp. $MMI_{angepasst}$) kennzeichnet die durchschnittliche Menge von Medien Multitasking, die während einer typischen Stunde Mediennutzung auftritt. Dieser Index lässt die Unterteilung zu in Personen, die starkes Medien-Multitasking betreiben ($HMMs$) und Personen, die schwach Medien-Multitasking betreiben ($LMMs$). Für alle die ein mittleres Medien-Multitasking betreiben, wird die Bezeichnung $NMMs$ verwendet. Für die Unterscheidung dieser beiden Bereiche wurde die Standardabweichung hinzugezogen (eine Standardabweichung oder mehr über dem Mittel für starkes Betreiben von Multitasking und eine Standardabweichung oder weniger unter dem Mittel für schwaches Betreiben von Multitasking): 
%Formel HMMs und LMMs
\begin{equation}\label{formel.hmms}
    HMMs>=M+SD
\end{equation}
\begin{equation}\label{formel.hmms}
    LMMs<=M-SD
\end{equation}

%---  
\subsection{Attentional Control Scale} \label{subsection.acs}
Bei der \textit{Aufmerksamkeitskontroll-Skala} (engl.: Attentional Control Scale - ACS) handelt es sich um einen Selbstbeurteilungsfragebogen, der von \citeA{Derryberry2002} für die Messung von individuellen Unterschieden in der Aufmerksamkeitskontrolle entwickelt wurde. Aufmerksamkeit entsteht in miteinenader verbundenen Netzwerken im Gehirn, eine davon ist das anteriore Aufmerksamkeitssystem (engl.: anterior attentional system), welches als exekutive Kontrollfunktion über die restlichen Aufmerksamkeitsprozesse waltet \cite{Posner1998}. Aufgrund unterschiedlich vorgeschlagener Funktionen des anterioren Systems, wurde die Aufmerksamkeitskontroll-Skale als Instrument für die Messung allgemeiner Unterschiede in der selbstinitierten Aufmerksamkeitskontrolle entwickelt \cite{Derryberry2001}.\\
Der Fragebogen umfasst 20 Fragen, die Ursprünglich in zwei unterschiedlichen Skalen enthalten waren \cite{Derryberry1988}: Ausrichten der Aufmerksamkeit und Aufmerksamkeitsverschiebung. Unter der Ausrichtung der Aufmerksamkeit verstanden die Forscher die Kapazität, den Aufmerksamkeitsfokus absichtlich in ein gewünschte Richtung zu lenken, ohne sich dabei unabsichtlich von irrelevanten oder störenden Reizen ablenken zu lassen. Unter Aufmerksamkeitsverschiebung verstanden die Forscher die Kapazität, die Aufmerksamkeit absichtlich von einer Richtung in die andere Richtung zu lenken und dabei die Aufmerksamkeit nicht in eine spezifische Richtung zu fokussieren. In den letzten Jahren wurden die zwei Skalen unter dem Sammelbegriff der Aufmerksamkeitskontroll-Skala zusammengefasst, die als Messmittel für die Fähigkeit der Aufmerksamkeitskontrolle von Personen dient. Gemäss \citeA{Derryberry2002} ergab die Faktoranalyse des ACS - Fragebogens einen Zusammenhang aus wechselseitigen Subfaktoren bezogen auf die Fähigkeiten wie (a) sich auf etwas zu fokussieren, (b) zwischen verschiedenen Aufgaben hin und her zu wechseln und (c) eine flexible Gedankenkontrolle \cite[S.~226]{Derryberry2002}. \\
Der ACS wurde für diese Untersuchung aus dem Englischen übersetzt (siehe \nameref{chap.appendix_attentionalControlScale}). Die Skala des ACS besteht aus 20 Fragen, die mittels vierstufiger Likert-Skala beantwortet werden konnten (1 = fast nie; 2 = manchmal; 3 = oft; 4 = immer). Je höher die Antwortzahl ausfällt, desto höher ist die Aufmerksamkeitskontrolle des Probanden (Max. 80 Punkte, Min. 20 Punkte). Hierbei ist zu beachten, dass 11 der 20 Fragen für die Bewertung invertiert werden müssen. Diese sind die Fragen 1 bis 3, 6 bis 8, 11, 12, 15, 16, 18 und 20 (siehe dazu \nameref{chap.appendix_attentionalControlScale}).

%---
\subsection{Flourishing Scale und Scale of Positive and Negative Experience} \label{subsection.flourishingScale}
Die von \citeA{Diener:2010} entwickelten Methoden, \textit{The Flourishing Scale (FS)} und \textit{The Scale of Positive and Negative Epxerience (SPANE)} sind Fragebögen für die Erfassung des psychologischen und sozialen Wohlbefindens. Mit diesen Skalen soll das psychosoziale Aufblühen (aus dem Englischen von flourishing), basierend auf dem psychologischen und sozialem Wohlbefinden, und dem positiven und negativen Befinden, erfasst werden. \\
Die FS-Skala wurde auf dem Hintergrund von sozialen Beziehungen entwickelt. Aus dem englischen übersetzt steht \textit{flourishing} im Zusammenhang mit Glück für menschliches Aufblühen, wachsen, entfalten und gutem Gedeihen \cite{Esch2014}. Im Folgenden wird vom menschlichen Aufblühen gesprochen, wenn auf diese Skala referenziert wird. Der Fragebogen erfasst menschliches Aufblühen in relevanten Bereichen wie Lebensinhalt, Beziehungen, Selbstwertgefühl, Gefühl der Kompetenz und Optimismus \cite{Silva2013}. Dieser Bereich von positivem Funktionieren scheint einen signifikanten Einfluss auf das persönliche Wohlbefinden zu haben \cite<e.g.,>{Ryan2000, Ryff1989}. Die Skala umfasst acht Items aus wichtigen Bereichen des menschlichen Funktionierens, wie positive Beziehungen, Gefühle der Kompetenz und Führen eines zielgerichteten und sinnvollen Lebens \cite{Diener:2010}. Die aus dem Englisch stammenden Fragen wurden für diese Arbeit ins Deutsche übersetzt (siehe dazu Anhang \ref{chap.appendix_fs}). Jedes Item konnte auf einer Skala von 1 - 7 beantwortet werden. Von 1 = trifft überhaupt nicht zu bis zu 7 = triff voll und ganz zu. Alle Fragen wurden in einer positiven Richtung formuliert. Die Skala verfügt über einen einzigen Ergebniswert für das psychologische Wohlbefinden. Die Summe der Punkte kann zwischen 8 (trifft überhaupt nicht zu) und 56 (trifft voll und ganz zu) liegen. Hohe Werte deuten darauf hin, dass sich die befragte Person bezogen auf ihr Funktionieren in einem positiven Licht sieht und über psychologische Ressourcen und Stärken verfügt. Die Skala zeigt jedoch keine einzelnen Aspekte des Wohlbefindens auf. Vielmehr erstellt sie eine Übersicht aus positivem Funktionieren in weithin als wichtig angeschauten Bereichen aus dem Leben heraus. \\ 
Die Skala für das Erfassung von positiven und negativen Erfahrungen (SPANE) beinhaltet einen zwölfteiligen Fragebogen, bestehend aus je sechs Items für die Erfassung von positiven und sechs Items für die Erfassung von negativen Empfindungen \cite{Diener:2010}. Mittels dieser 12 Fragen wird eine weite Bandbreite von negativen und positiven Erleben und Gefühlen abgeschätzt, die mittels Erfassung aller erlebten Gefühlen in den vergangenen vier Wochen hergeleitet werden.     
Für beide Skalen, positive und negative Empfindungen, sind drei Fragen genereller Natur (z.B.: positiv, negativ) und drei Fragen spezifischer Natur (z.B.: froh, traurig). Aufgrund der Erfassung von generellen positiven und negativen Gefühlen, erfasst der Fragebogen die gesamte Bandbreite von positiven und negativen Erlebnissen, inklusive spezifischer Gefühle, die unter Umständen in verschiedenen Kulturen unterschiedlich lauten. Dadurch wird angenommen, dass der Fragebogen kulturunabhängig funktioniert \cite{Silva2013}. Des Weiteren erfasst dieser Fragebogen nicht nur die erfreulichen und weniger erfreulichen Emotionen, sondern gibt auch Zustände wie Interessen, Flow, positives Engagement und physikalischer Freude wieder.
Jede Frage des SPANE - Fragebogens wird auf einer fünfstufigen Likert Skala von 1 bis 5, von 1 = sehr selten oder nie, bis 5 = sehr oft oder immer, beantwortet. Die positive und die negative Skala werden aufgrund der teilweise unabhängigen Typen der Empfindungen separat ausgewertet. Die einzelnen Fragebogenitems wurden aus dem Englischen ins Deutsche übersetzt und sind im Anhang \ref{chap.appendix_spane} aufgeführt. Die Summe der positiven Resultate ($SPANE-P$) sowie der negativen Resultate ($SPANE-N$) können zwischen einem Bereich von 6 und 30 liegen. Die Skala $SPANE-P$ beinhaltet die positiven Gefühle positiv, gut, angenehm, glücklich, froh und zufrieden. Die Skala $SPANE-N$ beinhaltet die negativen Gefühle negativ, schlecht, unangenehm, traurig, ängstlich und wütend. Diese beiden Skalen können miteinander kombiniert werden, indem die Resultate der negativen Gefühle-Skala von der positiven Gefühle-Skala abgezogen wird. Die Summe dieser neuen Skala ($SPANE-B$) kann zwischen -24 (am unglücklichsten) und 24 (höchste Affektstabilität) liegen. Probanden mit einem sehr hohen Ergebnis von 24 geben an, dass sie selten oder nie negative Gefühle erleben und sehr oft oder immer positive Gefühle verspüren.


% Kapitel Datenerhebung
%---------------------
\section{Datenerhebung}\label{section.datenerhebung}
Die Datenerhebung erfolgte mittels elektronischer Online-Befragung. Diese Befragung wurde mit Hilfe der von der ZHAW zur Verfügung gestellten Umfragesoftware EFS Survey von QuestBack Unipark \cite{QuestBack2014} erstellt. Der Fragebogen wurde anhand der zur Verfügung gestellten Dokumentation erstellt. Die Befragung erfolgte zu Beginn des Frühlingssemesters 2014 und wurde für den Zeitraum von vier Wochen aktiv geschaltet. Der Zugang zur Befragung erfolgte über die von der QuestBack Unipark bereitgestellte Internetadresse, welcher zum Zeitpunkt der Befragung wie folgt lautete: \url{http://www.unipark.de/uc/multitasking_and_swb/}.\\
Die Umsetzung der Erhebungsinstrumente erfolgte anhand von der Software vorgegebenen Fragemasken. Ein Auszug des Fragebogens in Papierform ist im \nameref{chap.appendix_fragebogen} zu finden. Dabei ist zu beachten, dass die Umsetzung in Papierform von der originalen Umfrage in digitaler Form abweichen kann. Der Fragebogen beinhaltete zu Beginn eine kurze Einführung mit der Anonymitätserklärung der Datenerhebnung und dem Hinweis auf den Wettbewerb. Danach folgten die einzelnen Erhebungsinstrumente, beginnend mit der Erfassung der demographischen Daten, gefolgt vom Mediennutzungs-Fragebogen, dem Aufmerksamkeitskontroll-Skala und dem Fragebogen zum Wohlbefinden. Am Ende der Befragung konnten sich die Teilnehmer für die Teilnahme an einem Wettbewerb für Gutscheine im Athleticum Sportmarket \cite{Athleticum2014} eintragen. Neben dem Wettbewerb konnten sich die Teilnehmer für die Ergebnisse der Studie eintragen, die am Ende dieser Arbeit versendet werden. \\
Der Fragebogen wurde mittels Pretest auf seine Funktionalität hin überprüft und erfolgte in zwei Schritten. Einerseits erlaubte die von QuestBack Unipark zur Verfügung gestellte Software eine elektronische Testung des Fragebogens. Dabei ging es um die Wählbarkeit der Frageoptionen und der Erreichbarkeit aller aufgestellter Fragen. In einem weiteren Schritt wurden 15 willkürlich ausgewählte Testpersonen aus dem Bekanntenkreis des Autors als Testpersonen rekrutiert, mit der Anweisung, den Fragebogen auszufüllen. Damit sollte sichergestellt werden, dass die Fragen und deren Anweisung verständlich formuliert wurden und die Beantwortung der Fragen möglich war. Vor der Aktivschaltung des Fragebogens wurden alle bis dahin erfassten Daten der Tester gelöscht und die gesamte Umfrage frisch initialisiert. 


% Kapitel Datenaufbereitung
%---------------------
\section{Datenaufbereitung}\label{section.datenaufbereitung}
Die erhobenen Daten wurden mittels SPSS Version 20 für Mac OS X aufbereitet \cite{Spss2011}. Die Stichprobendaten wurden direkt aus dem Fragebogen Onlinewerkzeug QuestBack Unipark \cite{QuestBack2014} mittels Datenexport für SPSS importiert. Allfällige Fehlerquellen wurden bei der Datenbereinigung korrigiert. Dies wurde durch die Umfragesoftware erleichtert, indem nur abgeschlossene Datensätze exportiert wurden und eine erste Werteprüfung bei der Eingabe erfolgte (z.B.: es wurden nur Zahlen bei der Eingabe des Alters zugelassen). Fehlende Werte wurden bereits von der Umfragesoftware gesetzt und konnten innerhalb von SPSS mit einem Wertelabel versehen werden. Für die Berechnung des Mulitmedia Indexes $MMI$ \cite{Ophir2009}, die Aufmerksamkeitskontroll-Skala  $ACS$ \cite{Posner1998}, die Skala des menschlichen Aufblühens $FS$ \cite{Diener:2010} und die Skala für das Erfassung von positiven und negativen Erfahrungen $SPANE-P$ und $SPANE-N$ \cite{Diener:2010} wurden zusätzliche Variablen in SPSS erstellt und anhand der von Umfrage übernommenen Werten berechnet. 

% Kapitel Statistische Verfahren
%---------------------
\section{Statistische Verfahren}\label{section.statistischeVerfahren}
Für die Beantwortung der Fragestellung erfolgte nach der Datenerfassung und der Datenaufbereitung die Datenauswertung. In einem ersten Schritt wurde eine deskriptive, univariate Analyse der Stichprobendaten durchgeführt, um mögliche Fehler bei der Datenerfassung und/oder Ausreisser im Datensatz zu entdecken. Dies erfolgte mittels Häufigkeitsverteilung, um die verschiedenen Merkmalsausprägungen einer Variablen im Datensatz zu beschreiben und mittels Verteilungsparameter, die die Verteilung der Merkmalsausprägung charakterisieren (Lage- und Streuungsparameter). Des weiteren wurde die Schiefe der Häufigkeitsverteilung gegenüber der Normalverteilung überprüft. Die Prüfung der Normalverteilung wurde mittels Sichtvergleich vorgenommen. Dazu wurde eine Graphik der Häufigkeitsverteilung der Stichprobendaten mit einem Bild einer Normalverteilung verglichen. Diese optische Prüfung erfolgte mit Hilfe von SPSS und der Darstellung eines Histogramms, das mit einer Häufigkeitsauszählung und der Darstellung der Normalverteilungskurve ausgegeben wurde. Ebenfalls mit SPSS wurde zusätzlich ein Q-Q-Diagram (Quantile-Quantile-Plot) erstellt, welches die Stichprobendaten mit einer Gerade vergleicht, die eine Normalverteilung repräsentiert. Anschliessend an die deskriptive Analyse wurde in einem weiteren Schritt die statistischen Zusammenhänge der Stichprobendaten für die Haupthypothese und der Nebenhypothese berechnet. \par 
Haupthypothese: Ziel dieser Hypothese war es den Zusammenhang zwischen Medien-Multitasking und dem subjektiven Wohbefinden zu belegen. Dies wurde mittels Produkt-Moment-Korrelationskoeffizienten (Pearson-Korrelations-Koeffizienten) berechnet. Diese Berechnung erfolgte zwischen den Variablen $MMI$ und $FS$, respektive $MMI$ und SPANE ($SPANE-B$, $SPANE-P$ und $SPANE-N$).  Das Signifikanzniveau der Analyse liegt bei $p=0.05$. \par
Nebenhypothese: Ziel dieser Hypothese war es die Aufmerksamkeitskontrolle als treibenden Faktor für den Einfluss von Medien-Multitasking auf das subjektive Wohlbefinden zu benennen. Für diese Beantwortung wurde einerseits der Zusammenhang zwischen Medien-Multitasking und Aufmerksamkeitskontrolle und der Zusammenhang zwischen Aufmerksamkeitskontrolle und subjektiven Wohlbefinden mittels Produkt-Moment-Korrelationskoeffizienten (Pearson-Korrelations-Koeffizienten) berechnet. Anschliessend wurde mittels Partialkorrelation der Scheinzusammenhang der Variable Aufmerksamkeitskontrolle berechnet. Das Signifikanzniveau der Analyse liegt bei $p=0.05$.







