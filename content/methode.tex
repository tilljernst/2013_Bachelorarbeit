%%%%%%%%%%%%%%%%%%%%%%%%%%%%%%%%%%%%%%%%%%%%%%%%%%%%%%%%%%%%%%%%%
%_____________ ___    _____  __      __ 
%\____    /   |   \  /  _  \/  \    /  \  Institute of Applied
%  /     /    ~    \/  /_\  \   \/\/   /  Psychology
% /     /\    Y    /    |    \        /   Zuercher Hochschule 
%/_______ \___|_  /\____|__  /\__/\  /    fuer Angewandte Wissen.
%        \/     \/         \/      \/                           
%%%%%%%%%%%%%%%%%%%%%%%%%%%%%%%%%%%%%%%%%%%%%%%%%%%%%%%%%%%%%%%%%
%
% Project     : Bachelorarbeit
% Title       : Methode
% File        : methode.tex Rev. 00
% Date        : 06.12.2013
% Author      : Till J. Ernst
%
%%%%%%%%%%%%%%%%%%%%%%%%%%%%%%%%%%%%%%%%%%%%%%%%%%%%%%%%%%%%%%%%%
\glsresetall
\let\raggedsection\centering 
\mychapter{2}{Methode}
%\chapter*{Methode}\label{chap.methode}
%\addcontentsline{toc}{chapter}{Methode}
\let\raggedsection\raggedright 

% Kapitel Studiendesign
%---------------------
\section{TBD: Studiendesign}\label{section.studiendesign}
Bei dieser Bachelorarbeit handelt es sich um eine empirische Querschnitts-Studie mit quantitativem Charakter. In dieser Arbeit wurde mittels Fragebogen in erster Linie das subjektive Wohlbefinden, die Häufigkeit von angewendetem Medien-Multitasking und die Fähigkeit, wie Medien-Multitasking betrieben wird, empirisch erhoben und mittels deskriptiver Statistik ausgewertet worden.
\subsection{TBD: Hauptzielparameter}\label{subsection.hauptzielparameter}
\begin{itemize}
    \item einzelne Variablen von den Tests wie Häufigkeit von MMT und SWB
    \item Demographische Daten    
\end{itemize}

\subsection{TBD: Durchführung der Untersuchung}\label{subsection.durchführung}

% Kapitel Auswahl der Versuchspersonen
%---------------------
\section{Auswahl der Versuchspersonen}\label{section.auswahlVersuchsp}
Gemäss Fragestellung wurde der Einfluss von Medien-Multitasking auf das subjektive Wohlbefinden von Studierenden untersucht. Dazu wurden Studierende aus der deutschsprachigen Schweiz, die an einer Universität oder Fachhochschule studieren, mittels Fragebogen zu ihrem Medien-Multitasking Verhalten befragt. Die Schulen wurden anhand der von SWITCH \cite{Switch2014} zur Verfügung gestellten Liste der Universitäten und Fachhochschulen der Schweiz ausgewählt (SWITCH ist ein Hochleistungsnetzwerk der Schweiz, welches die Verbindung von Internetnutzer weltweit und für die Vernetzung von Fachhochschulen im Bereich Schweizer Wissenschaft zuständig ist). Prinzipiell wurde versucht alle Studierenden der ausgewählten Schulen zu erreichen, die entweder im Teilzeit- oder Vollzeitmodus studieren. \\
Mit Hilfe der Poweranalyse (TBD: Referenz) wurde im Vorfeld untersucht, wie gross die angestrebte  Stichprobengrösse sein sollte. Für die Beantwortung der Hauptfragestellung ist die Stichprobengrösse von untergeordneter Rolle, da die Beantwortung mit einer Korrelation durchgeführt wurde. Grundsätzlich kann jedoch davon ausgegangen werden, dass je mehr abgeschlossene Datensätze von Probanden vorhanden sind, desto eher kann ein aussagekräftiges Resultat gefunden werden. Im Gegensatz zur primären Fragestellung ist die Stichprobengrösse für die Untersuchung der Nebenfragestellung relevant, da hier ein Signifikanzniveau angestrebt wurde. \\
Anhand der Nebenfragestellung wurde eine Stichprobengrösse von mindestens 176 Personen angestrebt (88 Teilzeitstudierende und 88 Vollzeitstudierende). Diese Stichprobengrösse wurde mittels G*Power für einen t-Test für unabhängige Stichproben mit einem mittleren Effekt von $d = 0.5$, einem Signifikanzniveau von $\alpha=0.05$ und einer Teststärke von $1-\beta=0.95$ berechnet.

%---------------------
\subsection{Rekrutierung}\label{subsection.rekrutierung}
Die Rekrutierung der Studierenden an den Universitäten und Fachhochschulen der deutschen Schweiz erfolgte wenn immer möglich direkt über die Mailverteiler der einzelnen Schulen (Liste der angeschriebenen Schulen siehe \nameref{chap.anhang_b}). Damit für diese Untersuchung eine möglichst grosse Stichprobe generiert werden konnte, wurde wenn immer möglich versucht innerhalb der einzelnen Schulen die einzelnen Departemente separat anzuschreiben. Da diese Mailverteiler der Schulen in den meisten Fällen nicht öffentlich zugänglich sind, wurde direkt auf den einzelnen Schulsekretariaten oder Departementsstellen (z.B. Dekanat) versucht die Umfrage zu platzieren. Dies wurde teilweise mittels telefonischem Kontakt und teilweise per vordefiniertem Mail vorgenommen. Das Mailing, das zu diesem Zweck erstellt wurde, ist in \nameref{chap.anhang_b} zu finden. In diesem Schreiben wurden die einzelnen Schulen um Unterstützung gebeten, in dem sie die Umfrage an ihre Studierenden weiter leiteten. Für die Weiterleitung an die Studierenden wurde dem dem Mailing eine weitere Nachricht mit dem Link zum Fragebogen angehängt. \\
Schulen die eher konservativ gegenüber Umfragebögen anderer Schulen eingestellt sind, lehnten das Weiterleiten der Umfrage an die Studierenden ab. In diesen Fällen war es jedoch oftmals möglich, einen Aushang am schwarzen Brett der Schule anbringen zu lassen. Dieser Anhang wurde auf Wunsch der entsprechenden Stelle nachgeliefert und von den Studiumsverantwortlichen selbständig angebracht (siehe Aushang \nameref{chap.anhang_b}). \\
In dieser Untersuchung wurde das Mailing an über 70 Stellen der in Kapitel '\nameref{section.auswahlVersuchsp}' genannten Schulen und deren Departemente versendet.

\subsection{Einschlusskriterien}\label{subsection.einschlusskriterien}
Alle Studierende von Universitäten und Fachhochschulen aus dem deutschsprachigen Raum, die einem Teilzeit- oder einem Vollzeitstudium nachgehen.
\subsection{Ausschlusskriterien}\label{subsection.ausschlusskriterien}
Ausgeschlossen werden Studierende eines Fernstudiums, da hier weitere Bedingungen gegeben sind, die eine Vergleichbarkeit erschwert. Es werden nur Personen berücksichtigt, die der deutschen Sprache mächtig sind. Der Fragebogen wird lediglich auf Deutsch zur Verfügung gestellt.


% Kapitel Erhebungsinstrumente
%---------------------
\section{Erhebungsinstrumente}\label{section.erhebungsinstrumente}
Die Erhebungsinstrumente der vorliegenden Arbeit basieren auf bereits existierenden Fragebögen, die in Form einer Onlineumfrage zusammengestellt wurden. In den folgenden Kapiteln werden die zur Überprüfung der Hypothesen verwendeten Fragebögen in der Reihenfolge der Durchführung beschrieben. Diese Fragebögen wurden aus dem Englischen übersetzt, deren Übersetzungwn sich im \nameref{chap.appendix_a} befinden.
%---
\subsection{Media Use Questionnaire und Media Multitasking Index}\label{subsection.muq}
Für die Bestimmung, ob es sich in der Stichprobe um Personen handelt die starkes oder schwaches Medien-Multitasking betreiben, wurde der Media Multitasking Index verwendet. Dieser Index wurde mittels Fragebogenerhebung berechnet, der die Häufigkeit der Nutzung von einzelnen Medien und die gleichzeitige Nutzung verschiedener Medien miteinander misst. Der Media Multitasking Index und der Fragebogen zur Häufigkeit von Mediennutzung basieren auf dem von \citeA{Ophir2009} entwickelten \textit{Media Multitasking Index (MMI)} und dem \textit{Media Use Questionnaire}. Diese Instrumente wurden für diese Arbeit aus dem Englischen ins Deutsche übersetzt (siehe \nameref{chap.appendix_a}). Der Fragebogen registriert wie viele Stunden eine Person eine von 12 vordefinierten Medienformen in einer Woche nutzt und wie oft diese Person diese Medienform mit anderen Medienformen gleichzeitig verwendet. \\ 
In dieser Umfrage wurde anstelle von Stunden in einer Woche, die durchschnittliche Anzahl Minuten an einem Tag erfasst. Dies erschien für Medien wie Textnachrichten einfacher abschätzbar zu sein. Durch diese Anpassung mussten anschliessend die erfassten Minuten in Stunden einer Woche hochgerechnet werden. Dies wurde mittels folgender Formel berechnet: \(\frac{m_{i} \times 7_{Tage}}{60}\), wobei \(m_{i}\) die Anzahl Minuten pro Tag sind, die für die Nutzung eines Mediums verwendet wurden. Die 12 erfassten Medienformen sind Druckmedien, Fernsehen, Online Video (wie zum Beispiel Youtube), Musik, Nicht-musikalische Audiomedien (z.B. Hörbücher, Podcast, etc.), Video oder Computer Games, Telefonieren (Mobil- und / oder Festnetz), Instant Messaging (z.B. Skype, Windows Live Messenger, Yahoo Messenger, etc.), SMS (Textnachrichten), Email, Internet-Surfen und Andere Computerbasierte-Tätigkeiten (z.B. Word, Videobearbeitung, programmieren, etc.). \\
Neben der Nutzungsdauer einzelner Medien wurde die gleichzeitige Nutzung verschiedener Medien mittels Medien-Multitasking-Matrix gebildet. Diese besteht im Wesentlichen aus den 12 definierten Medien, die in der vertikalen Spalte als Hauptmedien und in der horizontalen Zeile als Nebenmedien aufgelistet werden. Die Aufgabe für die Probanden bestand darin, zu jedem Hauptmedium eine Schätzung abzugeben, wie oft ein allfällig weiteres Nebenmedium gleichzeitig verwendet wird. Die Bewertung in der Matrize erfolgte mittels definierter Auswahl \textit{meistens}, \textit{etwas}, \textit{wenig} oder \textit{nie}.\\
Mittels der Nutzungsdauer und der gleichzeitigen Nutzung einzelner Medien wird der Medien Multitasking Index gebildet    . Die Formel für die Berechnung des Indexes lautet:
%Formel
\begin{equation}\label{formula.mmi}
    MMI=\sum_{i=1}^{11} \frac{m_{i} \times h_{i}}{h_{total}}
\end{equation}
In dieser Formel ist \(h_{i}\) die Anzahl Stunden pro Woche, die ein Hauptmedium \(i\) verwendet wurde. \(h_{total}\) steht für die totale Anzahl Stunden aller Medien in einer Woche. \(m_i\) beinhaltet die Abschätzung des Probanden bezüglich seiner Nutzung eines Nebenmediums, während dem er ein Hauptmedium nutzt. Den Abschätzungen des Probanden wurden numerische Werte wie folgt zugewiesen: 1 = \textit{meistens}, 0.67 = \textit{etwas}, 0.33 = \textit{wenig} oder 0 = \textit{nie}. \\
Der resultierende Medien Multitasking Index \(MMI\) kennzeichnet die durchschnittliche Menge von Medien Multitasking, die während einer typischen Stunde Mediennutzung auftritt. Die Unterscheidung, ob es sich bei der untersuchten Person um eine Person handelt die starkes oder schwaches Medien-Multitasking betreibt, wurde die Standardabweichung verwendet (eine Standardabweichung oder mehr über dem Mittel für starkes Betreiben von Multitasking und eine Standardabweichung oder weniger unter dem Mittel für schwaches Betreiben von Multitasking). 

%---  
\subsection{Attentional Control Scale} \label{subsection.acs}
Bei der Aufmerksamkeitskontroll-Skale (engl.: Attentional Control Scale - ACS) handelt es sich um einen Selbstbeurteilungsfragebogen, der von \citeA{Derryberry2002} für die Messung von individuellen Unterschieden in der Aufmerksamkeitskontrolle entwickelt wurde. Aufmerksamkeit entsteht in miteinenader verbundenen Netzwerken im Gehirn, eine davon ist das anteriore Aufmerksamkeitssystem (engl.: anterior attentional system), welches als exekutive Kontrollfunktion über die restlichen Aufmerksamkeitsprozesse waltet \cite{Posner1998}. Aufgrund unterschiedlich vorgeschlagener Funktionen des anterioren Systems, wurde die Aufmerksamkeitskontroll-Skale als Instrument für die Messung allgemeiner Unterschiede in der selbstinitierten Aufmerksamkeitskontrolle entwickelt \cite{Derryberry2001}.\\
Der Fragebogen umfasst 20 Fragen, die Ursprünglich in zwei unterschiedlichen Skalen enthalten waren \cite{Derryberry1988}: Ausrichten der Aufmerksamkeit und Aufmerksamkeitsverschiebung. Unter der Ausrichtung der Aufmerksamkeit verstanden die Forscher die Kapazität, den Aufmerksamkeitsfokus absichtlich in ein gewünschte Richtung zu lenken, ohne sich dabei unabsichtlich von irrelevanten oder störenden Reizen ablenken zu lassen. Unter Aufmerksamkeitsverschiebung verstanden die Forscher die Kapazität, die Aufmerksamkeit absichtlich von einer Richtung in die andere Richtung zu lenken und dabei die Aufmerksamkeit nicht in eine spezifische Richtung zu fokussieren. In den letzten Jahren wurden die zwei Skalen unter dem Sammelbegriff der Aufmerksamkeitskontroll-Skala zusammengefasst, die als Messmittel für die Fähigkeit der Aufmerksamkeitskontrolle von Personen dient. Gemäss \citeA{Derryberry2002} ergab die Faktoranalyse des ACS Fragebogens einen Zusammenhang aus wechselseitigen Subfaktoren bezogen auf die Fähigkeiten wie (a) sich auf etwas zu fokussieren, (b) zwischen verschiedenen Aufgaben hin und her zu wechseln und (c) eine flexible Gedankenkontrolle \cite[S.~226]{Derryberry2002}. \\
Der ACS wurde für diese Untersuchung aus dem Englischen übersetzt (siehe \nameref{chap.appendix_a}). Die Skala des ACS besteht aus 20 Fragen, die mittels vierstufiger Likert-Skala beantwortet werden konnten (1 = fast nie; 2 = manchmal; 3 = oft; 4 = immer). Je höher die Antwortzahl ausfällt, desto höher ist die Aufmerksamkeitskontrolle.

%---
\subsection{Flourishing Scale und Scale of Positive and Negative Experience} \label{subsection.flourishingScale}
Die von \citeA{Diener:2010} entwickelten Methoden, \textit{The Flourishing Scale (FS)} und \textit{The Scale of Positive and Negative Epxerience (SPANE)} sind Fragebögen für die Erfassung des psychologischen und sozialen Wohlbefindens. Mit diesen Skalen soll das psychosoziale Aufblühen (aus dem Englischen von flourishing), basierend auf dem psychologischen und sozialem Wohlbefinden und dem positiven und negativen Befinden, erfasst werden. \\
Die Skala über das menschliche Aufblühen (FS) wurde auf dem Hintergrund von sozialen Beziehungen entwickelt. Der Fragebogen erfasst menschliches Aufblühen in relevanten Bereichen wie Lebensinhalt, Beziehungen, Selbstwertgefühl, Gefühl der Kompetenz und Optimismus \cite{Silva2013}. Dieser Bereich von positivem Funktionieren scheint einen signifikanten Einfluss auf das persönliche Wohlbefinden zu haben \cite<e.g.,>{Ryan2000, Ryff1989}. Die Skala umfasst acht Items aus wichtigen Bereichen des menschlichen Funktionierens, wie positive Beziehungen, Gefühle der Kompetenz und Führen eines zielgerichteten und sinnvollen Lebens \cite{Diener:2010}. Die aus dem Englisch stammenden Fragen wurden für diese Arbeit ins Deutsche übersetzt (siehe dazu \nameref{chap.appendix_a}). Jedes Item konnte auf einer Skala von 1 - 7 beantwortet werden. Von 1 = trifft überhaupt nicht zu bis zu 7 = triff voll und ganz zu. Alle Fragen wurden in einer positiven Richtung formuliert. Die Skala verfügt über einen einzigen Ergebniswert für das psychologische Wohlbefinden. Die Summe der Punkte kann zwischen 8 (trifft überhaupt nicht zu) und 56 (trifft voll und ganz zu) liegen. Hohe Punktezahlen deuten darauf hin, dass sich die befragte Person in einem positiven Licht sieht, bezogen auf wichtige Bereiche in ihrem Leben. Die Skala zeigt jedoch keine einzelnen Aspekte des Wohlbefindens auf. Vielmehr erstellt sie eine Übersicht aus positivem Funktionieren in weithin als wichtig angeschauten Bereichen aus dem Leben heraus. \\ Die Skala für das Erfassung von positiven und negativen Erfahrungen (SPANE) beinhaltet einen zwölfteiligen Fragebogen, bestehend aus je sechs Items für die Erfassung von positiven und sechs Items für die Erfassung von negativen Empfindungen \cite{Diener:2010}. Für beide Skalen, positive und negative Empfindungen, sind drei Items genereller Natur (z.B.: positiv, negativ) und drei spezifischer Natur (z.B.: froh, traurig). Die Skala erfasst die gesamte Bandbreite von positiven und negativen Erfahrungen aufgrund der allgemein und spezifisch gehaltenen Empfindungen. Dadurch wird angenommen, dass der Fragebogen kulturunabhängig funktioniert \cite{Silva2013}. Jede Frage des SPANE Fragebogens wird auf einer fünfstufigen Likert Skala von 1 bis 5, von 1 = sehr selten oder nie bis 5 = sehr oft oder immer, beantwortet. Die positive und die negative Skala werden aufgrund der teilweise unabhängigen Typen der Empfindungen separat ausgewertet. Die Summe der positiven Resultate (SPANE-P) sowie der negativen Resultate (SPANE-N) können zwischen einem Bereich von 6 und 30 liegen. Diese beiden Skalen können miteinander kombiniert werden, indem die negative Skala von der positiven Skala abgezogen wird. Die Summe dieser neuen Skala (SPANE-B) kann zwischen -24 und 24 liegen.




% Kapitel Datenerhebung
%---------------------
\section{Datenerhebung}\label{section.datenerhebung}
TBD: 
\begin{itemize}
    \item Wie die Daten erhoben wurden - unipark, online 
    \item In welcher Zeit. 
    \item Verweis auf Anhang des Fragebogens
\end{itemize}


% Kapitel Datenaufbereitung
%---------------------
\section{Datenaufbereitung}\label{section.datenaufbereitung}
\begin{itemize}
    \item SPSS
    \item Datenbereinigung
    \item Dropouts
    \item wie werden die daten aufbereitet
\end{itemize}


% Kapitel Statistische Verfahren
%---------------------
\section{Statistische Verfahren}\label{section.statistischeVerfahren}
tbd:






