%%%%%%%%%%%%%%%%%%%%%%%%%%%%%%%%%%%%%%%%%%%%%%%%%%%%%%%%%%%%%%%%%
%_____________ ___    _____  __      __ 
%\____    /   |   \  /  _  \/  \    /  \  Institute of Applied
%  /     /    ~    \/  /_\  \   \/\/   /  Psychology
% /     /\    Y    /    |    \        /   Zuercher Hochschule 
%/_______ \___|_  /\____|__  /\__/\  /    fuer Angewandte Wissen.
%        \/     \/         \/      \/                           
%%%%%%%%%%%%%%%%%%%%%%%%%%%%%%%%%%%%%%%%%%%%%%%%%%%%%%%%%%%%%%%%%
%
% Project     : Bachelorarbeit
% Title       : Methode
% File        : methode.tex Rev. 00
% Date        : 06.12.2013
% Author      : Till J. Ernst
%
%%%%%%%%%%%%%%%%%%%%%%%%%%%%%%%%%%%%%%%%%%%%%%%%%%%%%%%%%%%%%%%%%
\glsresetall
\let\raggedsection\centering 
\mychapter{2}{Methode}
%\chapter*{Methode}\label{chap.methode}
%\addcontentsline{toc}{chapter}{Methode}
\let\raggedsection\raggedright 

% Kapitel Studiendesign
%---------------------
\section{Studiendesign}\label{section.studiendesign}
\subsection{Hauptzielparameter}\label{subsection.hauptzielparameter}
\subsection{Durchführung der Untersuchung}\label{subsection.durchführung}

% Kapitel Auswahl der Versuchspersonen
%---------------------
\section{Auswahl der Versuchspersonen}\label{section.auswahlVersuchsp}
\subsection{Rekrutierung}\label{subsection.rekrutierung}
\subsection{Einschlusskriterien}\label{subsection.einschlusskriterien}
\subsection{Ausschlusskriterien}\label{subsection.ausschlusskriterien}

% Kapitel Erhebungsinstrumente
%---------------------
\section{Erhebungsinstrumente}\label{section.erhebungsinstrumente}
\subsection{Media Use Questionnaire und Media Multitasking Index}\label{subsection.muq}
Die Messung von Medien Multitasking wurde mittels Medien Multitasking Index gebildet, der basierend auf einem Fragebogen zur Häufigkeit von Mediennutzung und dem gleichzeitigen Nutzen von verschiedenen Medien erhoben wurde. Der Media Multitasking Index und der Fragebogen zur Häufigkeit von Mediennutzung basiert auf dem von \citeA{Ophir2009} entwickelten \textit{Media Multitasking Index} und dem \textit{Media Use Questionnaire}. Dieses Instrument wurde für diese Arbeit aus dem Englischen ins Deutsche übersetzt (siehe \nameref{chap.appendix_a}). Der Fragebogen registriert wie viele Stunden eine Person eine von 12 Medienformen in einer Woche nutzt und wie oft sie diese Medienform mit anderen Medienformen gleichzeitig verwendet. Für diese Umfrage wurde anstelle der Anzahl Stunden in einer Woche, die durchschnittliche Anzahl Minuten an einem Tag erfasst. Die Erfassung von Minuten pro Tag anstelle von Stunden erschien für Medien wie Textnachrichten einfacher abzuschätzen. Für die abschliessende Berechnung wurde die Dauer in Minuten auf Stunden einer Woche hochgerechnet \(\frac{m_{i} \times 7_{Tage}}{60}\), wobei \(m_{i}\) die Anzahl Minuten pro Tag sind. Die 12 erfassten Medienformen sind Druckmedien, Fernsehen, Online Video (wie zum Beispiel Youtube), Musik, Nicht-musikalische Audiomedien (z.B. Hörbücher, Podcast, etc.), Video oder Computer Games, Telefonieren (Mobil- und / oder Festnetz), Instant Messaging (z.B. Skype, Windows Live Messenger, Yahoo Messenger, etc.), SMS (Textnachrichten), Email, Internet-Surfen und Andere Computerbasierte-Tätigkeiten (z.B. Word, Videobearbeitung, programmieren, etc.). Neben der Mediennutzungsdauer wurde der Media Multitasking Index mittels Medien-Multitasking-Matrix gebildet. Diese erfasst die gleichzeitige Nutzung von einem Hauptmedium mit einem weiteren Medium. Jedes Medium das gleichzeitig genutzt wird, konnte in der Matrize mittels Auswahl \textit{meistens}, \textit{etwas}, \textit{wenig} oder \textit{nie} kategorisiert werden.\\
  
\subsection{Attentional Control Scale}\label{subsection.acs}
\subsection{Flourishing Scale und Scale of Positive and Negative Experience}\label{subsection.flourishingScale}

% Kapitel Datenerhebung
%---------------------
\section{Datenerhebung}\label{section.datenerhebung}

% Kapitel Datenaufbereitung
%---------------------
\section{Datenaufbereitung}\label{section.datenaufbereitung}

% Kapitel Statistische Verfahren
%---------------------
\section{Statistische Verfahren}\label{section.statistischeVerfahren}





