%%%%%%%%%%%%%%%%%%%%%%%%%%%%%%%%%%%%%%%%%%%%%%%%%%%%%%%%%%%%%%%%%
%_____________ ___    _____  __      __ 
%\____    /   |   \  /  _  \/  \    /  \  Institute of Applied
%  /     /    ~    \/  /_\  \   \/\/   /  Psychology
% /     /\    Y    /    |    \        /   Zuercher Hochschule 
%/_______ \___|_  /\____|__  /\__/\  /    fuer Angewandte Wissen.
%        \/     \/         \/      \/                           
%%%%%%%%%%%%%%%%%%%%%%%%%%%%%%%%%%%%%%%%%%%%%%%%%%%%%%%%%%%%%%%%%
%
% Project     : Bachelorarbeit
% Title       : Methode
% File        : methode.tex Rev. 00
% Date        : 06.12.2013
% Author      : Till J. Ernst
%
%%%%%%%%%%%%%%%%%%%%%%%%%%%%%%%%%%%%%%%%%%%%%%%%%%%%%%%%%%%%%%%%%
\glsresetall
\let\raggedsection\centering 
\mychapter{2}{Methode}
%\chapter*{Methode}\label{chap.methode}
%\addcontentsline{toc}{chapter}{Methode}
\let\raggedsection\raggedright 

% Kapitel Studiendesign
%---------------------
\section{Studiendesign}\label{section.studiendesign}
\subsection{Hauptzielparameter}\label{subsection.hauptzielparameter}
\subsection{Durchführung der Untersuchung}\label{subsection.durchführung}

% Kapitel Auswahl der Versuchspersonen
%---------------------
\section{Auswahl der Versuchspersonen}\label{section.auswahlVersuchsp}
\subsection{Rekrutierung}\label{subsection.rekrutierung}
\subsection{Einschlusskriterien}\label{subsection.einschlusskriterien}
\subsection{Ausschlusskriterien}\label{subsection.ausschlusskriterien}

% Kapitel Erhebungsinstrumente
%---------------------
\section{Erhebungsinstrumente}\label{section.erhebungsinstrumente}
\subsection{Media Use Questionnaire und Media Multitasking Index}\label{subsection.muq}
Für die Messung des Medien Multitasking wurde der Media Multitasking Index gebildet. Dieser Index wurde mittels Fragebogen berechnet, der die Häufigkeit der Nutzung von einzelnen Medien und die gleichzeitige Nutzung verschiedener Medien miteinander misst. Der Media Multitasking Index und der Fragebogen zur Häufigkeit von Mediennutzung basieren auf dem von \citeA{Ophir2009} entwickelten \textit{Media Multitasking Index (MMI)} und dem \textit{Media Use Questionnaire}. Diese Instrumente wurden für diese Arbeit aus dem Englischen ins Deutsche übersetzt (siehe \nameref{chap.appendix_a}). Der Fragebogen registriert wie viele Stunden eine Person eine von 12 vordefinierten Medienformen in einer Woche nutzt und wie oft diese Person diese Medienform mit anderen Medienformen gleichzeitig verwendet. \\ 
In dieser Umfrage wurde anstelle von Stunden in einer Woche, die durchschnittliche Anzahl Minuten an einem Tag erfasst. Dies erschien für Medien wie Textnachrichten einfacher abschätzbar zu sein. Durch diese Anpassung mussten anschliessend die erfassten Minuten in Stunden einer Woche hochgerechnet werden. Dies wurde mittels folgender Formel berechnet: \(\frac{m_{i} \times 7_{Tage}}{60}\), wobei \(m_{i}\) die Anzahl Minuten pro Tag sind, die für die Nutzung eines Mediums verwendet wurden. Die 12 erfassten Medienformen sind Druckmedien, Fernsehen, Online Video (wie zum Beispiel Youtube), Musik, Nicht-musikalische Audiomedien (z.B. Hörbücher, Podcast, etc.), Video oder Computer Games, Telefonieren (Mobil- und / oder Festnetz), Instant Messaging (z.B. Skype, Windows Live Messenger, Yahoo Messenger, etc.), SMS (Textnachrichten), Email, Internet-Surfen und Andere Computerbasierte-Tätigkeiten (z.B. Word, Videobearbeitung, programmieren, etc.). \\
Neben der Nutzungsdauer einzelner Medien wurde die gleichzeitige Nutzung verschiedener Medien mittels Medien-Multitasking-Matrix gebildet. Diese besteht im Wesentlichen aus den 12 definierten Medien, die in der vertikalen Spalte als Hauptmedien und in der horizontalen Zeile als Nebenmedien aufgelistet werden. Die Aufgabe für die Probanden bestand darin, zu jedem Hauptmedium eine Schätzung abzugeben, wie oft ein allfällig weiteres Nebenmedium gleichzeitig verwendet wird. Die Bewertung in der Matrize erfolgte mittels definierter Auswahl \textit{meistens}, \textit{etwas}, \textit{wenig} oder \textit{nie}.\\
Mittels der Nutzungsdauer und der gleichzeitigen Nutzung einzelner Medien wird der Medien Multitasking Index gebildet    . Die Formel für die Berechnung des Indexes lautet:
%Formel
\begin{equation}\label{formula.mmi}
    MMI=\sum_{i=1}^{11} \frac{m_{i} \times h_{i}}{h_{total}}
\end{equation}
In dieser Formel ist \(h_{i}\) die Anzahl Stunden pro Woche, die ein Hauptmedium \(i\) verwendet wurde. \(h_{total}\) steht für die totale Anzahl Stunden aller Medien in einer Woche. \(m_i\) beinhaltet die Abschätzung des Probanden bezüglich seiner Nutzung eines Nebenmediums, während dem er ein Hauptmedium nutzt. Den Abschätzungen des Probanden wurden numerische Werte wie folgt zugewiesen: 1 = \textit{meistens}, 0.67 = \textit{etwas}, 0.33 = \textit{wenig} oder 0 = \textit{nie}. Der resultierende Medien Multitasking Index \(MMI\) kennzeichnet die durchschnittliche Menge von Medien Multitasking, die während einer typischen Stunde Mediennutzung auftritt.

  
\subsection{Attentional Control Scale}\label{subsection.acs}
\subsection{Flourishing Scale und Scale of Positive and Negative Experience}\label{subsection.flourishingScale}

% Kapitel Datenerhebung
%---------------------
\section{Datenerhebung}\label{section.datenerhebung}

% Kapitel Datenaufbereitung
%---------------------
\section{Datenaufbereitung}\label{section.datenaufbereitung}

% Kapitel Statistische Verfahren
%---------------------
\section{Statistische Verfahren}\label{section.statistischeVerfahren}





