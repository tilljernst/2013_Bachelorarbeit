%%%%%%%%%%%%%%%%%%%%%%%%%%%%%%%%%%%%%%%%%%%%%%%%%%%%%%%%%%%%%%%%%
%_____________ ___    _____  __      __ 
%\____    /   |   \  /  _  \/  \    /  \  Institute of Applied
%  /     /    ~    \/  /_\  \   \/\/   /  Psychology
% /     /\    Y    /    |    \        /   Zuercher Hochschule 
%/_______ \___|_  /\____|__  /\__/\  /    fuer Angewandte Wissen.
%        \/     \/         \/      \/                           
%%%%%%%%%%%%%%%%%%%%%%%%%%%%%%%%%%%%%%%%%%%%%%%%%%%%%%%%%%%%%%%%%
%
% Project     : Bachelorarbeit
% Title       : 
% File        : Rev. 00
% Date        : 06.12.2013
% Author      : Till J. Ernst
%
%%%%%%%%%%%%%%%%%%%%%%%%%%%%%%%%%%%%%%%%%%%%%%%%%%%%%%%%%%%%%%%%%
\glsresetall

\let\raggedsection\centering 
\chapter{Fragebogen}\label{chap.appendix_fragebogen}
\let\raggedsection\raggedright 
\begin{RaggedRight}
TBD: Einleitung
Durch die Übernahme des Umfragebogens aus dem Internet, wurde dieser für die Darstellung in diesem Dokument im Folgenden leicht angepasst. Dies betrifft insbesondere die Eingabefelder und Auswahllisten, die im Onlinefragebogen zur Verfügung gestanden haben. Diese Elemente werden bei den einzelnen Fragen innerhalb von Klammern angedeutet.
%Startseite
\section{Startseite}\label{anhangSesction.Startseite}
\subsection*{Herzlich Willkommen}
Vielen Dank, dass Du Dir Zeit nimmst, den folgenden Fragebogen zu beantworten. Das Ausfüllen des Fragebogens wird ca. 10 bis 15 Minuten in Anspruch nehmen.\par
Bei den Fragen kommt es mir auf deine subjektiven Einschätzungen an, d.h. es gibt keine richtigen oder falschen Antworten. Mich interessiert deine persönliche Meinung. Bitte lies jeweils genau die Instruktionen zu den einzelnen Fragen und beantworte alle Fragen zügig und vertraue dabei deinem spontanen Urteil. Wenn dennoch eine Aussage für Dich schwierig einzuschätzen ist, versuche diese bitte trotzdem zu beantworten. \par
Sämtliche Angaben werden streng vertraulich behandelt. Daten werden nicht an dritte Personen weiter gegeben. \par
Besten Dank für deine Mithilfe.\par
Till J. Ernst\\
Mailto: ernsttil(at)students.zhaw.ch
\subsection*{Ergebnisse}
Die Ergebnisse der Studie werden anonymisiert, ausgewertet und in einer Bachelorarbeit publiziert. Rückschlüsse auf einzelne Teilnehmende der Befragung sind nicht möglich.\par
Falls Du an den Ergebnissen interessiert bist, hast Du am Schluss die Möglichkeit deine Mailadresse zu hinterlegen, damit ich Dir die Ergebnisse persönlich zusenden kann. \par
Die Ergebnisse werden Ende Frühlingssemester 2014 in Form einer Bachelorarbeit im Netz publiziert (\url{http://www.zhaw.ch/de/zhaw/die­zhaw/publikationen.html}).
\subsection*{Wettbewerb}
Am Ende von diesem Fragebogen hast Du die Möglichkeit an einem Wettbewerb teilzunehmen. Du kannst dabei Gutscheine beim Athleticum Sportmarkets gewinnen. \par
Um am Wettbewerb teilzunehmen wirst Du am Schluss gebeten deine Mailadresse zu hinterlegen. Diese Mailadresse dient nur zur Teilnahme am Wettbewerb. Ein Rückschluss auf die Antworten ist damit nicht möglich. Über den Wettbewerb wird keine Korrespondenz geführt.
\begin{figure}[h]
     \centering
     \includegraphics[scale=0.6]{images/anhang/sponsoring_athleticum_mittel.jpg}
\end{figure}
%Demographische Daten
\section{Demographische Daten}\label{anhangSesction.demograpData}
\subsection*{Angaben zu deiner Person}
Gleich zu Beginn möchte ich Dich um einige Angaben zu deiner Person bitten.
\subsubsection*{Geschlecht (Auswahl)}
    \begin{itemize}
      \item weiblich
      \item männlich
      \item keine Angabe
    \end{itemize}
\end{RaggedRight}
\subsubsection*{Wie alt bis Du? (Eingabefeld)}
Bitte das aktuelle Altersjahr in Zahlen angeben.
\subsubsection*{Wo studierst Du? (Auswahl)}
    \begin{itemize}
      \item Universität
      \item Fachhochschule
    \end{itemize}
\subsubsection*{Studienrichtung (Eingabefeld)}
    \begin{itemize}
      \item Hauptfach
      \item Nebenfach (optional)
    \end{itemize}
\subsubsection*{Studentenstatus (Auswahl)}
    \begin{itemize}
      \item Vollzeit
      \item Teilzeit
    \end{itemize}
\subsubsection*{Gehst Du neben dem Studium einer bezahlten Arbeit nach? (Auswahl und Eingabefeld)}
Falls Du berufstätig bist, fülle bitte die durchschnittliche Anzahl Stunden ein, die Du in einer Woche arbeitest.
    \begin{itemize}
      \item Ja (inkl. Eingabefeld)
      \item Nein
    \end{itemize}
\subsubsection*{Zivilstand (Auswahl)}
    \begin{itemize}
      \item ledig
      \item verheiratet
      \item getrennt
      \item geschieden
      \item verwitwet
    \end{itemize}
\subsubsection*{Hast Du Kinder? (Auswahl)} 
    \begin{itemize}
      \item Ja
      \item Nein
    \end{itemize}    
%Medie Use Questionnaire
\section{MUQ ­-- Media Use Questtionnaire}\label{anhangSection.muq}   
\subsection*{Mediennutzung}    
In den folgenden Fragen möchte ich gerne von Dir wissen, welche Medien Du wie oft einsetzt und welche Du gleichzeitig verwendest.
\subsubsection*{Wieviele Minuten pro Tag nutzt Du die folgenden 12 Medienformen durchschnittlich (privat, Studium und geschäftlich)? (Eingabefeld)}
Bitte fülle die Anzahl Minuten in ganzen Zahlen ein (Beispiel: Druckmedien = 60; Fernsehen = 45).
    \begin{itemize}
      \item Druckmedien (z.B. Bücher, Zeitungen, Zeitschriften, etc.)
      \item Fernsehen
      \item Online Video (wie zum Beispiel Youtube)
      \item Musik
      \item Nicht­musikalische Audiomedien (wie Hörbücher, Podcast, etc.)
      \item Video oder Computer Games
      \item Telefonieren (Mobil­ und / oder Festnetz)
      \item Instant Messaging (wie zum Beispiel Skype, Windows Live Messenger, Yahoo Messenger, etc.)
      \item SMS (Textnachrichten)
      \item Email
      \item Internet­Surfen
      \item Andere computerbasierte Tätigkeiten (z.B. Textverarbeitung, Videobearbeitung, programmieren, etc.)
    \end{itemize} 
\subsubsection*{Bitte trage ein, wie oft Du ein Medium gleichzeitig mit einem anderen verwendest. Ein Beispiel könnte sein, wenn Du ein Buch liest und gleichzeitig Musik hörst (Auswahl mittels Matrix)}
Auf der linken Spalte befindet sich das Medium, dem Du Dich hauptsächlich widmest. Auf der horizontalen Zeile befinden sich die Medien, die Du zusammen mit dem Medium auf der linken Seite verwendest.\par
Verwende für deine Zuteilung die Wertung mittels Dropdown--­Liste (meistens, etwas, wenig, nie). Machst Du keine Angaben zu einem Nebenmedium gilt dies als 'nie' (Standard).\par
\textbf{Probleme mit der Ansicht:} Falls die Darstellung einer Liste gleicht, kann dies daran liegen, dass das Fenster deines Browsers zu klein ist. Bitte vergrössere das Fenster, damit Du alle Medien auf einer Zeile siehst oder falls dies nicht geht, gehe bitte der Liste nach vor. Die fett markierten Medien sind die Hauptmedien, die folgenden normal gedruckten Medien sind diejenigen, die Du gleichzeitig nutzt.






    
    