%%%%%%%%%%%%%%%%%%%%%%%%%%%%%%%%%%%%%%%%%%%%%%%%%%%%%%%%%%%%%%%%%
%_____________ ___    _____  __      __ 
%\____    /   |   \  /  _  \/  \    /  \  Institute of Applied
%  /     /    ~    \/  /_\  \   \/\/   /  Psychology
% /     /\    Y    /    |    \        /   Zuercher Hochschule 
%/_______ \___|_  /\____|__  /\__/\  /    fuer Angewandte Wissen.
%        \/     \/         \/      \/                           
%%%%%%%%%%%%%%%%%%%%%%%%%%%%%%%%%%%%%%%%%%%%%%%%%%%%%%%%%%%%%%%%%
%
% Project     : Bachelorarbeit
% Title       : 
% File        : Rev. 00
% Date        : 06.12.2013
% Author      : Till J. Ernst
%
%%%%%%%%%%%%%%%%%%%%%%%%%%%%%%%%%%%%%%%%%%%%%%%%%%%%%%%%%%%%%%%%%
\glsresetall

\let\raggedsection\centering 
\chapter{Korrelationstabellen}\label{anhang.korrelationsTabellen}
\let\raggedsection\raggedright 
\begin{RaggedRight}
Für eine verbesserte Übersicht, wurden die beiden Variablen $MMI_{ext}$ und $MMI$ wo immer sinnvoll zusammengefasst.
%MMI zu SWB anhand Geschlecht
\section{Korrelation zwischen MMI und SWB -- Aufteilung anhand Geschlecht}\label{anhangKorrelationen.geschlecht}
%Table Geschlecht MMI SWB
\begin{table}[H] 
    \centering
    \caption{Zusammenhang zwischen dem Medien-Multitasking und dem subjektivem Wohlbefinden, Korrelationen aufgeteilt anhand dem Geschlecht}
    \begin{tabular}[t]{|L{5mm} R{15mm}|R{25mm}|R{16mm}|R{15mm}|R{14mm}|R{16mm}|R{17mm}|} 
        \hline
        \multicolumn{8}{|c|}{\textbf{Korrelationen aufgeteilt nach dem Geschlecht}}\\ 
        \hline       
        \multicolumn{3}{|c|}{} & \multicolumn{1}{c|}{$MMI_{ext}$} &\multicolumn{1}{c|}{$FS$} & \multicolumn{3}{c|}{$SPANE$}\\
        \multicolumn{3}{|l|}{Geschlecht} & \multicolumn{1}{c|}{($MMI$)} & \multicolumn{1}{c|}{} &\multicolumn{1}{c|}{$P$} &  \multicolumn{1}{c|}{$N$} & \multicolumn{1}{c|}{$B$}\\
        \hline
        \multicolumn{2}{|l|}{weiblich} & & & & & &\\
        & $MMI_{ext}$ ($MMI$) & Korrelation nach Pearson & 1 \newline (1) & -.064* (-.068*) & -.046 (-.054) & .126** (.119**) & -.101** (-.101**) \\
        & & Signifikanz (2-seitig) & & .050 (.036) & .152 (.095) & .000 (.000) & .002 (.002)\\
        & & $N$ & 955 & 953 & 951 & 951 & 951\\
        \hline
        \multicolumn{2}{|l|}{männlich} & & & & & &\\
        & $MMI_{ext}$ ($MMI$) & Korrelation nach Pearson & 1 \newline (1) & .046 (.040) & .020 (.017) & .060 (.059) & -.023 (-.024)\\
        & & Signifikanz (2-seitig) & & .358 (.423) & .689 (.733) & .233 (.243) & .645 (.632)\\
        & & $N$ & 401 & 400 & 399 & 399 & 399\\
        \hline
        \multicolumn{8}{l}{**. Die Korrelation ist auf dem Niveau von 0.01 (2-seitig) signifikant.}\\
        \multicolumn{8}{l}{*. Die Korrelation ist auf dem Niveau von 0.05 (2-seitig) signifikant.}\\
    \end{tabular}
    \label{table.ergebnis.geschlecht}
\end{table}

%MMI zu SWB anhand Alter (Mediansplit)
\section{Korrelation zwischen MMI und SWB -- Aufteilung anhand Alter (Mediansplit)}\label{anhangKorrelationen.alter}
%Table Alter Mediansplit MMI SWB
\begin{table}[H] 
    \centering
    \caption{Zusammenhang zwischen dem Medien-Multitasking und dem subjektivem Wohlbefinden, Korrelationen aufgeteilt anhand dem Alter (Mediansplit)}
    \begin{tabular}[t]{|L{5mm} R{15mm}|R{25mm}|R{16mm}|R{15mm}|R{15mm}|R{16mm}|R{15mm}|} 
        \hline
        \multicolumn{8}{|c|}{\textbf{Korrelationen aufgeteilt nach dem Alter (Mediansplit)}}\\ 
        \hline       
        \multicolumn{3}{|c|}{} & \multicolumn{1}{c|}{$MMI_{ext}$} &\multicolumn{1}{c|}{$FS$} & \multicolumn{3}{c|}{$SPANE$}\\
        \multicolumn{3}{|l|}{Alter (Median = 24)} & \multicolumn{1}{c|}{($MMI$)} & \multicolumn{1}{c|}{} &\multicolumn{1}{c|}{$P$} &  \multicolumn{1}{c|}{$N$} & \multicolumn{1}{c|}{$B$}\\
        \hline
        \multicolumn{2}{|l|}{< 24} & & & & & &\\
        & $MMI_{ext}$ ($MMI$) & Korrelation nach Pearson & 1 \newline (1) & .009 (-.005) & .030 (.021) & .134** (.127**) & -.064 (-.065) \\
        & & Signifikanz (2-seitig) & & .817 (.896) & .438 (.589) & .001 (.001) & .099 (.097)\\
        & & $N$ & 662 & 661 & 661 & 661 & 661\\
        \hline
        \multicolumn{2}{|l|}{>= 24} & & & & & &\\
        & $MMI_{ext}$ ($MMI$) & Korrelation nach Pearson & 1 \newline (1) & -.039 (-.039) & -.061 (-.064) & .099** (.092*) & -.092* (-.090*)\\
        & & Signifikanz (2-seitig) & & .300 (.308) & .109 (.093) & .009 (.016) & .015 (.019)\\
        & & $N$ & 695 & 693 & 690 & 690 & 690\\
        \hline
        \multicolumn{8}{l}{**. Die Korrelation ist auf dem Niveau von 0.01 (2-seitig) signifikant.}\\
        \multicolumn{8}{l}{*. Die Korrelation ist auf dem Niveau von 0.05 (2-seitig) signifikant.}\\
    \end{tabular}
    \label{table.ergebnis.alter}
\end{table}

%MMI zu SWB anhand Alter (Mediansplit)
\section{Korrelation zwischen MMI und SWB -- Aufteilung anhand Mediennutzung}\label{anhangKorrelationen.medienNutzung}
%Table Mediennutzung MMI SWB MMIangepasst
\begin{table}[H] 
    \centering
    \caption{Zusammenhang zwischen dem Medien-Multitasking und dem subjektivem Wohlbefinden, Korrelationen aufgeteilt anhand der Mediennutzung}
    \begin{tabular}[t]{|L{10mm} R{15mm}|R{25mm}|R{16mm}|R{15mm}|R{15mm}|R{16mm}|R{15mm}|} 
        \hline
        \multicolumn{8}{|c|}{\textbf{Korrelationen aufgeteilt nach der Mediennutzung}}\\ 
        \hline       
        \multicolumn{3}{|c|}{} & \multicolumn{1}{c|}{$MMI_{ext}$} &\multicolumn{1}{c|}{$FS$} & \multicolumn{3}{c|}{$SPANE$}\\
        \multicolumn{3}{|l|}{Mediennutzung} & \multicolumn{1}{c|}{($MMI$)} & \multicolumn{1}{c|}{} &\multicolumn{1}{c|}{$P$} &  \multicolumn{1}{c|}{$N$} & \multicolumn{1}{c|}{$B$}\\
        \hline
        %\multicolumn{2}{|l|}{LMMs} & & & & & &\\
        LMMs & $MMI_{ext}$ ($MMI$) & Korrelation nach Pearson & 1 \newline (1) & -.021 (.022) & .080 (.153*) & .002 (-.040) & .042 (.106)\\
        & & Signifikanz (2-seitig) & & .778 (.765) & .276 (.040) & .976 (.593) & .569 (.158)\\
        & & $N$ & 188 (181) & 188 (181) & 187 (180) & 187 (180) & 187 (180)\\
        \hline
        %\multicolumn{2}{|l|}{HMMs} & & & & & &\\
        HMMs & $MMI_{ext}$ ($MMI$) & Korrelation nach Pearson & 1 \newline (1) & -.178* (-.108) & -.208** (-.168*) & .148* (.073) & -.200** (-.134)\\
        & & Signifikanz (2-seitig) & & .011 (.122) & .003 (.015) & .035 (.294) & .004 (.055)\\
        & & $N$ & 202 (207) & 202 (207) & 202 (207) & 202 (207) & 202 (207)\\
        \hline
        %\multicolumn{2}{|l|}{NMMs} & & & & & &\\
        NMMs & $MMI_{ext}$ ($MMI$) & Korrelation nach Pearson & 1 \newline (1) & -.005 (.011) & .003 (.022) & .091** (.058) & -.054 (-.023)\\
        & & Signifikanz (2-seitig) & & .872 (.730) & .930 (.494) & .005 (.071) & .094 (.466)\\
        & & $N$ & 969 (971) & 966 (968) & 964 (966) & 964 (966) & 964 (966)\\
        \hline
        \multicolumn{8}{l}{**. Die Korrelation ist auf dem Niveau von 0.01 (2-seitig) signifikant.}\\
        \multicolumn{8}{l}{*. Die Korrelation ist auf dem Niveau von 0.05 (2-seitig) signifikant.}\\
    \end{tabular}
    \label{table.ergebnis.medienNutzung}
\end{table}

%MMI zu ACS
\section{Korrelation zwischen MMI und ACS}\label{anhangKorrelationen.mmiZuAcs}
%Tabelle MMI zu ACS
\begin{table}[H] 
    \centering
    \caption{Zusammenhang zwischen dem Medien-Multitasking und der Aufmerksamkeitskontrolle, Korrelationen}
    \begin{tabular}[t]{|l R{30mm}|r|R{16mm}|R{16mm}|} 
        \hline
        \multicolumn{5}{|c|}{\textbf{Korrelationen}}\\ 
        \hline       
        \multicolumn{2}{|c}{} & \multicolumn{1}{c|}{$MMI_{ext}$} & \multicolumn{1}{c|}{$MMI$} &\multicolumn{1}{c|}{$ACS$}\\
        \hline
        $MMI_{ext}$ & Korrelation nach Pearson & 1 & .990** & .078**\\
        & Signifikanz (2-seitig) & & .000 & .004\\
        & $N$ & 1359 & 1359 & 1359\\
        \hline
        $MMI$ & Korrelation nach Pearson & .990** & 1 & .077**\\
        & Signifikanz (2-seitig) & .000 & & .005 \\
         & $N$ & 1359 & 1359 & 1356\\
        \hline
        \multicolumn{5}{l}{**. Die Korrelation ist auf dem Niveau von 0.01 (2-seitig) signifikant.}
    \end{tabular}
    \label{table.korrelationMmiZuAcs}
\end{table}

% ACS zu SWB
\section{Korrelation zwischen MMI und SWB}\label{anhangKorrelationen.MmiZuSwb}
%Table Korrelation MMI SWB
\begin{table}[H] 
    \centering
    \caption{Zusammenhang zwischen Medien-Multitasking und dem subjektivem Wohlbefinden, Korrelationen}
    \begin{tabular}[t]{|L{20mm} R{25mm}|R{18mm}|R{15mm}|R{15mm}|R{15mm}|R{15mm}|} 
        \hline
        \multicolumn{7}{|c|}{\textbf{Korrelationen}}\\ 
        \hline       
        \multicolumn{2}{|c}{} & \multicolumn{1}{c|}{$MMI_{ext}$} &\multicolumn{1}{c|}{$FS$} & \multicolumn{3}{c|}{$SPANE$}\\
        \multicolumn{2}{|c|}{} & \multicolumn{1}{c|}{($MMI$)} & \multicolumn{1}{c|}{} &\multicolumn{1}{c|}{$P$} &  \multicolumn{1}{c|}{$N$} & \multicolumn{1}{c|}{$B$}\\
        \hline
        $MMI_{ext}$ \newline ($MMI$) & Korrelation nach Pearson & 1 (.990**) & -.015 (-.022) & -.016 (-.024) & .114** (.107**) & -.077** (077**)\\
        & Signifikanz (2-seitig) & - \newline (.000) & .575 (.419) & .566 (.387) & .000 (.000) & .005 (.005)\\
        & $N$ & 1359 & 1356 & 1353 & 1353 & 1353\\
        \hline
        \multicolumn{7}{l}{**. Die Korrelation ist auf dem Niveau von 0.01 (2-seitig) signifikant.}
    \end{tabular}
    \label{table.korrelationMmi}
\end{table}

% ACS zu SWB
\section{Korrelation zwischen ACS und SWB}\label{anhangKorrelationen.acsZuSwb}
%Table Korrelation MMI SWB
\begin{table}[H] 
    \centering
    \caption{Zusammenhang zwischen der Aufmerksamkeitskontrolle und dem subjektiven Wohlbefinden, Korrelationen}
    \begin{tabular}[t]{|l R{25mm}|R{15mm}|R{15mm}|R{15mm}|R{15mm}|R{15mm}|R{15mm}|} 
        \hline
        \multicolumn{7}{|c|}{\textbf{Korrelationen}}\\ 
        \hline       
        \multicolumn{2}{|c}{} & \multicolumn{1}{c|}{$ACS$} &\multicolumn{1}{c|}{$FS$} & \multicolumn{3}{c|}{$SPANE$}\\
        \multicolumn{2}{|c|}{} & \multicolumn{1}{c|}{} & \multicolumn{1}{c|}{} &\multicolumn{1}{c|}{$P$} &  \multicolumn{1}{c|}{$N$} & \multicolumn{1}{c|}{$B$}\\
        \hline
        $ACS$ & Korrelation nach Pearson & 1 & .265** & .280** & -.281** & .316** \\
        & Signifikanz (2-seitig) & & .000 & .000 & .000 & .000 \\
        & $N$ & 1367 & 1364 & 1361 & 1361 & 1361 \\
        \hline
        \multicolumn{7}{l}{**. Die Korrelation ist auf dem Niveau von 0.01 (2-seitig) signifikant.}
    \end{tabular}
    \label{table.korrelationAcsZuSwb}
\end{table}

% MMI zu ACS -- Geschlecht
\section{Korrelation zwischen MMI und ACS -- Aufteilung Geschlecht}\label{anhangKorrelationen.mmiZuAcsGeschlecht}

%Table MMI zu ACS -- Geschlecht
\begin{table}[H] 
    \centering
    \caption{Zusammenhang zwischen dem Medien-Multitasking und der Aufmerksamkeitskontrolle, Korrelationen aufgeteilt anhand dem Geschlecht}
    \begin{tabular}[t]{|L{15mm} R{25mm}|R{25mm}|R{25mm}|R{25mm}|} 
        \hline
        \multicolumn{5}{|c|}{\textbf{Korrelationen aufgeteilt nach dem Geschlecht}}\\ 
        \hline       
        \multicolumn{3}{|c|}{} & \multicolumn{1}{c|}{$MMI_{ext}$} &\multicolumn{1}{c|}{$ACS$} \\
        \multicolumn{3}{|l|}{Geschlecht} & \multicolumn{1}{c|}{($MMI$)} & \multicolumn{1}{c|}{}\\
        \hline
        weiblich & $MMI_{ext}$ ($MMI$) & Korrelation nach Pearson & 1 \newline (1) & .052 \newline (.051)\\
        & & Signifikanz (2-seitig) & & .111 \newline (.117)\\
        & & $N$ & 955 & 955\\
        \hline
        männlich & $MMI_{ext}$ ($MMI$) & Korrelation nach Pearson & 1 \newline (1) & .127* \newline (.127*) \\
        & & Signifikanz (2-seitig) & & .011 \newline (.011)\\
        & & $N$ & 401 & 401\\
        \hline
        \multicolumn{5}{l}{**. Die Korrelation ist auf dem Niveau von 0.01 (2-seitig) signifikant.}\\
        \multicolumn{5}{l}{*. Die Korrelation ist auf dem Niveau von 0.05 (2-seitig) signifikant.}\\
    \end{tabular}
    \label{table.ergebnis.mmiZuAcsGeschlecht}
\end{table}

% ACS zu SWB -- Geschlecht
\section{Korrelation zwischen ACS und SWB -- Aufteilung Geschlecht}\label{anhangKorrelationen.acsZuSwbGeschlecht}

%Table ACS zu SWB -- Geschlecht
%Table Korrelation MMI SWB
\begin{table}[H] 
    \centering
    \caption{Zusammenhang zwischen der Aufmerksamkeitskontrolle und dem subjektiven Wohlbefinden, Korrelationen aufgeteilt anhand dem Geschlecht}
    \begin{tabular}[t]{|l R{8mm}|R{25mm}|R{15mm}|R{15mm}|R{15mm}|R{15mm}|R{15mm}|} 
        \hline
        \multicolumn{8}{|c|}{\textbf{Korrelationen}}\\ 
        \hline       
        \multicolumn{3}{|c}{} & \multicolumn{1}{c|}{$ACS$} &\multicolumn{1}{c|}{$FS$} & \multicolumn{3}{c|}{$SPANE$}\\
        \multicolumn{3}{|l|}{Geschlecht} & \multicolumn{1}{c|}{} & \multicolumn{1}{c|}{} &\multicolumn{1}{c|}{$P$} &  \multicolumn{1}{c|}{$N$} & \multicolumn{1}{c|}{$B$}\\
        \hline
        weiblich &$ACS$ & Korrelation nach Pearson & 1 & .245** & .243** & -.247** & .276** \\
        && Signifikanz (2-seitig) & & .000 & .000 & .000 & .000 \\
        && $N$ & 960 & 958 & 956 & 956 & 956 \\
        \hline
        männlich &$ACS$ & Korrelation nach Pearson & 1 & .302** & .355** & -.389** & .414** \\
        && Signifikanz (2-seitig) & & .000 & .000 & .000 & .000 \\
        && $N$ & 404 & 403 & 402 & 402 & 402 \\
        \hline
        \multicolumn{8}{l}{**. Die Korrelation ist auf dem Niveau von 0.01 (2-seitig) signifikant.}
    \end{tabular}
    \label{table.korrelationAcsZuSwbGeschlecht}
\end{table}

\end{RaggedRight}