%%%%%%%%%%%%%%%%%%%%%%%%%%%%%%%%%%%%%%%%%%%%%%%%%%%%%%%%%%%%%%%%%
%_____________ ___    _____  __      __ 
%\____    /   |   \  /  _  \/  \    /  \  Institute of Applied
%  /     /    ~    \/  /_\  \   \/\/   /  Psychology
% /     /\    Y    /    |    \        /   Zuercher Hochschule 
%/_______ \___|_  /\____|__  /\__/\  /    fuer Angewandte Wissen.
%        \/     \/         \/      \/                           
%%%%%%%%%%%%%%%%%%%%%%%%%%%%%%%%%%%%%%%%%%%%%%%%%%%%%%%%%%%%%%%%%
%
% Project     : Bachelorarbeit
% Title       : 
% File        : Rev. 00
% Date        : 06.12.2013
% Author      : Till J. Ernst
%
%%%%%%%%%%%%%%%%%%%%%%%%%%%%%%%%%%%%%%%%%%%%%%%%%%%%%%%%%%%%%%%%%
\glsresetall

\let\raggedsection\centering 
\chapter{Übersetzung -- Scale of Positive and Negative Experience}\label{chap.appendix_spane}
\let\raggedsection\raggedright 
\begin{RaggedRight}
% SPANE
Die Skala für die Erfassung von positiven und negativen Erfahrungen (engl.: Scale of Positive and Negative Experience - SPANE) \cite{Diener:2010} besteht aus einer einleitenden Anweisung und den zugehörenden Fragebogen-Elemente. Bewertet werden die Elemente auf einer fünfstufiger Skala: 1 = sehr selten oder nie (engl.: very rarely or never); 2 = selten (engl.: rarely); 3 = manchmal (engl.: sometimes); 4 = oft (engl.: often); 5 = sehr oft oder immer (engl.: very often or always). 
% Table
\begin{center}
    \small
    \begin{longtable}[t]{|p{15 cm}|}
    \caption{Übersetzung Anweisung -- Scale of Positive and Negative Experience} \\ \hline
        \textbf{Englisch} \\ \hline
        Please think about what you have been doing and experiencing during the past 4 weeks. Then report how much you experienced each of the following feelings, using the scale below. For each item, select a number from 1 to 5, and indicate that number on your response sheet. \\ \hline
        \textbf{Deutsch} \\ \hline 
        Erinnere dich daran, was Du in den letzten vier Wochen gemacht hast und wie Du Dich dabei gefühlt hast. Anschliessend entscheide Dich anhand der unten stehenden Skala, wie oft Du diese unten stehenden Empfindungen erlebt hast. Bewerte jede Aussage auf einer Skala von 1 bis 5 (1 = sehr selten oder nie; 2 = selten; 3 = manchmal; 4 = oft; 5 = sehr oft oder immer). \\ \hline   
    \end{longtable}
	\label{tab:SpaneAnweisung}
\end{center}

% Table
\begin{center}
    \small
    \begin{longtable}[t]{|p{0.8 cm}|p{6.6 cm}|p{6.6 cm}|}
    \caption{Übersetzung Elemente -- Scale of Positive and Negative Experience} \\ \hline
        \textbf{Nr.} & \textbf{Englisch} & \textbf{Deutsch} \\ \hline
        \endfirsthead
        \hline
        \textbf{Nr.} & \textbf{Englisch} & \textbf{Deutsch} \\ \hline
        \endhead 
        & \multicolumn{2}{c|}{Fortsetzung auf der nächsten Seite $...$ } \\ \hline
        \endfoot
        \hline
        \endlastfoot
        1 & Positive & Positiv\\
        2 & Negative & Negativ\\
        3 & Good & Gut\\
        4 & Bad & Schlecht\\
        5 & Pleasant & Angenehm\\
        6 & Unpleasant & Unangenehm\\
        7 & Happy & Glücklich\\
        8 & Sad & Traurig\\
        9 & Afraid & Ängstlich\\
        10 & Joyful & Froh\\
        11 & Angry & Wütend\\
        12 & Contented & Zufrieden\\  
    \end{longtable}
	\label{tab:SpaneElemente}
\end{center}

\end{RaggedRight}
