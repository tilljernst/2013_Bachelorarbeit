%%%%%%%%%%%%%%%%%%%%%%%%%%%%%%%%%%%%%%%%%%%%%%%%%%%%%%%%%%%%%%%%%
%_____________ ___    _____  __      __ 
%\____    /   |   \  /  _  \/  \    /  \  Institute of Applied
%  /     /    ~    \/  /_\  \   \/\/   /  Psychology
% /     /\    Y    /    |    \        /   Zuercher Hochschule 
%/_______ \___|_  /\____|__  /\__/\  /    fuer Angewandte Wissen.
%        \/     \/         \/      \/                           
%%%%%%%%%%%%%%%%%%%%%%%%%%%%%%%%%%%%%%%%%%%%%%%%%%%%%%%%%%%%%%%%%
%
% Project     : Bachelorarbeit
% Title       : Diskussion
% File        : diskussion.tex Rev. 00
% Date        : 06.12.2013
% Author      : Till J. Ernst
%
%%%%%%%%%%%%%%%%%%%%%%%%%%%%%%%%%%%%%%%%%%%%%%%%%%%%%%%%%%%%%%%%%
\glsresetall
% Kapitel Diskussion
\let\raggedsection\centering
\addchap{Diskussion}
\setcounter{chapter}{4}
\setcounter{section}{0}
\let\raggedsection\raggedright 

%####################################################
% Kapitel Beantwortung der Fragestellung
%####################################################
\section{Beantwortung der Fragestellung}\label{section.diskussion.fragestellung}
Im Folgenden wird die Fragestellung beantwortet, die bezüglich Übersichtlichkeit in eine Hauptfragestellung und drei Nebenfragestellungen unterteilt wurde. 
\par
\subsection{Hauptfragestellung.} Beeinflusst die Häufigkeit und die Fähigkeit bezogen auf Medien-Multitasking das subjektive Wohlbefinden von Studierenden?
\par
Gemäss den Ergebnissen dieser Studie konnte kein direkter Einfluss der Fähigkeit zur Aufmerksamkeitskontrolle auf das Medien-Multitasking Verhalten und das subjektive Wohlbefinden von Studierenden nachgewiesen werden. Es konnte jedoch nachgewiesen werden, dass die Fähigkeit zur Aufmerksamkeitskontrolle einen direkten Einfluss auf das subjektive Wohlbefinden haben. Dies wurde in der Unterfragestellung 2 beantwortet. Die geringe Korrelation zwischen dem Medien-Multitasking und dem subjektiven Wohlbefinden von $r=.114$ bei einem Signifikanzniveau von $p=.01$ wurde nicht durch die Fähigkeit zur Aufmerksamkeitskontrolle beeinflusst. Die Partialkorrelation mittels Fähigkeit zur Aufmerksamkeitskontrolle ergab ein $r_{par}=.141$ bei einem Signifikanzniveau von $p=.000$ und $df=1350$. Somit ist der direkte Zusammenhang zwischen Medien-Multitasking und dem subjektiven Wohlbefinden nicht auf die Fähigkeit zur Aufmerksamkeitskontrolle zurückzuführen $r<r_{par}$. Für diese Fragestellung kann die Alternativhypothese 2 der Haupthypothese angenommen werden.

\subsection{Unterfragestellung 1} Welchen Einfluss hat die Häufigkeit von Medien-Multitasking auf das subjektive Wohlbefinden von Studierenden?
\par
Es wurde ein geringer signifikanter Zusammenhang zwischen der Variable für Medien-Multitasking und der Variable für das subjektive Wohlbefinden, in der Skala für negative Gefühle, von $r=.114$ bei einem Signifikanzniveau von $p=.01$ gefunden. Die Alternativhypothese der Arbeitshypothese 1 kann somit angenommen werden. Es scheint, dass die Häufigkeit von Medien-Multitasking einen negativen Einfluss auf das subjektive Wohlbefinden hat. \\
Wird eine Aufteilung der Stichprobe anhand des Geschlechts vorgenommen, so kann ein geringer negativ signifikanter Zusammenhang bei den weiblichen Teilnehmern ($N=951$) zwischen dem Medien-Multitasking und der Skala für positive Gefühle von $r=-.126$ bei einem Signifikanzniveau von $p=.01$ festgestellt werden. Bei den Männern ($N=401$) konnte kein Zusammenhang gefunden werden. Erfolgt die Unterteilung der Stichprobe anhand dem Multitasking Index der Personen, die stark Medien-Multitasking betreiben ($N=202$), können gering signifikante Zusammenhänge zwischen dem Medien-Multitasking und dem subjektiven Wohlbefinden ausgemacht werden. Die Auswirkung auf die Skala des  menschlichen Aufblühens ergab eine geringe negativ signifikante Korrelation von $r=-.178$ bei einem Signifikanzniveau von $p=.05$. Auf die Skala positive und negative Gefühle wurde eine geringe negativ signifikante Korrelation von $r=-.200$ bei einem Signifikanzniveau von $p=.01$ festgestellt. Auch für diese Zusammenhänge unterteilt nach Gruppen kann die Alternativhypothese der Arbeitshypothese 1 angenommen werden. Es scheint, dass Medien-Multitasking einen negativen Einfluss auf das subjektive Wohlbefinden hat.

\subsection{Unterfragestellung 2.} Welchen Einfluss hat die Fähigkeit zur Aufmerksamkeitskontrolle von Studierenden auf deren subjektives Wohlbefinden?

Zwischen der Fähigkeit zur Aufmerksamkeitskontrolle und dem subjektive Wohlbefinden konnten mehrere signifikante Ergebnisse gefunden werden. Ein geringer signifikanter Zusammenhang zwischen der Fähigkeit zur Aufmerksamkeitskontrolle und der Skala für das menschliche Aufblühen ($N=1364$) von $r=.265$ und ein mittlerer Zusammenhang zwischen der Aufmerksamkeitskontrolle und der Skala für positive und negative Gefühle ($N=1361$) von $r=.316$ bei einem Signifikanzniveau von $p=.01$. Die Alternativhypothese der Arbeitshypothese 2 kann somit angenommen werden. Die Fähigkeit zur Aufmerksamkeitskontrolle scheint einen geringen bis mittleren Einfluss auf das subjektive Wohlbefinden der Studierenden zu haben.

\subsection{Unterfragestellung 3.} Welchen Einfluss hat die Fähigkeit zur Aufmerksamkeitskontrolle auf das Medien-Multitasking-Verhalten von Studierenden?

Zwischen der Fähigkeit zur Aufmerksamkeitskontrolle und dem Medien-Multitasking konnte kein direkter Zusammenhang bei den Studierenden gefunden werden. Für diese Fragestellung muss somit die Nullhypothese der Arbeitshypothese 3 angenommen werden. Somit scheint es, dass die Fähigkeit zur Aufmerksamkeitskontrolle keinen Einfluss auf das Medien-Multitasking in dieser Untersuchung hat. \\
Jedoch lieferte die Unterteilung der Stichprobe anhand des Geschlechts ein geringer signifikanter Zusammenhang von $r=.127$ für die Kategorie Männer ($N=401$), zwischen dem Medien-Multitasking und der Aufmerksamkeitskontrolle bei einem Signifikanzniveau von $p=.05$. Einen weiteren gering signifikanten Zusammenhang zwischen der Fähigkeit zur Aufmerksamkeitskontrolle und dem Medien-Multitasking konnte bei der Aufteilung des Alters in Kategorien anhand Kohorten gefunden werden. Hierbei konnte bei der Kohorte Digital Immigrants ($N=216$) einen geringer Zusammenhang von $r=.137$ bei einem Signifikanzniveau von $p=.05$ gefunden werden. Die Aufteilung der Stichprobe anhand der Elternschaft ($N=79$), hatten die Probanden Kinder zum Zeitpunkt der Untersuchung, lieferte einen gering signifikanten Zusammenhang zwischen dem Medien-Multitasking und der Fähigkeit zur Aufmerksamkeitskontrolle von $r=.253$ bei einem Signifikanzniveau von $p=.05$. Für diese Unterteilung der Stichprobe in Kategorien kann die Nullhypothese der Arbeitshypothese 3 verworfen werden und die Alternativhypothese angenommen werden. Es scheint, dass die Fähigkeit zur Aufmerksamkeitskontrolle in spezifischen Gruppen einen Einfluss auf das Medien-Multitasking hat.

%####################################################
% Kapitel Interpretation
%####################################################
\section{TBD Interpretation}\label{section.diskussion.interpretation}
Folgende Themen sollen in diesem Kapitel behandelt werden (TBD):
\begin{itemize}
    \item HMM stärkere Auswirkung auf das SWB -> zusätzliche Hinweise zu bestehender Theorie
    \item Geschlecht (w) stärkere Auswirkung auf SWB -> nicht naheliegend; in der Theorie Unterschiede zwischen w und m nicht bekannt
    \item unerwartetes Ergebnis der Studenten, die unter (oder gleich) 24 Jahre alt waren -> negative Auswirkung auf das SWB
    \item keine direkte Auswirkung auf die Fähigkeit zur Aufmerksamkeitskontrolle. Jedoch wirken sich Kinder, das Geschlecht und das Alter auf diese Komponente aus
    \item direkter Zusammenhang zwischen Fähigkeit zur Aufmerksamkeitskontrolle und SWB -> anhand der Theorie der positiven Psychologie (Flow) erklärbar
    \item Von Mail Rivka: Was mir gerade vorher beim Einscannen noch aufgefallen ist: Im Titel heisst es ja „Beeinflusst Multitasking das eigene Glück?“. Diese Frage nimmst du nirgends auf in der Arbeit bzw. nur in einer ganz anderen / wissenschaftlichen Sprache. Allenfalls könnte man diese umgangssprachliche Frage in der Einleitung bzw. beim Ziel und/oder bei der Zusammenfassung der Ergebnisse noch einbringen. Im Stile von: „Die saloppe Frage, welche im Titel dieser Arbeit aufgeworfen wird, nämlich ob das Multitasking das eigene Glück beeinflusst, kann damit wie folgt beantwortet werden: ….“.
\end{itemize}
Gemäss den Ergebnissen der Gruppe von stark Medien-Multitasking betreibenden Studenten konnte ein durchgehend geringer Zusammenhang zwischen Medien-Multitasking und dem subjektiven Wohlbefinden auf alle SWB-Skalen gefunden werden. Dieses Ergebnis ist insofern interessant, da es in diesem Bereich wenig existierende Studien gibt und sich diese gegenseitig eher widersprechen \cite{Pea2012, Shih2013}. Durch dieses konnte zumindest dargestellt werden, dass die Häufigkeit von Medien-Multitasking bei Studierenden eine mögliche Auswirkung auf das subjektive Wohlbefinden hat. Ob es sich beim Medien-Multitasking um die Auslösende Variable handelt oder ob eine weitere, verdeckte Variable für diesen Zusammenhang verantwortlich ist, müssen weiterführende Studien erst noch belegen. 
%####################################################
% Kapitel Methodenkritik
%####################################################
\section{TBD Methodenkritik}\label{section.diskussion.methodenkritik}
Folgende Themen sollen in diesem Kapitel behandelt werden (TBD):
\begin{itemize}
    \item $MMI_{ext}$ und $MMI$ einmal mit SMS als Hauptmedium und einmal ohne SMS als Hauptmedium sondern nur als Nebenmedium (originale Version). Dies, da die Nutzung pro Tag und Minute erfragt wurde (orig. in Stunden pro Woche)
    \item MUQ - Alle Medien werden gleich bewertet (z.B.: sms und tv)
    \item Im Onlinefragebogen bei der Sektion MUQ wurden bereits vorgewählte Antworten bei den simultan verwendeten Medien verwendet (ein allfälliges Nebenmedium wird nicht verwendet)
    \item ACS - Selbstbeurteilung der Fähigkeit zur Aufmerksamkeitskontrolle (Frage bezüglich Validität)
\end{itemize}


%####################################################
% Kapitel Ausblick
%####################################################
\section{Ausblick}\label{section.diskussion.ausblick}
Die Ergebnisse dieser Studie liefern eine gute Grundlage für weitere Untersuchungen im Bereich des Medien-Multitasking. Durch die grosse Stichprobe können weitere Zusammenhänge und Eigenschaften des Medien-Multitasking Verhaltens untersucht werden, die in dieser Studie keinen Platz fanden. Die Resultate lassen die Annahme zu, dass es einen Zusammenhang zwischen dem Medien-Multitasking und dem subjektiven Wohlbefinden geben könnte. Dieser Bereich benötigt zusätzliche Forschung und eine vertiefte Auseinandersetzung mit dem Thema. Die fortführende Auseinandersetzung mit der Fähigkeit zur Aufmerksamkeitskontrolle und die Auswirkung von Multitasking auf diese benötigt weitere Ergebnisse. Ein spannendes Thema könnte die Befassung mit möglichen Einflussfaktoren neben der Fähigkeit zur Aufmerksamkeitskontrolle auf das subjektive Wohlbefinden sein. Zum Beispiel könnten soziale Faktoren wie Umgebung, Beziehung oder auch Bildung mögliche Wirkfaktoren für das Multitasking verhalten und die Auswirkung auf das subjektive Wohlbefinden darstellen. Ebenso ist vorstellbar, dass das subjektive Wohlbefinden in weitere Teilbereiche unterteilt wird, um eine allfällige Beeinflussung des Multitasking aufzuzeigen. Eine zentrale Rolle in dieser Diskussion scheint der Langzeitauswirkung von Medien-Multitasking zuzukommen. Wie wirken sich diese Medien-Präferenzen auf unser Gehirn und deren Areale aus. Tendiert unser Gehirn dazu, umgebungsbedingte Einflüsse zu adaptieren oder stossen wir an unsere Kapazitäts- und Leistungsgrenzen, was unsere kognitiven Fähigkeiten anbelangt. Diese notwendigen Untersuchungen sollen dazu beitragen, den künftigen Umgang mit Medien und wie Medien technologisch umgesetzt werden, zu erleichtern. Eine auf unsere Bedürfnisse und Fähigkeiten abgestimmte Medienwelt, die unser Potential anregt und unsere Fähigkeiten erweitert, ist einer von Reizen überfluteten, teilweise überfordernden und stressverursachenden Umwelt definitiv vorzuziehen. 

