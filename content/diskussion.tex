%%%%%%%%%%%%%%%%%%%%%%%%%%%%%%%%%%%%%%%%%%%%%%%%%%%%%%%%%%%%%%%%%
%_____________ ___    _____  __      __ 
%\____    /   |   \  /  _  \/  \    /  \  Institute of Applied
%  /     /    ~    \/  /_\  \   \/\/   /  Psychology
% /     /\    Y    /    |    \        /   Zuercher Hochschule 
%/_______ \___|_  /\____|__  /\__/\  /    fuer Angewandte Wissen.
%        \/     \/         \/      \/                           
%%%%%%%%%%%%%%%%%%%%%%%%%%%%%%%%%%%%%%%%%%%%%%%%%%%%%%%%%%%%%%%%%
%
% Project     : Bachelorarbeit
% Title       : Diskussion
% File        : diskussion.tex Rev. 00
% Date        : 06.12.2013
% Author      : Till J. Ernst
%
%%%%%%%%%%%%%%%%%%%%%%%%%%%%%%%%%%%%%%%%%%%%%%%%%%%%%%%%%%%%%%%%%
\glsresetall
% Kapitel Diskussion
\let\raggedsection\centering
\chapter{TBD Diskussion (meist Gegenwart)}
\let\raggedsection\raggedright 
% Beantwortung der Fragestellung
\section{Beantwortung der Fragestellung}\label{section.diskussion.fragestellung}
Im Folgenden wird die Fragestellung beantwortet. Die bezüglich Übersichtlichkeit in eine Hauptfragestellung und drei Nebenfragestellungen unterteilt wurde. 
\par
\textbf{Hauptfragestellung:} Beeinflusst die Häufigkeit und die Fähigkeit bezogen auf Medien-Multitasking das subjektive Wohlbefinden von Studierenden?
\par
Gemäss den Ergebnissen dieser Studie konnte keinen direkten Einfluss der kognitiven Fähigkeit auf das Medien-Multitasking Verhalten und das subjektive Wohlbefinden von Studierenden nachgewiesen werden. Es konnte jedoch nachgewiesen werden, dass die kognitiven Fähigkeiten einen direkten Einfluss auf das subjektive Wohlbefinden haben. Dies wurde in der Unterfragestellung 2 beantwortet. Die geringe Korrelation zwischen dem Medien-Multitasking und dem subjektiven Wohlbefinden von $r=.114$ bei einem Signifikanzniveau von $p=.01$ wurde nicht durch die kognitiven Fähigkeiten beeinflusst. Die Partialkorrelation mittels kognitiver Fähigkeit ergab ein $r_{par}=.141$ bei einem Signifikanzniveau von $p=.000$ und $df=1350$. Somit ist der direkter Zusammenhang zwischen Medien-Multitasking und dem subjektiven Wohlbefinden nicht auf die kognitiven Fähigkeiten zurückzuführen $r<r_{par}$. Für diese Fragestellung kann die Alternativhypothese 2 der Haupthypothese angenommen werden.
\par
\textbf{Unterfragestellung 1:} Welchen Einfluss hat die Häufigkeit von Medien-Multitasking auf das subjektive Wohlbefinden von Studierenden?
\par
Es wurde ein geringer signifikanter Zusammenhang zwischen der Variable für Medien-Multitasking und der Variable für das subjektive Wohlbefinden, in der Skala für negative Gefühle, von $r=.114$ bei einem Signifikanzniveau von $p=.01$ gefunden. Die Alternativhypothese der Arbeitshypothese 1 kann somit angenommen werden. Es scheint, als ob die Häufigkeit von Medien-Multitasking einen negativen Einfluss auf das subjektive Wohlbefinden hat. 
\par
\textbf{Unterfragestellung 2:} Welchen Einfluss hat die kognitive Fähigkeit von Studierenden auf deren subjektives Wohlbefinden?
\par
Zwischen der kognitiven Fähigkeit, abgebildet durch die Variable Aufmerksamkeitskontrolle, und dem subjektive Wohlbefinden konnten mehrere signifikante Ergebnisse gefunden werden. Ein geringer signifikanter Zusammenhang zwischen der Aufmerksamkeitskontrolle und der Skala für das menschliche Aufblühen ($N=1364$) von $r=.265$ und ein mittlerer Zusammenhang zwischen der Aufmerksamkeitskontrolle und der Skala für positive und negative Gefühle ($N=1361$) von $r=.316$ bei einem Signifikanzniveau von $p=.01$. Die Alternativhypothese der Arbeitshypothese 2 kann somit angenommen werden. Der Einfluss der Variable für kognitive Fähigkeiten scheint einen geringen bis mittleren Einfluss auf das subjektive Wohlbefinden der Studierenden zu haben.
\par
\textbf{Unterfragestellung 3:} Welchen Einfluss hat die kognitive Fähigkeit auf das Medien-Multitasking Verhalten von Studierenden?
\par
Zwischen der kognitiven Fähigkeit bei Studierenden konnte kein direkter Zusammenhang auf das Medien-Multitasking gefunden werden. Für diese Fragestellung muss somit die Nullhypothese der Arbeitshypothese 3 angenommen werden. Somit scheint es, dass die kognitive Fähigkeit keinen Einfluss auf das Medien-Multitasking in dieser Untersuchung hat. \\
Jedoch lieferte die Unterteilung der Stichprobe anhand des Geschlechts ein geringer signifikanter Zusammenhang von $r=.127$ für die Kategorie Männer ($N=401$), zwischen dem Medien-Multitasking und der Aufmerksamkeitskontrolle bei einem Signifikanzniveau von $p=.05$. Einen weiteren gering signifikanten Zusammenhang zwischen der kognitiven Kontrolle und dem Medien-Multitasking konnte bei der Aufteilung des Alters in Kategorien anhand Kohorten gefunden werden. Hierbei konnte bei der Kohorte Digital Immigrants ($N=216$) einen geringer Zusammenhang von $r=.137$ bei einem Signifikanzniveau von $p=.05$ gefunden werden. Die Aufteilung der Stichprobe anhand der Elternschaft ($N=79$), hatten die Probanden Kinder zum Zeitpunkt der Untersuchung, lieferte einen gering signifikanten Zusammenhang zwischen dem Medien-Multitasking und der Aufmerksamkeitskontrolle von $r=.253$ bei einem Signifikanzniveau von $p=.05$. Für diese Unterteilung der Stichprobe in Kategorien kann die Nullhypothese der Arbeitshypothese 3 verworfen werden und die Alternativhypothese angenommen werden. Es scheint als ob die kognitive Fähigkeit in spezifischen Gruppen einen Einfluss auf das Medien-Multitasking hat.
% Interpretation
\section{Interpretation}\label{section.diskussion.interpretation}
% Methodenkritik
\section{Methodenkritik}\label{section.diskussion.methodenkritik}
% Ausblick
\section{Ausblick}\label{section.diskussion.ausblick}


