%%%%%%%%%%%%%%%%%%%%%%%%%%%%%%%%%%%%%%%%%%%%%%%%%%%%%%%%%%%%%%%%%
%_____________ ___    _____  __      __ 
%\____    /   |   \  /  _  \/  \    /  \  Institute of Applied
%  /     /    ~    \/  /_\  \   \/\/   /  Psychology
% /     /\    Y    /    |    \        /   Zuercher Hochschule 
%/_______ \___|_  /\____|__  /\__/\  /    fuer Angewandte Wissen.
%        \/     \/         \/      \/                           
%%%%%%%%%%%%%%%%%%%%%%%%%%%%%%%%%%%%%%%%%%%%%%%%%%%%%%%%%%%%%%%%%
%
% Project     : Bachelorarbeit
% Title       : Diskussion
% File        : diskussion.tex Rev. 00
% Date        : 06.12.2013
% Author      : Till J. Ernst
%
%%%%%%%%%%%%%%%%%%%%%%%%%%%%%%%%%%%%%%%%%%%%%%%%%%%%%%%%%%%%%%%%%
\glsresetall
% Kapitel Diskussion
\let\raggedsection\centering
\addchap{Diskussion}
\setcounter{chapter}{4}
\setcounter{section}{0}
\let\raggedsection\raggedright 

%####################################################
% Kapitel Beantwortung der Fragestellung
%####################################################
\addsec{Beantwortung der Fragestellung}\label{section.diskussion.fragestellung}
Im Folgenden wird die Fragestellung beantwortet, die bezüglich Übersichtlichkeit in eine Hauptfragestellung und drei Nebenfragestellungen unterteilt wurde. 
\par
\subsection{Hauptfragestellung.} Beeinflusst die Häufigkeit und die Fähigkeit bezogen auf Medien-Multitasking das subjektive Wohlbefinden von Studierenden?
\par
Gemäss den Ergebnissen dieser Studie konnte kein direkter Einfluss der Fähigkeit zur Aufmerksamkeitskontrolle auf das Medien-Multitasking Verhalten und das \gls{labelSWB} von Studierenden nachgewiesen werden. Es konnte jedoch nachgewiesen werden, dass die Fähigkeit zur Aufmerksamkeitskontrolle einen direkten Einfluss auf das \gls{labelSWB} haben. Dies wurde in der Unterfragestellung 2 beantwortet. Die geringe Korrelation zwischen dem Medien-Multitasking und dem subjektiven Wohlbefinden von $r=.114$ bei einem Signifikanzniveau von $p=.01$ wurde nicht durch die Fähigkeit zur Aufmerksamkeitskontrolle beeinflusst. Die Partialkorrelation mittels Fähigkeit zur Aufmerksamkeitskontrolle ergab ein $r_{par}=.141$ bei einem Signifikanzniveau von $p=.000$ und $df=1350$. Somit ist der direkte Zusammenhang zwischen Medien-Multitasking und dem \gls{labelSWB} nicht auf die Fähigkeit zur Aufmerksamkeitskontrolle zurückzuführen $r<r_{par}$. Für diese Fragestellung kann die Alternativhypothese 2 der Haupthypothese angenommen werden.

\subsection{Unterfragestellung 1} Welchen Einfluss hat die Häufigkeit von Medien-Multitasking auf das subjektive Wohlbefinden von Studierenden?
\par
Es wurde ein geringer signifikanter Zusammenhang zwischen der Variable für Medien-Multitasking und der Variable für das subjektive Wohlbefinden, in der Skala für negative Gefühle, von $r=.114$ bei einem Signifikanzniveau von $p=.01$ gefunden. Die Alternativhypothese der Arbeitshypothese 1 kann somit angenommen werden. Es scheint, dass die Häufigkeit von Medien-Multitasking einen negativen Einfluss auf das \gls{labelSWB} hat. \\
Wird eine Aufteilung der Stichprobe anhand des Geschlechts vorgenommen, so kann ein geringer negativ signifikanter Zusammenhang bei den weiblichen Teilnehmern ($N=951$) zwischen dem Medien-Multitasking und der Skala für positive Gefühle von $r=-.126$ bei einem Signifikanzniveau von $p=.01$ festgestellt werden. Bei den Männern ($N=401$) konnte kein Zusammenhang gefunden werden. Erfolgt die Unterteilung der Stichprobe anhand dem Multitasking Index der Personen, die stark Medien-Multitasking betreiben ($N=202$), können gering signifikante Zusammenhänge zwischen dem Medien-Multitasking und dem subjektiven Wohlbefinden ausgemacht werden. Die Auswirkung auf die Skala des  menschlichen Aufblühens ergab eine geringe negativ signifikante Korrelation von $r=-.178$ bei einem Signifikanzniveau von $p=.05$. Auf die Skala positive und negative Gefühle wurde eine geringe negativ signifikante Korrelation von $r=-.200$ bei einem Signifikanzniveau von $p=.01$ festgestellt. Auch für diese Zusammenhänge unterteilt nach Gruppen kann die Alternativhypothese der Arbeitshypothese 1 angenommen werden. Es scheint, dass Medien-Multitasking einen negativen Einfluss auf das subjektive Wohlbefinden hat.

\subsection{Unterfragestellung 2.} Welchen Einfluss hat die Fähigkeit zur Aufmerksamkeitskontrolle von Studierenden auf deren subjektives Wohlbefinden?

Zwischen der Fähigkeit zur Aufmerksamkeitskontrolle und dem \gls{labelSWB} konnten mehrere signifikante Ergebnisse gefunden werden. Ein geringer signifikanter Zusammenhang zwischen der Fähigkeit zur Aufmerksamkeitskontrolle und der Skala für das menschliche Aufblühen ($N=1364$) von $r=.265$ und ein mittlerer Zusammenhang zwischen der Aufmerksamkeitskontrolle und der Skala für positive und negative Gefühle ($N=1361$) von $r=.316$ bei einem Signifikanzniveau von $p=.01$. Die Alternativhypothese der Arbeitshypothese 2 kann somit angenommen werden. Die Fähigkeit zur Aufmerksamkeitskontrolle scheint einen geringen bis mittleren Einfluss auf das subjektive Wohlbefinden der Studierenden zu haben.

\subsection{Unterfragestellung 3.} Welchen Einfluss hat die Fähigkeit zur Aufmerksamkeitskontrolle auf das Medien-Multitasking-Verhalten von Studierenden?

Zwischen der Fähigkeit zur Aufmerksamkeitskontrolle und dem Medien-Multitasking konnte kein direkter Zusammenhang bei den Studierenden gefunden werden. Für diese Fragestellung muss somit die Nullhypothese der Arbeitshypothese 3 angenommen werden. Somit scheint es, dass die Fähigkeit zur Aufmerksamkeitskontrolle keinen Einfluss auf das Medien-Multitasking in dieser Untersuchung hat. \\
Jedoch lieferte die Unterteilung der Stichprobe anhand des Geschlechts ein geringer signifikanter Zusammenhang von $r=.127$ für die Kategorie Männer ($N=401$), zwischen dem Medien-Multitasking und der Aufmerksamkeitskontrolle bei einem Signifikanzniveau von $p=.05$. Einen weiteren gering signifikanten Zusammenhang zwischen der Fähigkeit zur Aufmerksamkeitskontrolle und dem Medien-Multitasking konnte bei der Aufteilung des Alters in Kategorien anhand Kohorten gefunden werden. Hierbei konnte bei der Kohorte Digital Immigrants ($N=216$) einen geringer Zusammenhang von $r=.137$ bei einem Signifikanzniveau von $p=.05$ gefunden werden. Die Aufteilung der Stichprobe anhand der Elternschaft ($N=79$), hatten die Probanden Kinder zum Zeitpunkt der Untersuchung, lieferte einen gering signifikanten Zusammenhang zwischen dem Medien-Multitasking und der Fähigkeit zur Aufmerksamkeitskontrolle von $r=.253$ bei einem Signifikanzniveau von $p=.05$. Für diese Unterteilung der Stichprobe in Kategorien kann die Nullhypothese der Arbeitshypothese 3 verworfen werden und die Alternativhypothese angenommen werden. Es scheint, dass die Fähigkeit zur Aufmerksamkeitskontrolle in spezifischen Gruppen einen Einfluss auf das Medien-Multitasking hat.

%####################################################
% Kapitel Interpretation
%####################################################
\addsec{Interpretation}\label{section.diskussion.interpretation}
Beeinflusst nun das Multitasking das eigene Glück? Diese salopp aufgeworfene Frage aus dem Titel, nämlich ob das Multitasking das eigene Glück beeinflusst, wird im Folgenden zu beantworten versucht.
 
\subsection{Unterschiede und Gemeinsamkeiten in der aktuellen Forschung.} Ge\-mäss den Ergebnissen der Gruppe von Studierenden, die stark Medien-Multitasking betreiben (HMMs), konnte ein durchgehend geringer Zusammenhang zwischen \gls{labelMMT} und dem \gls{labelSWB} in allen \gls{labelSWB}-Skalen gefunden werden. Dies ist deshalb interessant, weil in der Studie von \citeA{Shih2013} kein Zusammenhang zwischen \gls{labelMMT} und \gls{labelSWB} nachgewiesen werden konnte. Die Studie von \citeauthor{Shih2013} erfasste die Medien-Aktivitäten von 138 Studierenden im Alter zwischen 18 und 43 Jahren und setzte dazu einerseits einen speziell für diese Umfrage entwickelten Fragebogen ein, den \textit{\gls{labelSPD}}, und andererseits den \gls{labelMUQ}. Für die Erfassung des \gls{labelSWB} wurden Fragebögen zu Sensation Seeking, dem \gls{labelFFM}, der Aufmerksamkeitskontrolle und Impulsivität erfasst. Im Gegensatz zur vorliegenden Studie, wurde die Aufmerksamkeitskontrolle als einen Teil des \gls{labelSWB} aufgefasst \cite{Fergus2012}. Diese Zuordnung kann insofern nachvollzogen werden, da in der vorliegenden Studie ein kleiner bis mittlerer Zusammenhang zwischen der Aufmerksamkeitskontrolle und den \gls{labelSWB}-Skalen gefunden werden konnte. Somit könnten sich diese beiden Konstrukte gegenseitig beeinflussen. Die unterschiedlichen Ergebnisse bezüglich einer Auswirkung auf das \gls{labelSWB} könnte auf die geringere Stichprobengrösse und die unterschiedlichen Erhebungsinstrumente für die Erfassung des \gls{labelSWB} und der Medien-Nutzung zurückzuführen sein. Beide Studien erfassten die Mediennutzung mittels \gls{labelMUQ}. Dieser wurde jedoch für die vorliegende Studie angepasst, indem er anstelle der eingeschätzten Mediennutzung pro Woche, diese pro Tag erfasste. Eine weitere Abweichung zu der Studie von \citeA{Shih2013} konnte bei der Dauer der Mediennutzung gefunden werden. In der vorliegenden Studie wurde eine durchschnittliche Mediennutzung von 49.28 Stunden pro Woche gemessen. Bei der Studie von \citeauthor{Shih2013} wurde die durchschnittliche Nutzung mit 97.73 Stunden angegeben. Dieser Unterschied könnte darauf beruhen, dass die Selbsteinschätzung der Mediennutzung für eine Woche unterschiedlich ausfällt, als wenn die Einschätzung pro Tag vorgenommen wird. Neben dem \gls{labelMUQ} wurde zusätzlich der \gls{labelSPD} angewendet. Dieser erfasste die Mediennutzung mittels Tagebuch und wurde in der Abfolge vor dem \gls{labelMUQ} durchgeführt. Dies könnte dazu geführt haben, dass die Probanden die Selbsteinschätzung des \gls{labelMUQ} genauer vornehmen konnten, da durch den vorgängigen Aufschrieb der Mediennutzung bereits eine zeitliche Angabe stattgefunden hatte und sich die Studierenden beim Ausfüllen des \gls{labelMUQ} möglicherweise an diese Angabe erinnern konnten. Diese Unterschiede könnten auch der Grund für die Differenz des $MMI$ der beiden Umfragen sein: Der Mittelwert der Studie von \citeA{Shih2013} belief sich auf $M=3.41$ ($SD=1.28$) im Gegensatz zur vorliegend Studie mit einem Mittelwert von $M=1.35$ ($SD=1.00$). \citeauthor{Shih2013} konnte keine einzige Person erfassen, die nicht Medien-Multitasking betrieben hatte (Minimum $MMI = 0.49$). Die vorliegende Studie erfasste 45 Personen (3.3\%), die kein Medien-Multitasking betrieben haben. Dies spricht dafür, dass die Personen in dieser Umfrage gegenüber der Studie von \citeauthor{Shih2013} weniger \gls{labelMMT} betrieben haben. Werden diese Resultate der Mediennutzung mit der Umfrage der Kaiser Family Foundation verglichen \cite{Rideout2010}, so fällt hier auch ein Unterschied auf, diesmal jedoch in die andere Richtung: Gemäss \citeA{Wallis2010} betreiben 15 bis 20\% der über 2000 befragten jugendlichen Amerikaner zwischen 8 und 18 Jahren kein \gls{labelMMT}. Dieser Unterschied könnte sich damit erklären lassen, dass die vorliegende Stichprobe ausschliesslich Studierende befragte. Diese könnten Medien aufgrund ihres Studiums nutzen (z.B.: Computer) und deshalb eher zu Multitasking neigen. Weiter konnte \citeA{Rideout2010} und ihr Team eine Mediennutzungsdauer von 7.5 Stunden pro Tag feststellen. Dieses Resultat deckt sich annähernd mit dem Resultat der vorliegenden Studie von 49.3 Stunden pro Woche, was einem Tagesdurchschnitt von 7.04 Stunden entspricht. Wobei hier zu erwähnen ist, dass die Studie um \cite<ebda.,>{Rideout2010} zwischen der reinen Mediennutzung und der kumulireten Mediennutzung durch simultane Medien-Nutzung durch Multitasking unterscheidet. In der vorliegenden Studie wurde die Mediennutzung im Total erfasst (inkl. Multitasking). Aus dieser Perspektive wurden bei den Probanden der Kaiser Familiy Foundation 10.75 Stunden Mediennutzung inklusive Multitasking erfasst. Diese Differenz von über 3 Stunden könnte sich anhand dem Alter der Probanden erklären lassen: Je Jünger die Stichprobe, desto mehr Multitasking wird angewendet. Diese Hypothese müsste jedoch anhand der vorliegenden Daten weiter untersucht werden. Einen weiterer Grund für die unterschiedliche Resultate könnte aufgrund der unterschiedlichen Nationalitäten entstanden sein. Eventuell wenden amerikanische Jugendliche mehr \gls{labelMMT} an als Studierende in der Schweiz. Wie bereits bei der Studie um \citeA{Shih2013} könnte die unterschiedliche Mediennutzung aufgrund der verschiedenen Erhebungsinstrumente aufgetreten sein (Tagebuch gegenüber Selbsteinschätzung der Nutzung in Minuten und Tag). Die Selbsteinschätzung mittels \gls{labelMUQ} könnte gegenüber einem Medien-Tagebuch ungenaue Werte liefern \cite{Greenberg2005}.\\
Die Auswirkung von \gls{labelMMT} auf die Fähigkeit zur Aufmerksamkeitskontrolle konnte in der vorliegenden Studie nicht nachgewiesen werden. Im Gegensatz dazu konnten die Forschungsergebnissen von \citeA{Ophir2009} aufzeigen, dass starkes Media-Multitasking mit einem erhöhter kognitiven Kontrolle für das Herausfiltern von ablenkenden Umgebungsreizen und einer erhöhten Schwierigkeit für das Wechseln zwischen einfachen Aufgaben, die das Arbeitsgedächtnis betreffen, einhergeht. Die Studie untersuchte 262 Universitätstudierende zu ihrem Medien-Multitasking-Verhalten mittels \gls{labelMUQ}. Der \gls{labelMUQ} ergab einen Mittelwert von $M=4.38$ und einer Standardabweichung von $SD=1.52$. Anschliessend wurden diejenigen Studierenden, die stark \gls{labelMMT} betreiben und diejenigen, die schwach \gls{labelMMT} betreiben bezüglich ihrer kognitiven Fähigkeiten untersucht. Starke \gls{labelMMT}-Studierenden werden von mehreren gleichzeitigen Medien eher abgelenkt und können sich nicht auf ein Medium konzentrieren. Schwache \gls{labelMMT}-Studierende haben die Fähigkeit, die Aufmerksamkeitskontrolle auf eine Aufgabe zu lenken und störende Reize auszublenden. Diese Resultate konnten in der vorliegenden Studie mittels Auswirkung von \gls{labelMMT} auf die Fähigkeit zur Aufmerksamkeitskontrolle nicht bestätigt werden. Dieser Unterschied der beiden Resultate könnte aufgrund der unterschiedlichen Messung der kognitiven Fähigkeiten herrühren. Die Studie um \citeA{Ophir2009} setzte anstelle eines Selbstbeurteilungsfragebogens einen Testcomputer für die Messung der kognitiven Tests ein. Diese Messung könnte womöglich zu valideren Resultaten der kognitiven Fähigkeiten und des Arbeitsgedächtnis geführt haben. Ein weiterer möglicher Grund könnten die unterschiedlichen Ergebnisse des \gls{labelMUQ} sein. Diese fielen in der vorliegenden Studie klar geringer aus ($M=1.35$ und $SD=1.00$), was auf die Erfassung der Mediennutzung in Minuten pro Tag, anstelle von Stunden pro Woche zurückzuführen sein könnte. Die Resultate von \citeA{Ophir2009} konnten jedoch in der Wiederholungsstudie von \citeA{Alzahabi2013} nicht wiederholt werden. Sie stellten fest, dass starkes \gls{labelMMT} mit einer besseren Leistung im Wechseln zwischen Aufgaben einhergeht. In der vorliegenden Arbeit kann eine Tendenz in diese Richtung erkannt werden, in dem der \gls{labelMMI} anhand einer Gruppierung nach Geschlecht, Elternschaft oder Alter unterteilt wird. Je mehr die männlichen Probanden, die Probanden mit Kindern oder die Studierenden, die vom Alter her den Digital Immigrants entsprechen, \gls{labelMMT} betreiben, desto besser ist ihre Fähigkeit zur Aufmerksamkeitskontrolle. Jedoch unterscheiden sich auch hier die Resultate des \gls{labelMUQ}-Fragebogens zwischen \citeA{Alzahabi2013} und der vorliegende Studie. \citeauthor{Alzahabi2013} brachte einen Mittelwert von $M=4.07$ ($SD=.64$) hervor, welcher nahezu dem Wert von \citeA{Ophir2009} entspricht. Auch hier könnte die Differenz mit der unterschiedlichen Ausführung des \gls{labelMUQ} begründet werden. 

\subsection{Auffällige Resultate in der vorliegenden Arbeit.}
In dieser Arbeit konnte ein signifikanter direkter Zusammenhang zwischen dem \gls{labelMMT} und dem SPANE-N (Fragebogen zur Erfassung von negativen Erfahrungen) hergestellt werden. Dieses Resultat zeigt eine mögliche Tendenz einer Auswirkung von \gls{labelMMT} auf das \gls{labelSWB} auf. Dieser Zusammenhang kann weiter ausgebaut werden, indem eine Aufteilung anhand der stärke der Mediennutzung vorgenommen wird. Ein negativer Zusammenhang zwischen dem \gls{labelMMT} bei Personen die starkes Multitasking betreiben (HMMs) und dem \gls{labelSWB} konnte bei allen getesteten Skalen festgestellt werden. Bei den Probanden, die wenig \gls{labelMMT} betreiben (LMMs), konnte für den $MMI$ ohne SMS ein leichter Zusammenhang zur Skala für die Erfassung von positiven Erfahrungen hergestellt werden. Daraus lässt sich ableiten, dass starkes Medien-Multitasking-Verhalten einen tendenziell negativen Einfluss und geringes \gls{labelMMT} einen eher positiven Einfluss auf das subjektive Wohlbefinden und somit das Glück haben. Zu beachten dabei ist, dass normales oder mittleres Multitasking (NMMs) keinen Einfluss auf das \gls{labelSWB} zu haben scheint. Eine weitere Auffälligkeit im Bezug auf das Glück ist die Unterteilung anhand des Geschlechts. Gemäss den vorliegenden Resultaten scheint es, dass einen geringen negativen Zusammenhang zwischen dem Medien-Multitasking von Frauen und dem subjektiven Wohlbefinden gibt. In der bisherigen Forschung konnte festgestellt werden, dass weibliche Probanden stärker zu \gls{labelMMT} neigen als die männlichen Probanden \cite{Rideout2010}. Dies konnte in der vorliegenden Studie nicht bestätigt werden. Falls die vorliegende Erfassung der Mediennutzung validiert werden könnte, würde dieses Resultat ein möglicher Hinweis auf einen tendenziellen Unterschied zwischen den Geschlechtern hindeuten. Des weiteren konnte ein negativer Zusammenhang zum \gls{labelSWB} bei der Aufteilung anhand des Alters festgestellt werden. Daraus lässt sich ablesen, dass Personen die jünger als 25 Jahre alt sind, stärker auf Medien-Multitasking reagieren als ältere Probanden. Das Alter als weiteres Kriterium könnte auf eine mögliche Reife im Umgang mit Multitasking hinweisen. Dies müsste weiter untersucht werden. Werden die Auswirkungen von Medien-Multitasking in Richtung der Fähigkeit zur Aufmerksamkeitskontrolle näher betrachtet, kann ein beinahe mittlerer Zusammenhang bei den Probanden mit Kindern festgestellt werden. Eine mögliche Erklärung dafür könnte sein, dass Eltern durch ihre Rolle mehr Aufgaben parallel Ausführen müssen und somit Multitasking anwenden. Durch diese häufige Ausübung könnte sich eine Verbesserung der Fähigkeit zur Aufmerksamkeitskontrolle ergeben, da sie für das Grossziehen des Nachwuchses notwendig ist (die Aufmerksamkeit zwischen dem Nachwuchs und der aufgeführten Aufgabe wechseln). Dies Hypothese würde für die Theorie von \citeA{Klingberg2008} sprechen und müsste genauer untersucht werden. Ebenso erstaunlich ist die Auswirkung auf die Fähigkeit zur Aufmerksamkeitskontrolle bei den Männern. Dieses Ergebnis könnte der weitläufigen Meinung, dass Frauen besser sind im Multitasking, wiedersprechen \cite{Oconnell2002}. Durch das Ausüben von Medien-Multitasking könnte sich die Fähigkeit zur Aufmerksamkeitskontrolle schulen und verbessern.

\subsection{Fazit für das Glück und Umgang und Folgerung für die Praxis.}


%####################################################
% Kapitel Methodenkritik
%####################################################
\addsec{Methodenkritik}\label{section.diskussion.methodenkritik}
Die Erfassung der Medien-Nutzung mittels \gls{labelMUQ} und deren Überführung in den \gls{labelMMI} erfolgte in mittels Abschätzung der einzelnen Medien in Minuten pro Tag. In der ursprünglichen Form dieses Fragebogens erfolgte die Abschätzung in Stunden pro Woche. Dies könnte der Grund für die markant unterschiedlichen Resultate der Mediennutzung und des Indexes verglichen mit bestehenden Studien ausmachen \cite{Ophir2009,Alzahabi2013}. Einen weiteren Grund für die Unterschiede könnte in der Umsetzung des \gls{labelMUQ}-Fragebogens liegen. In der Frage nach einer simultanen Nutzung eines Nebenmediums zusammen mit einem Hauptmediums, wurde die Antwort bereits auf die Standardkategorie 'nie' gesetzt. Die Auswahl der Probanden wurden demzufolge auf nie gesetzt, auch wenn sie keine Angaben zu einem Nebenmedium gemacht hatten. Der Grund für dieses Vorgehen sollte der Vereinfachung der komplexen Beantwortung dienen und sollte für zukünftige Umfragen vermieden werden.\\
Weiter wurde in der originalen Version des \gls{labelMUQ} das Hauptmedium SMS mit der Begründung der schwierigen Abschätzung ausgelassen. In der vorliegenden Arbeit wurde SMS als Hauptmedium belassen, da die Abschätzung in Minuten pro Tag erfolgte und danach in Stunden pro Woche hochgerechnet wurde.  Wie aus Tabelle \ref{table.deskrptMedien} hervorgeht, unterscheiden sich die Mittelwerte marginal. Weshalb davon ausgegangen werden kann, dass dieser Einbezug das Ergebnis nicht zu stark verfälschte.\\
Für die Berechnung des \gls{labelMMI} wurden alle Hauptmedien in der Formel gleich gewichtet. Zum Beispiel wurde simultanes Musikhören gleich gewertet wie simultanes Emailschreiben. Hier müsste genauer untersucht werden, ob eine Gewichtung anhand der nötigen Aufmerksamkeitskontrolle Sinn machen würde, um einen repräsentativen Index des Medien-Multitasking zu berechnen.\\
Die Erfassung der Fähigkeit zur Aufmerksamkeitskontrolle wurde anhand einem Selbstbeurteilungsfragebogen durchgeführt. Gemäss \citeA{Greenberg2005} sind Selbstbeurteilungsfragebogen gegenüber Tagebüchern ungenauer und unterliegen stärkeren Schwankungen. Dies gilt genauso für die Erfassung der Mediennutzung.\\
Diese Umfrage berücksichtigte nur Studenten, die entweder an der Universität oder an einer Fachhochschule studierten. Es ist anzunehmen, dass sich die Medien-Nutzung unterschiedlich gestaltet gegenüber den übrigen Alters- und Populationsgruppen. 
%####################################################
% Kapitel Ausblick
%####################################################
\addsec{Ausblick}\label{section.diskussion.ausblick}
Die Ergebnisse dieser Studie liefern eine gute Grundlage für weitere Untersuchungen im Bereich des Medien-Multitasking. Durch die grosse Stichprobe können weitere Zusammenhänge und Eigenschaften des Medien-Multitasking Verhaltens untersucht werden, die in dieser Studie keinen Platz fanden. Die Resultate lassen die Annahme zu, dass es einen Zusammenhang zwischen dem Medien-Multitasking und dem subjektiven Wohlbefinden geben könnte. Dieser Bereich benötigt zusätzliche Forschung und eine vertiefte Auseinandersetzung mit dem Thema. Die fortführende Auseinandersetzung mit der Fähigkeit zur Aufmerksamkeitskontrolle und die Auswirkung von Multitasking auf diese benötigt weitere Ergebnisse. Ein spannendes Thema könnte die Befassung mit möglichen Einflussfaktoren neben der Fähigkeit zur Aufmerksamkeitskontrolle auf das subjektive Wohlbefinden sein. Zum Beispiel könnten soziale Faktoren wie Umgebung, Beziehung oder auch Bildung mögliche Wirkfaktoren für das Multitasking verhalten und die Auswirkung auf das subjektive Wohlbefinden darstellen. Ebenso ist vorstellbar, dass das subjektive Wohlbefinden in weitere Teilbereiche unterteilt wird, um eine allfällige Beeinflussung des Multitasking aufzuzeigen. Eine zentrale Rolle in dieser Diskussion scheint der Langzeitauswirkung von Medien-Multitasking zuzukommen. Wie wirken sich diese Medien-Präferenzen auf unser Gehirn und deren Areale aus. Tendiert unser Gehirn dazu, umgebungsbedingte Einflüsse zu adaptieren oder stossen wir an unsere Kapazitäts- und Leistungsgrenzen, was unsere kognitiven Fähigkeiten anbelangt. Diese notwendigen Untersuchungen sollen dazu beitragen, den künftigen Umgang mit Medien und wie Medien technologisch umgesetzt werden, zu erleichtern. Eine auf unsere Bedürfnisse und Fähigkeiten abgestimmte Medienwelt, die unser Potential anregt und unsere Fähigkeiten erweitert, ist einer von Reizen überfluteten, teilweise überfordernden und stressverursachenden Umwelt definitiv vorzuziehen. 

