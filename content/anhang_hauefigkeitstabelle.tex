%%%%%%%%%%%%%%%%%%%%%%%%%%%%%%%%%%%%%%%%%%%%%%%%%%%%%%%%%%%%%%%%%
%_____________ ___    _____  __      __ 
%\____    /   |   \  /  _  \/  \    /  \  Institute of Applied
%  /     /    ~    \/  /_\  \   \/\/   /  Psychology
% /     /\    Y    /    |    \        /   Zuercher Hochschule 
%/_______ \___|_  /\____|__  /\__/\  /    fuer Angewandte Wissen.
%        \/     \/         \/      \/                           
%%%%%%%%%%%%%%%%%%%%%%%%%%%%%%%%%%%%%%%%%%%%%%%%%%%%%%%%%%%%%%%%%
%
% Project     : Bachelorarbeit
% Title       : 
% File        : Rev. 00
% Date        : 06.12.2013
% Author      : Till J. Ernst
%
%%%%%%%%%%%%%%%%%%%%%%%%%%%%%%%%%%%%%%%%%%%%%%%%%%%%%%%%%%%%%%%%%
\glsresetall

\let\raggedsection\centering 
\chapter{Häufgkeitstabellen}\label{anhang.hauefigkeitstabelle}
\let\raggedsection\raggedright 
\begin{RaggedRight}
%Demographische Daten
\section{Demographische Daten}\label{anhangHaeufigkeit.demoDaten}
%Tabelle Altergruppen
\begin{table}[ht]
    \centering 
    \caption{Soziodemografische Charakteristik der Altersgruppen, Häufigkeitstabelle}
    \begin{tabular}[t]{|r r|R{20mm}|R{20mm}|R{20mm}|R{20mm}|} 
        \hline
        \multicolumn{6}{|c|}{\textbf{Altersgruppen}}\\        
        \multicolumn{2}{|c}{} & \multicolumn{1}{c|}{Häufigkeit} & \multicolumn{1}{c|}{Prozent} & \multicolumn{1}{c|}{Gültige} & \multicolumn{1}{c|}{Kumulierte}\\
        \multicolumn{2}{|c}{} & \multicolumn{1}{c|}{(N)} & \multicolumn{1}{c|}{(\%)} & \multicolumn{1}{c|}{Prozente} & \multicolumn{1}{c|}{Prozente} \\
        \hline       
        Gültig &< 16 & 1 & .1 & .1 & .1\\
        &16 - 20 & 163 & 11.9 & 11.9 & 12.0\\
        &21 - 25 & 771 & 56.4 & 56.4 & 68.4\\
        &26 - 30 & 244 & 17.8 & 17.8 & 86.2\\
        &31 - 35 & 86 & 6.3 & 6.3 & 92.5\\
        &36 - 40 & 30 & 2.2 & 2.2 & 94.7\\
        &41 - 45 & 28 & 2.0 & 2.0 & 96.8\\  
        &46 - 50 & 26 & 1.9 & 1.9 & 98.7\\
        &> 50  & 18 & 1.3 & 1.3 & 100\\
        & Gesamt & 1367 & 100 & 100 & \\
        \hline
    \end{tabular}
    \label{table.sozidemoAlter5}
\end{table}
%Table Zivilstand
\begin{table}[ht] 
    \centering
    \caption{Soziodemografische Charakteristik, Häufigkeitstabelle}
    \begin{tabular}[t]{|r|r|r|r|r|r|r|} 
        \hline
        \multicolumn{5}{|c|}{\textbf{Zivilstand}}\\ 
        \hline       
        \multicolumn{1}{|c}{} & \multicolumn{1}{c|}{Häufigkeit} & \multicolumn{1}{|c|}{Prozent} & \multicolumn{1}{|c|}{Gültige} & \multicolumn{1}{|c|}{Kumulierte}\\
        \multicolumn{1}{|c}{} & \multicolumn{1}{c|}{(N)} & \multicolumn{1}{|c|}{(\%)} & \multicolumn{1}{|c|}{Prozente} & \multicolumn{1}{|c|}{Prozente} \\
        \hline
        keine Angabe & 2 & ,1 & ,1 & ,1\\
        ledig & 1244 & 91,0 & 91,0 & 91,1\\
        verheiratet & 100 & 7,3 & 7,3 & 98,5\\
        getrennt & 5 & ,4 & ,4 & 98,8\\
        geschieden & 16 & 1,2 & 1,2 & 100\\
        Gesamt  & 1367 & 100,0 & 100,0 & \\
        \hline
    \end{tabular}
    \label{table.sozidemoZivil}
\end{table}
\end{RaggedRight}

%MMI
\section{Medien Multitasking Index -- MMI}\label{anhangHaeufigkeit.mmi}
% Table HMM und LMM Häufigkeiten
\begin{table}[H] 
    \centering
    \caption{Charakteristik der Mediennutzung, Häufigkeit}
    \begin{tabular}[t]{|r r|r|r|r|r|} 
        \hline
        \multicolumn{6}{|c|}{\textbf{Mediennutzung basierend auf $MMI_{angepasst}$ und $MMI$}}\\ 
        \hline       
        \multicolumn{2}{|c}{} & \multicolumn{2}{c|}{$MMI_{angepasst}$} & \multicolumn{2}{c|}{$MMI$}\\
        \multicolumn{2}{|c}{} & \multicolumn{1}{c|}{Häufigkeit} & \multicolumn{1}{c|}{Prozent}&\multicolumn{1}{c|}{Häufigkeit} & \multicolumn{1}{c|}{Prozent}\\
        \hline
        Gültig & LMMs & 188 & 13,8 & 181 & 13,2\\
        & HMMs & 202 & 14,8 & 207 & 15,1\\
        & NMMs & 969 & 70,9 & 971 & 71,0\\
        &Gesamt & 1359 & 99,4 & 1359 & 99,4\\
        Fehlend & & 8 & ,6 & 8 & ,6\\
        Gesamt & & 1367 & 100 & 1367 & 100\\
        \hline
    \end{tabular}
    \label{table.deskrptMeediennutzung}
\end{table}

% Sektion ACS
\section{Aufmerksamekeitskontroll-Skala -- ACS}\label{anhangHaeufigkeit.ACS}
% Table ACS
\begin{table}[H] 
    \centering
    \caption{Charakteristik des Aufmerkasmkeitskontroll-Index, Häufigkeit und Verteilung}
    \begin{tabular}[t]{|r r|R{27mm}|} 
        \hline
        \multicolumn{3}{|c|}{\textbf{Aufmerksamkeitskontroll-Skala -- $ACS$}}\\ 
        \hline       
        Gesamtwert (N) & Gültig & 1367\\
        & Fehlend & 0\\
        Mittelwert & $M$ & 53,43\\
        Median & $med$ & 54,00\\
        Standardabweichung & $SD$ & 6,82\\
        Minimum & $Min$ & 34\\
        Maximum & $Max$ & 75\\
        \hline
    \end{tabular}
    \label{table.deskrptAcs}
\end{table}
% SEKTION work
\section{Korrelationen}\label{anhangHaeufigkeit.korrelationen}
Für eine verbesserte Übersicht, wurden die beiden Variablen $MMI_{angepasst}$ und $MMI$ in den folgenden Tabellen zusammengefasst.
%Table Geschlecht MMI SWB
\begin{table}[H] 
    \centering
    \caption{Zusammenhang zwischen Medien-Multitasking und subjektivem Wohlbefinden Aufteilung anhand dem Geschlecht, Korrelationen}
    \begin{tabular}[t]{|l R{15mm}|R{25mm}|R{16mm}|R{15mm}|R{18mm}|R{18mm}|R{18mm}|} 
        \hline
        \multicolumn{8}{|c|}{\textbf{Korrelationen aufgeteilt nach dem Geschlecht}}\\ 
        \hline       
        \multicolumn{3}{|c|}{} & \multicolumn{1}{c|}{$MMI_{ang}$} &\multicolumn{1}{c|}{$FS$} & \multicolumn{3}{c|}{$SPANE$}\\
        \multicolumn{3}{|l|}{Geschlecht} & \multicolumn{1}{c|}{($MMI$)} & \multicolumn{1}{c|}{} &\multicolumn{1}{c|}{$P$} &  \multicolumn{1}{c|}{$N$} & \multicolumn{1}{c|}{$B$}\\
        \hline
        w & $MMI_{ang}$ ($MMI$) & Korrelation nach Pearson & 1 (.987**) & -.064* (-.068*) & -.046 (-.054) & .126** (.119**) & -.101** (-.101**) \\
        & & Signifikanz (2-seitig) & & .050 (.036) & .152 (.095) & .000 (.000) & .002 (.002)\\
        & & $N$ & 955 & 953 & 951 & 951 & 951\\
        \hline
        m & $MMI_{ang}$ ($MMI$) & Korrelation nach Pearson & 1 (.998**) & .046 (.040) & .020 (.017) & .060 (.059) & -.023 (-.024)\\
        & & Signifikanz (2-seitig) & & .358 (.423) & .689 (.733) & .233 (.243) & .645 (.632)\\
        & & $N$ & 401 & 400 & 399 & 399 & 399\\
        \hline
        \multicolumn{2}{|l|}{k.A.} & $N$ & 3 & 3 & 3 & 3 & 3\\
        \hline
        \multicolumn{8}{l}{**. Die Korrelation ist auf dem Niveau von 0.01 (2-seitig) signifikant.}\\
        \multicolumn{8}{l}{*. Die Korrelation ist auf dem Niveau von 0.05 (2-seitig) signifikant.}\\
    \end{tabular}
    \label{table.ergebnis.alter}
\end{table}

