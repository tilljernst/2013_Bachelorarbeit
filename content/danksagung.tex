%%%%%%%%%%%%%%%%%%%%%%%%%%%%%%%%%%%%%%%%%%%%%%%%%%%%%%%%%%%%%%%%%
%_____________ ___    _____  __      __ 
%\____    /   |   \  /  _  \/  \    /  \  Institute of Applied
%  /     /    ~    \/  /_\  \   \/\/   /  Psychology
% /     /\    Y    /    |    \        /   Zuercher Hochschule 
%/_______ \___|_  /\____|__  /\__/\  /    fuer Angewandte Wissen.
%        \/     \/         \/      \/                           
%%%%%%%%%%%%%%%%%%%%%%%%%%%%%%%%%%%%%%%%%%%%%%%%%%%%%%%%%%%%%%%%%
%
% Project     : Bachelorarbeit
% Title       : Danksagung
% File        : danksagung.tex Rev. 00
% Date        : 06.12.2013
% Author      : Till J. Ernst
%
%%%%%%%%%%%%%%%%%%%%%%%%%%%%%%%%%%%%%%%%%%%%%%%%%%%%%%%%%%%%%%%%%
\thispagestyle{empty} 
\let\raggedsection\centering
\chapter*{Danksagung}\label{label.danksagung}
\let\raggedsection\raggedright 
An dieser stelle möchte ich mich bei all denjenigen Personen bedanken, die mich in dem Entstehungsprozess dieser Bachelorarbeit unterstützt und motiviert haben. Insbesondere möchte ich meinem Referenten und Dozenten Prof. Dr. Daniel Süss für seine tatkräftige Unterstützung und die wertvollen Tips danken. Diese ermöglichten mir in dem für mich nicht immer ganz durchsichtigen Umfeld der psychologischen Forschung den Überblick zu bewahren. Durch seine kritischen Inputs hat er mir wertvolle Hinweise zu diesem Thema gegeben und mich motiviert, über meinen Horizont hinaus zu denken und mich der wissenschaftlichen Betrachtungsweise spielerisch zu nähern. Des weiteren möchte ich den Dozenten Silvia Passalacqua und Prof. Dr. Markus Hackenfort für das hilfreiche Methodenkolloquium danken. Hierbei konnte ich von der langjährigen Erfahrung der Dozierenden in Forschungsfragen profitieren und einige nützlich Informationen für meine Arbeit in Erfahrung bringen. Meinen Dank gilt auch John Meister, der als Filialleiter von Athleticum Zürich, die grosszügige Spende für den Wettbewerb im Fragebogen ermöglich hat. Ebenso möchte ich Rivka Honegger für die Durchsicht und die Korrektur danken. Sie hat mir geholfen, die kleinen und die grösseren Fehler ausfindig zu machen, die sich mit der Zeit gekonnt zu verstecken vermochten. Auch gilt meinen Dank Liane Brandt und meiner Arbeitskollegin (sie möchte hier ungenannt bleiben, A.d.Vf.) für die Hilfe bei der Übersetzung der Fragebögen aus dem Englischen. Natürlich gilt mein Dank auch den zahlreichen Kolleginnen und Kollegen, die den Onlinefragebogen als Pre-Tester auf Herz und Niere geprüft haben. Sie haben geholfen, die Stolpersteine und Schwachstellen einer Onlinebefragung zu nehmen. Nicht zuletzt möchte ich meinen Vorgesetzten für das Entgegenkommen und das Verständnis in der nicht immer ganz leichten Zeit danken. Sie erlaubten mir immer wieder ein paar extra Stunden für diese Arbeit herauszuschlagen. Zu guter letzt möchte ich noch meiner Partnerin für die Unterstützung in dieser turbulenter Zeit danken. Da wir beide zur selben Zeit diese Arbeit geschrieben und parallel dazu noch gearbeitet haben, war ich über die gegenseitige emotionale Unterstützung und Motivation sehr froh. 